\documentclass[12pt]{article}
%%%%%%%%%%%%%%%%%%%%%%%%%%%%%%%%%%%%%%%%%%%%%%%%%%%%%%%%%%%%%%%%%%%%%%%%%%%%%%%%%%%%%%%%%%%%%%%%%%%%%%%%%%%%%%%%%%%%%%%%%%%%%%%%%%%%%%%%%%%%%%%%%%%%%%%%%%%%%%%%%%%%%%%%%%%%%%%%%%%%%%%%%%%%%%%%%%%%%%%%%%%%%%%%%%%%%%%%%%%%%%%%%%%%%%%%%%%%%%%%%%%%%%%%%%%%
\usepackage{amsfonts}
\usepackage{eurosym}
\usepackage{geometry}
\usepackage{amsmath,amsthm,amssymb}
\usepackage{ulem} 
\usepackage{graphicx}
\usepackage{comment}
%\usepackage[sort,comma]{natbib}
\usepackage[utf8]{inputenc}
\usepackage{setspace}
\usepackage[backend=biber, style = apa]{biblatex}
\usepackage{placeins} % to separate sections

\usepackage{adjustbox}
\usepackage{array}
\usepackage{multirow}
\usepackage{graphicx}
\usepackage{subcaption}
\usepackage{pifont}
\usepackage{amssymb}
\usepackage{comment}
\usepackage[hang, flushmargin, bottom]{footmisc}
\usepackage{footnotebackref}
\usepackage{xcolor}
\usepackage{hyperref}
\usepackage{booktabs}
\usepackage{pifont}
\usepackage{caption}
\usepackage{float}
\usepackage{todonotes}
\setcounter{MaxMatrixCols}{10}
%TCIDATA{OutputFilter=LATEX.DLL}
%TCIDATA{Version=5.50.0.2960}
%TCIDATA{<META NAME="SaveForMode" CONTENT="1">}
%TCIDATA{BibliographyScheme=BibTeX}
%TCIDATA{LastRevised=Sunday, April 28, 2024 18:12:38}
%TCIDATA{<META NAME="GraphicsSave" CONTENT="32">}
%TCIDATA{Language=American English}

%\setlength{\bibsep}{0.3pt}
\setlength{\textfloatsep}{5pt}
\hypersetup{breaklinks=true,hypertexnames=false,colorlinks=true,citecolor = teal}
\captionsetup{font=normalsize}
\newcommand{\cmark}{\ding{51}}
\def\sym#1{\ifmmode^{#1}\else\(^{#1}\)\fi}
\renewcommand{\thetable}{\Roman{table}}
\geometry{verbose,tmargin=.9in,bmargin=1in,lmargin=.8in,rmargin=.8in,nomarginpar}
\makeatletter
\DeclareTextSymbolDefault{\textquotedbl}{T1}
\theoremstyle{plain}
\newtheorem{thm}{\protect\theoremname}
\theoremstyle{plain}
\newtheorem{prop}[thm]{\protect\propositionname}
\providecommand{\propositionname}{Proposition}
\providecommand{\theoremname}{Theorem}
\makeatother
\providecommand{\propositionname}{Proposition}
\providecommand{\theoremname}{Theorem}
\newtheorem{ass}[thm]{Assumption}
% \input{tcilatex}
\usepackage{tikz}
\usetikzlibrary{shapes.geometric, arrows, positioning}





\addbibresource{references.bib}
\begin{document}






Let $J$ denote the number of firms, and let $\theta \in \{0,1\}$ represent the consumer type where $\theta = 0 $ represent the shoppers and $\theta = 1$ the non-shoppers. A share $\lambda$ is non-shoppers. We denote by $D_i^S$ the demand of searchers and by $D_i^{NS}$  the demand of non-searchers. 

Denote by $\eta_i(p)= \frac{\partial D_i(p)}{\partial p} \frac{p}{D_i(p)}$ the elasticity of types $i$. We assume: 

\[
\eta_{0}(p) > \eta_{1}(p) 
\]

We consider a setting with $J=2$ firms.  
Each firm $j \in \{1,2\}$ sets a price $p_{tj}^t$ in the $t$ stage of the game. 

\paragraph{Profits.}
The profit function for firm $i$ is:
\begin{align}
\Pi_i(P_i, P_{-i}) 
&= \lambda D_i^S(P_{11}^T, P_{12}^T)(P_{11}^T - c_1) 
+ \frac{(1 - \lambda)}{2}
\left[
D_i^{NS}(P_{21}^T, P_{12}^T)(P_{21}^T - c_i),
+   D_i^{NS}(P_{11}^T, P_{22}^T)(P_{11}^T - c_i),
\right]
\end{align}
where $c_i$ is the marginal cost.

The first term corresponds to the profit from non-searchers.  The second and third terms correspond to searchers, depending on whether they request a revised offer from firm 1 or 2. 


 


\paragraph{First-Order Conditions.}
the derivative of profits with respect to firm $1$’s own price $p_{11}^T$ is:
\begin{align}
\frac{\partial \Pi_1}{\partial p_{11}^T} 
&= \lambda \left[
\frac{\partial D_1^S(p_{11}^T, p_{12}^T)}{\partial p_{11}^T}(p_{11}^T - c_1)
+ D_1^S(p_{11}^T, p_{12}^T)
\right] \notag \\
&\quad + \frac{(1 - \lambda)}{2}
\left[
\frac{\partial D_1^{NS}(p_{11}^T, p_{22}^T)}{\partial p_{11}^T}(p_{11}^T - c_1)
+ D_1^{NS}(p_{11}^T, p_{22}^T)\right] \notag \\
& \quad + \frac{(1 - \lambda)}{2}
\left[
\frac{\partial D_1^{NS}(p_{11}^T, p_{22}^T)}{\partial p_{22}^T}(p_{11}^T - c_1) \frac{\partial P_{22}^T}{\partial P_{11}^T} +
\underbrace{\frac{\partial D_1^{NS}(p_{21}^T, p_{12}^T)}{\partial p_{21}^T}(p_{21}^T - c_1) \frac{\partial P_{21}^T}{\partial P_{11}^T}}_{=0} \right] 
\end{align}

 Define $g_i(p) = \frac{\partial D_i^S(p)}{\partial p_{i}}(p_i - c_i) + D_i^S(p)$ and  $\tilde g_i(p) = \frac{\partial D_i^{NS}(p)}{\partial p_{i}}(p_i - c_i) + D_i^{NS}(p)$

Then the FOC can be written as: 



\begin{align}
\frac{\partial \Pi_1}{\partial p_{11}^T} 
= \lambda g_1(p_{11}^T, p_{12}^T) 
+ \frac{(1 - \lambda)}{2}
\left[ g_1(p_{11}^T, p_{22}^T)
+ 
\frac{\partial D_1^{NS}(p_{11}^T, p_{22}^T)}{\partial p_{22}^T}(p_{11}^T - c_1) \frac{\partial P_{22}^T}{\partial P_{11}^T}  \right] 
\end{align}



























\end{document}