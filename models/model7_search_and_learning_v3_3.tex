\documentclass[12pt]{article}
%%%%%%%%%%%%%%%%%%%%%%%%%%%%%%%%%%%%%%%%%%%%%%%%%%%%%%%%%%%%%%%%%%%%%%%%%%%%%%%%%%%%%%%%%%%%%%%%%%%%%%%%%%%%%%%%%%%%%%%%%%%%%%%%%%%%%%%%%%%%%%%%%%%%%%%%%%%%%%%%%%%%%%%%%%%%%%%%%%%%%%%%%%%%%%%%%%%%%%%%%%%%%%%%%%%%%%%%%%%%%%%%%%%%%%%%%%%%%%%%%%%%%%%%%%%%
\usepackage{amsfonts}
\usepackage{eurosym}
\usepackage{geometry}
\usepackage{amsmath,amsthm,amssymb}
\usepackage{ulem} 
\usepackage{graphicx}
\usepackage{comment}
%\usepackage[sort,comma]{natbib}
\usepackage[utf8]{inputenc}
\usepackage{setspace}
\usepackage[backend=biber, style = apa]{biblatex}
\usepackage{placeins} % to separate sections

\usepackage{adjustbox}
\usepackage{array}
\usepackage{multirow}
\usepackage{graphicx}
\usepackage{subcaption}
\usepackage{pifont}
\usepackage{amssymb}
\usepackage{comment}
\usepackage[hang, flushmargin, bottom]{footmisc}
\usepackage{footnotebackref}
\usepackage{xcolor}
\usepackage{hyperref}
\usepackage{booktabs}
\usepackage{pifont}
\usepackage{caption}
\usepackage{float}
\usepackage{todonotes}
\setcounter{MaxMatrixCols}{10}


%\setlength{\bibsep}{0.3pt}
\setlength{\textfloatsep}{5pt}
\hypersetup{breaklinks=true,hypertexnames=false,colorlinks=true,citecolor = teal}
\captionsetup{font=normalsize}
\newcommand{\cmark}{\ding{51}}
\def\sym#1{\ifmmode^{#1}\else\(^{#1}\)\fi}
\renewcommand{\thetable}{\Roman{table}}
\geometry{verbose,tmargin=.9in,bmargin=1in,lmargin=.8in,rmargin=.8in,nomarginpar}
\makeatletter
\DeclareTextSymbolDefault{\textquotedbl}{T1}
\theoremstyle{plain}
\newtheorem{thm}{\protect\theoremname}
\theoremstyle{plain}
\newtheorem{prop}[thm]{\protect\propositionname}
\providecommand{\propositionname}{Proposition}
\providecommand{\theoremname}{Theorem}
\makeatother
\providecommand{\propositionname}{Proposition}
\providecommand{\theoremname}{Theorem}
\newtheorem{ass}[thm]{Assumption}
% \input{tcilatex}
\usepackage{tikz}
\usetikzlibrary{shapes.geometric, arrows, positioning}


\addbibresource{../references.bib}
\begin{document}

This document presents a model of Nash-Bertrand competition that includes selection, search and assymmetric information. The assymmetric information is modeled following \textcite{cosconati_competing_2025}. 


%% timing of the game 
A buyer has a private type, firms observe a noisy signal about the buyer's type and make their first stage offers. Consumers decide whether to accept one of the initial offers or to request revised offers (search).  In a second stage we assume that firms are able to observe the offers of their rivals and make their second stage offers.

% what happens between stages: learning about other firms signals and also updating priors about the consumer. 

%% selection interacts with search 

In the model search interacts with assymmetric information and with selection. 
The interaction between search and selection is given by the joint density of search costs and types. Buyers decide to search based on two forces: their search cost and also the expected benefit from searching which will depend on their type. 
% search interacts with common value auction 

Moreover when consumers search, firms are able to observe the offers of their rivals, which reveals information about the signals their rivals received in the first stage. Hence, searching can increase competition by reducing the assymmetric information between firms. This intuition is similar to \textcite{cosconati_competing_2025} who finds that pooling information across firms reduces informational rents and increases competition. 
 








Consumer search and selection are prevalent in selection markets. 

If search propensity and private information are not independent, then searching might reveal something about the consumer's private information. 

Particularly, if search costs are correlated with private information then searching could reveal information about the consumer's type and improve or worsen selection issues. For example if healthy consumers search then they can get better quotes, which could help with adverse selection. 

What happens when there is interaction between search and selection. 

 

\section{Model}
There are $J$ insurers in the market. 

Consumer type is given by $\theta$ which is distributed with a pdf $f_0(\theta)$.

Firms do not observe the true type, firm $j$ receives a signal $\hat \theta_j \sim \mathcal{N}(\theta, \sigma_j^2)$ with density $\phi(\hat \theta_j; \theta, \sigma_j)$. The joint density of signals is given by $\phi(\hat \theta; \theta) = \prod_{j =1 }^{J} \phi(\hat \theta_j; \theta, \sigma_j)$. Also we use the notation  $\phi(\hat \theta_{-j}; \theta) = \prod_{k \neq j }^{} \phi(\hat \theta_k; \theta, \sigma_k)$

Consumer utility is given by: $u_{ij} = \gamma(\theta_i) F_j(\theta) + \xi_j + \beta r_j +  \epsilon_{ij}$, where $F_j(\theta)$ is the flow payment of firm $j$ for consumer type $\theta$, $r_j$ is the risk-rating of firm $j$ and $\xi_j$ is a firm-specific unobserved characteristic. We assume that $\epsilon_{ij}$ is distributed Gumbel.

The consumer has search cost $s_i$. The joint distribution of search costs and consumer types is given by $F(s, \theta)$. We denote by $S = 1 $ if the consumer searches and $S=0$ otherwise.

Following \textcite{cosconati_competing_2025} we assume that firms pricing strategy is linear on its posterior expectation of the consumer type. But we allow for different pricing strategies for the first and second stage.  Specifically, firm $j$ sets prices according to: 

\begin{align}
    F_{j}^{1T}(\hat \theta_j) = \alpha_j + \beta_j E[\theta \mid \hat \theta_j, D = j,  S=0] \\ 
    F_{j}^{2T}(\hat \theta) = \alpha_j^{2T} + \beta_j^{2T} E[\theta \mid (\hat \theta_j, \hat\theta_{-j}), D = j,  S=1] 
\end{align}

note that prices in the second stage depend on the signals of all firms, since all firms observe the offers made in the first stage. 

Given signals which determine the offers, consumers of type $\theta$ who do not search have a demand for firm $j$ given by: 

\begin{align}
    Pr( D = j \mid \hat \theta, \theta, S=0) = \frac{\exp( \gamma(\theta) F_j^{1T}(\hat \theta_j) + \xi_j + \beta r_j )}{1+ \sum_{k =1 }^J \exp(\gamma(\theta) F_k^{1T}(\hat \theta_k) + \xi_k + \beta r_k )}
\end{align}

and the probability that the a consumer who searches and is of type $\theta$ chooses firm $j$ is given by: 

\begin{align}
    Pr( D = j \mid \hat \theta, \theta, S=1) = \frac{\exp( \gamma(\theta) F_j^{2T}(\hat \theta) + \xi_j + \beta r_j )}{1+ \sum_{k =1 }^J \exp(\gamma(\theta) F_k^{2T}(\hat \theta) + \xi_k + \beta r_k )}
\end{align}


 

The consumer decides whether to search. Define $U_{i0}(\theta, \hat \theta) = \max_j u_{ij}(F_j^{1T}(\hat \theta_j))$ and $U_{i1}(\theta, \hat \theta) = \max_j u_{ij}(F_j^{2T}(\hat \theta))$ he searches if\footnote{\textcolor{red}{FIX THIS PART, THE CONSUMER SHOULD CHOOSE ONLY TAKING INTO ACCOUNT THE UTILITY WITHOUT THE IDIOSINCRATIC ERROR. }}: 

\begin{align}
    \underbrace{U_{i1}(\theta, \hat \theta)  - U_{i0}(\theta, \hat \theta)}_{\equiv G(\theta, \hat \theta)}> s_i
\end{align}

hence the search probability given the signals and his type is given by: 

\begin{align}
    \pi_1(\theta, \hat \theta) = F_{s\mid \theta}(G(\theta, \hat \theta))  
\end{align}

%%%%%%%%%%%%%%%%%%%%%%%%%%%%%%%%%%%%%%%%%%%%%%%%%%%%%%%%%%%%%%%%%%%%%%%%%


The conditional demand for non-searchers in the first stage is given by:

\begin{align}
    D_{j0}(\theta, \hat \theta_j) &= Pr( D = j \mid \hat \theta_j, \theta, S=0) 
    = \int_{\hat \theta_{-j}} Pr( D = j \mid \hat \theta, \theta, S=0)
    % [1-\pi_1(\theta, (\hat \theta_{-j}, \hat \theta_j))]
      \phi(\hat \theta_{-j}; \theta) d\hat \theta_{-j} \notag \\
    &= \int_{\hat \theta_{-j}} \frac{\exp( \gamma(\theta) F_j^{1T}(\hat \theta_j) + \xi_j + \beta r_j )}{1+ \sum_{k =1 }^J \exp(\gamma(\theta) F_k^{1T}(\hat \theta_k) + \xi_k + \beta r_k )}
    % [1-\pi_1(\theta, (\hat \theta_{-j}, \hat \theta_j))] 
    \phi(\hat \theta_{-j}; \theta) d\hat \theta_{-j}
\end{align}

% \vspace{2cm}
%
% \begin{align}
%     D_{j0}(\theta, \hat \theta_j) &= Pr( D = j, S = 0  \mid \hat \theta_j, \theta, S=0) 
%     = \int_{\hat \theta_{-j}} Pr( D = j,  S = 0 \mid \hat \theta, \theta) [1-\pi_1(\theta, (\hat \theta_{-j}, \hat \theta_j))] \phi(\hat \theta_{-j}; \theta) d\hat \theta_{-j} \notag \\
%     &= \int_{\hat \theta_{-j}} \frac{\exp( \gamma(\theta) F_j^{1T}(\hat \theta_j) + \xi_j + \beta r_j )}{1+ \sum_{k =1 }^J \exp(\gamma(\theta) F_k^{1T}(\hat \theta_k) + \xi_k + \beta r_k )} [1-\pi_1(\theta, (\hat \theta_{-j}, \hat \theta_j))] \phi(\hat \theta_{-j}; \theta) d\hat \theta_{-j}
% \end{align}
%
% \vspace{2cm}

%Then the conditional demand for searchers in the first stage is given by: 

%\begin{align}
%    D_{j1} (\theta, \hat \theta) = Pr(D=j, S=1 \mid \theta, \hat \theta) =  \pi_1(\theta, \hat \theta) \cdot Pr( D = j \mid \hat \theta, \theta, S=1) 
%\end{align}

and the conditional demand for searchers is given by:  

\begin{align} 
    D_{j1}(\theta, \hat \theta_j) = Pr(D=j | S =1, \hat  \theta_j, \theta) =\int_{\hat \theta_{-j}}  Pr(D = j \mid S =1, \hat \theta, \theta) 
    \phi(\hat \theta_{-j}; \theta)  d\hat \theta_{-j}
    %\phi(\hat \theta_{-j}; \theta) f_0(\theta) d\theta d\hat \theta_{-j}
\end{align}

Also define 
\begin{align}
    \hat D_{j0}(\theta, \hat \theta_j) &= Pr( D = j, S= 0  \mid \hat \theta_j, \theta) 
    = \int_{\hat \theta_{-j}} Pr( D = j \mid \hat \theta, \theta, S=0) [1-\pi_1(\theta, (\hat \theta_{-j}, \hat \theta_j))]
      \phi(\hat \theta_{-j}; \theta) d\hat \theta_{-j} \notag \\
    &= \int_{\hat \theta_{-j}} \frac{\exp( \gamma(\theta) F_j^{1T}(\hat \theta_j) + \xi_j + \beta r_j )}{1+ \sum_{k =1 }^J \exp(\gamma(\theta) F_k^{1T}(\hat \theta_k) + \xi_k + \beta r_k )}
     [1-\pi_1(\theta, (\hat \theta_{-j}, \hat \theta_j))] 
    \phi(\hat \theta_{-j}; \theta) d\hat \theta_{-j}
\end{align}
and similarly for searchers define:

%\begin{align}
%    \hat D_{j1}(\theta, \hat \theta_j) &= Pr( D = j, S= 1  \mid \hat \theta_j, \theta) 
%    = \int_{\hat \theta_{-j}} Pr( D = j \mid \hat \theta, \theta, S=1) \pi_1(\theta, (\hat \theta_{-j}, \hat \theta_j))
%      \phi(\hat \theta_{-j}; \theta) d\hat \theta_{-j} \notag \\
%    &= \int_{\hat \theta_{-j}} \frac{\exp( \gamma(\theta) F_j^{1T}(\hat \theta_j) + \xi_j + \beta r_j )}{1+ \sum_{k =1 }^J \exp(\gamma(\theta) F_k^{1T}(\hat \theta_k) + \xi_k + \beta r_k )}
    % [1-\pi_1(\theta, (\hat \theta_{-j}, \hat \theta_j))] 
    %\phi(\hat \theta_{-j}; \theta) d\hat \theta_{-j}
%\end{align}


\begin{align}
    \hat D_{j1}(\theta, \hat \theta) &= Pr( D = j, S= 1  \mid \hat \theta, \theta) 
    =  Pr( D = j \mid \hat \theta, \theta, S=1) \pi_1(\theta, (\hat \theta_{-j}, \hat \theta_j))
       \notag \\
    &= \frac{\exp( \gamma(\theta) F_j^{2T}(\hat \theta) + \xi_j + \beta r_j )}{1+ \sum_{k =1 }^J \exp(\gamma(\theta) F_k^{2T}(\hat \theta) + \xi_k + \beta r_k )}
    \pi_1(\theta, (\hat \theta_{-j}, \hat \theta_j))   
\end{align}

 

Then the posterior for non-searchers is\footnote{See section \ref{sec:non_posterior} for the derivation}:

\begin{align}
E\!\left(\theta \mid \hat\theta_j, D=j, S=0\right)
&=
\frac{\displaystyle
        \int \theta \,
             \widehat D_{j0}(\theta,\hat\theta_j)\,
             \varphi\!\left(\hat\theta_j;\theta,\sigma_j\right)\,
             f_0(\theta)\, d\theta}
       {\displaystyle
        \int
             \widehat D_{j0}(\theta',\hat\theta_j)\,
             \varphi\!\left(\hat\theta_j;\theta',\sigma_j\right)\,
             f_0(\theta')\, d\theta'} .
\label{eq:post0_final}
\end{align}

 


and the posterior mean given that the consumer searches is given by\footnote{See section \ref{sec:ser_posterior}}: 


\begin{align}
E\!\left(\theta \mid \hat\theta, D=j, S=1\right)
&=
\frac{\displaystyle
        \int \theta \,
             \hat D_{j1}(\theta,\hat\theta)\,
             \phi\!\left(\hat\theta;\theta\right)\,
             f_0(\theta)\, d\theta}
     {\displaystyle
        \int
             \hat D_{j1}(\theta',\hat\theta)\,
             \phi\!\left(\hat\theta;\theta'\right)\,
             f_0(\theta')\, d\theta'} .
\label{eq:post1_final}
\end{align}
 



We use $\alpha^{1T}, \beta^{1T}, \alpha^{2T}, \beta^{2T}$ to denote the vector of parameters for all firms. The profits of fimr $j$ are: 


\begin{align}
    \pi_j(\alpha^{1T}, \beta^{1T}, \alpha^{2T}, \beta^{2T}) 
    &= \int_\theta \int_{\hat \theta} (F_j^{1T}(\hat \theta_j) - k_j \theta) Pr(D = j, S = 0 \mid \theta, \hat \theta) \phi(\hat \theta; \theta) f_0(\theta) d\theta d\hat \theta \notag \\ 
    & \quad   + \int_\theta \int_{\hat \theta} (F_j^{2T}(\hat \theta) - k_j \theta) Pr(D = j, S= 1\mid \theta, \hat \theta)  \phi(\hat \theta; \theta) f_0(\theta) d\theta d\hat \theta
\end{align}

we can use the fact that: $Pr(D = j, S = 0 \mid \theta, \hat \theta) = Pr(D = j \mid \theta, \hat \theta, S=0) \cdot (1-\pi_1(\theta, \hat \theta))$ and $Pr(D = j, S = 1 \mid \theta, \hat \theta) = Pr(D = j \mid \theta, \hat \theta, S=1) \cdot \pi_1(\theta, \hat \theta)$ to rewrite the profits as:

\begin{align}
    \pi_j(\alpha^{1T}, \beta^{1T}, \alpha^{2T}, \beta^{2T}) 
    &= \int_\theta \int_{\hat \theta} (F_j^{1T}(\hat \theta_j) - k_j \theta)  Pr(D = j \mid \theta, \hat \theta, S=0) \cdot (1-\pi_1(\theta, \hat \theta)) \phi(\hat \theta; \theta) f_0(\theta) d\theta d\hat \theta \notag \\ 
    & \quad   + \int_\theta \int_{\hat \theta} (F_j^{2T}(\hat \theta) - k_j \theta)  Pr(D = j \mid \theta, \hat \theta, S=1) \cdot \pi_1(\theta, \hat \theta) \phi(\hat \theta; \theta) f_0(\theta) d\theta d\hat \theta
\end{align}


\subsection{Analysis}


Your model introduces a dynamic screening mechanism into a selection market where the act of searching does more than just uncover a better price; it fundamentally alters the information structure of the equilibrium. This transformation occurs through two simultaneous channels that determine whether search ultimately helps or hurts market efficiency.

First, for consumers who search, there is an **information aggregation channel**. By soliciting second-stage offers, searchers effectively compel firms to reveal their private signals through their posted prices. Since all competing firms observe these initial offers, the information asymmetry *between* firms vanishes for the searcher pool. This eliminates the "winner's curse"—the fear of winning a customer only because rivals saw a risk factor you missed—and allows firms to price aggressively based on a pooled, highly precise estimate of the consumer's "true" risk. This mechanism effectively replicates the benefits of a centralized information bureau for the subset of consumers who search.

Second, there is a **screening channel** driven by the correlation between search costs and risk types. The decision to search sorts consumers into two distinct pools with different average risk profiles. If low-risk consumers are more likely to search, perhaps due to lower search costs, they successfully "cream-skim" themselves out of the general pool. They benefit doubly: first from the competitive, precise pricing of the second stage, and second by signaling they are low-risk. However, this creates a pecuniary externality on those who remain. As low-risk consumers exit, the pool of non-searchers deteriorates, forcing firms to raise first-stage prices to cover the higher expected cost of this residual "adverse" group. Conversely, if high-risk consumers are the ones searching, the act of searching itself becomes a negative signal. This "signaling trap" can backfire, causing firms to offer worse terms to searchers and potentially collapsing the search market entirely. Thus, search "helps" when it efficiently identifies low-risk drivers but "hurts" when it concentrates bad risk in the stagnant pool of non-searchers.




\newpage 

\section{Derivations}
\subsection{Non-searchers posterior}\label{sec:non_posterior}% Posterior for non-searchers: step-by-step derivation

\begin{align}
E\!\left(\theta \mid \hat\theta_j, D=j, S=0\right)
&= \int \theta \, f\!\left(\theta \mid \hat\theta_j, D=j, S=0\right)\, d\theta 
 \quad \quad \quad \text{(definition of conditional expectation)} \label{eq:post0_def}\\ 
&= \int \theta \,
   \frac{f\!\left(\theta, D=j, S=0 \mid \hat\theta_j\right)}
        {\Pr\!\left(D=j, S=0 \mid \hat\theta_j\right)}\, d\theta
\quad \quad \quad \text{(Bayes' rule)} \label{eq:post0_bayes}\\
&= \int \theta \,
   \frac{\Pr\!\left(D=j, S=0 \mid \theta, \hat\theta_j\right)\,
         f\!\left(\theta \mid \hat\theta_j\right)}
        {\Pr\!\left(D=j, S=0 \mid \hat\theta_j\right)}\, d\theta
\quad \quad \quad  \text{(joint = prob $\times$ density)} \label{eq:post0_chain}\\
&= \frac{
        \int \theta \,
             \Pr\!\left(D=j, S=0 \mid \theta, \hat\theta_j\right)\,
             f\!\left(\theta \mid \hat\theta_j\right)\, d\theta}
       {\int
             \Pr\!\left(D=j, S=0 \mid \theta', \hat\theta_j\right)\,
             f\!\left(\theta' \mid \hat\theta_j\right)\, d\theta'}
\quad \quad \quad  \text{(factor common denominator outside integral).}
\label{eq:post0_ratio}
\end{align}

Now use Bayes' rule for the signal posterior:
\begin{align}
f\!\left(\theta \mid \hat\theta_j\right)
&= \frac{f\!\left(\hat\theta_j \mid \theta\right)\,f_0(\theta)}
         {\displaystyle\int f\!\left(\hat\theta_j \mid \theta'\right)\,f_0(\theta')\,d\theta'}
 = \frac{\varphi\!\left(\hat\theta_j;\theta,\sigma_j\right)\,f_0(\theta)}
         {\displaystyle\int \varphi\!\left(\hat\theta_j;\theta',\sigma_j\right)\,f_0(\theta')\,d\theta'} .
\label{eq:signal_posterior}
\end{align}

Substitute \eqref{eq:signal_posterior} into \eqref{eq:post0_ratio}. The normalizing
denominator in \eqref{eq:signal_posterior} cancels between numerator and denominator
of \eqref{eq:post0_ratio}, giving
\begin{align}
E\!\left(\theta \mid \hat\theta_j, D=j, S=0\right)
&=
\frac{\displaystyle
        \int \theta \,
             \Pr\!\left(D=j, S=0 \mid \theta, \hat\theta_j\right)\,
             \varphi\!\left(\hat\theta_j;\theta,\sigma_j\right)\,
             f_0(\theta)\, d\theta}
       {\displaystyle
        \int
             \Pr\!\left(D=j, S=0 \mid \theta', \hat\theta_j\right)\,
             \varphi\!\left(\hat\theta_j;\theta',\sigma_j\right)\,
             f_0(\theta')\, d\theta'} .
\label{eq:post0_phi}
\end{align}

It is convenient to introduce the shorthand
\begin{align}
\widehat D_{j0}(\theta,\hat\theta_j)
&:= \Pr\!\left(D=j, S=0 \mid \theta, \hat\theta_j\right),
\end{align}
in which case \eqref{eq:post0_phi} becomes
\begin{align}
E\!\left(\theta \mid \hat\theta_j, D=j, S=0\right)
&=
\frac{\displaystyle
        \int \theta \,
             \widehat D_{j0}(\theta,\hat\theta_j)\,
             \varphi\!\left(\hat\theta_j;\theta,\sigma_j\right)\,
             f_0(\theta)\, d\theta}
       {\displaystyle
        \int
             \widehat D_{j0}(\theta',\hat\theta_j)\,
             \varphi\!\left(\hat\theta_j;\theta',\sigma_j\right)\,
             f_0(\theta')\, d\theta'} .
\label{eq:post0_final}
\end{align}



\subsection{Searchers posterior}\label{sec:ser_posterior}

We want the posterior mean when the consumer searches and firms observe the
full signal vector $\hat\theta $:


\begin{align}
E\!\left(\theta \mid \hat\theta, D=j, S=1\right)
&= \int \theta \,
      f\!\left(\theta \mid \hat\theta, D=j, S=1\right)\, d\theta
&&\text{(definition of conditional expectation)}
\label{eq:post1_def}\\
&= \int \theta \,
   \frac{f\!\left(\theta, D=j, S=1 \mid \hat\theta\right)}
        {\Pr\!\left(D=j, S=1 \mid \hat\theta\right)}\, d\theta
&&\text{(Bayes' rule)}
\label{eq:post1_bayes}\\
&= \int \theta \,
   \frac{\Pr\!\left(D=j, S=1 \mid \theta,\hat\theta\right)\,
         f\!\left(\theta \mid \hat\theta\right)}
        {\Pr\!\left(D=j, S=1 \mid \hat\theta\right)}\, d\theta
&&\text{(joint = prob $\times$ density)}
\label{eq:post1_chain}\\[0.75em]
&=
\frac{\displaystyle
        \int \theta \,
             \Pr\!\left(D=j, S=1 \mid \theta,\hat\theta\right)\,
             f\!\left(\theta \mid \hat\theta\right)\, d\theta}
     {\displaystyle
        \int
             \Pr\!\left(D=j, S=1 \mid \theta',\hat\theta\right)\,
             f\!\left(\theta' \mid \hat\theta\right)\, d\theta'}
&&\text{(factor common denominator).}
\label{eq:post1_ratio}
\end{align}

Now apply Bayes' rule to the signal posterior:
\begin{align}
f\!\left(\theta \mid \hat\theta\right)
&=
\frac{f\!\left(\hat\theta \mid \theta\right)\,f_0(\theta)}
     {\displaystyle\int f\!\left(\hat\theta \mid \theta'\right)\,f_0(\theta')\,d\theta'}
\label{eq:signal_post_search_1}\\[0.5em]
&=
\frac{\phi\!\left(\hat\theta;\theta\right)\,f_0(\theta)}
     {\displaystyle\int \phi\!\left(\hat\theta;\theta'\right)\,f_0(\theta')\,d\theta'} ,
\label{eq:signal_post_search_2}
\end{align}

Substituting \eqref{eq:signal_post_search_2} into \eqref{eq:post1_ratio}, the normalising
denominator in \eqref{eq:signal_post_search_2} cancels between numerator and denominator
of \eqref{eq:post1_ratio}, yielding
\begin{align}
E\!\left(\theta \mid \hat\theta, D=j, S=1\right)
&=
\frac{\displaystyle
        \int \theta \,
             \Pr\!\left(D=j, S=1 \mid \theta,\hat\theta\right)\,
             \phi\!\left(\hat\theta;\theta\right)\,
             f_0(\theta)\, d\theta}
     {\displaystyle
        \int
             \Pr\!\left(D=j, S=1 \mid \theta',\hat\theta\right)\,
             \phi\!\left(\hat\theta;\theta'\right)\,
             f_0(\theta')\, d\theta'} .
\label{eq:post1_phi}
\end{align}

\begin{align}
E\!\left(\theta \mid \hat\theta, D=j, S=1\right)
&=
\frac{\displaystyle
        \int \theta \,
             \hat D_{j1}(\theta,\hat\theta)\,
             \phi\!\left(\hat\theta;\theta\right)\,
             f_0(\theta)\, d\theta}
     {\displaystyle
        \int
             \hat D_{j1}(\theta',\hat\theta)\,
             \phi\!\left(\hat\theta;\theta'\right)\,
             f_0(\theta')\, d\theta'} .
\label{eq:post1_final}
\end{align}


\end{document}