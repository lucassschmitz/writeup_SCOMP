\documentclass[12pt]{article}
%%%%%%%%%%%%%%%%%%%%%%%%%%%%%%%%%%%%%%%%%%%%%%%%%%%%%%%%%%%%%%%%%%%%%%%%%%%%%%%%%%%%%%%%%%%%%%%%%%%%%%%%%%%%%%%%%%%%%%%%%%%%%%%%%%%%%%%%%%%%%%%%%%%%%%%%%%%%%%%%%%%%%%%%%%%%%%%%%%%%%%%%%%%%%%%%%%%%%%%%%%%%%%%%%%%%%%%%%%%%%%%%%%%%%%%%%%%%%%%%%%%%%%%%%%%%
\usepackage{amsfonts}
\usepackage{eurosym}
\usepackage{geometry}
\usepackage{amsmath,amsthm,amssymb}
\usepackage{ulem} 
\usepackage{graphicx}
\usepackage{comment}
%\usepackage[sort,comma]{natbib}
\usepackage[utf8]{inputenc}
\usepackage{setspace}
\usepackage[backend=biber, style = apa]{biblatex}
\usepackage{placeins} % to separate sections

\usepackage{adjustbox}
\usepackage{array}
\usepackage{multirow}
\usepackage{graphicx}
\usepackage{subcaption}
\usepackage{pifont}
\usepackage{amssymb}
\usepackage{comment}
\usepackage[hang, flushmargin, bottom]{footmisc}
\usepackage{footnotebackref}
\usepackage{xcolor}
\usepackage{hyperref}
\usepackage{booktabs}
\usepackage{pifont}
\usepackage{caption}
\usepackage{float}
\usepackage{todonotes}
\setcounter{MaxMatrixCols}{10}


%\setlength{\bibsep}{0.3pt}
\setlength{\textfloatsep}{5pt}
\hypersetup{breaklinks=true,hypertexnames=false,colorlinks=true,citecolor = teal}
\captionsetup{font=normalsize}
\newcommand{\cmark}{\ding{51}}
\def\sym#1{\ifmmode^{#1}\else\(^{#1}\)\fi}
\renewcommand{\thetable}{\Roman{table}}
\geometry{verbose,tmargin=.9in,bmargin=1in,lmargin=.8in,rmargin=.8in,nomarginpar}
\makeatletter
\DeclareTextSymbolDefault{\textquotedbl}{T1}
\theoremstyle{plain}
\newtheorem{thm}{\protect\theoremname}
\theoremstyle{plain}
\newtheorem{prop}[thm]{\protect\propositionname}
\providecommand{\propositionname}{Proposition}
\providecommand{\theoremname}{Theorem}
\makeatother
\providecommand{\propositionname}{Proposition}
\providecommand{\theoremname}{Theorem}
\newtheorem{ass}[thm]{Assumption}
% \input{tcilatex}
\usepackage{tikz}
\usetikzlibrary{shapes.geometric, arrows, positioning}


\addbibresource{../references.bib}
\begin{document}


Consumer search and selection are prevalent in selection markets. 

If search propensity and private information are not independent, then searching might reveal something about the consumer's private information. 

Particularly, if search costs are correlated with private information then searching could reveal information about the consumer's type and improve or worsen selection issues. For example if healthy consumers search then they can get better quotes, which could help with adverse selection. 

What happens when there is interaction between search and selection. 

Setting: 
\begin{itemize}
    \item Common value auction 
    \item Selection 
    \item Search cost
\end{itemize}


There are $J$ insurers in the market. 

Consumer type is given by $\theta$ which is distributed with a pdf $f_0(\theta)$.

Firms do not observe the true type, firm $j$ receives a signal $\hat \theta_j \sim \mathcal{N}(\theta, \sigma_j^2)$ with density $\phi(\hat \theta_j; \theta, \sigma_j)$. The joint density of signals is given by $\phi(\hat \theta; \theta) = \prod_{j =1 }^{J} \phi(\hat \theta_j; \theta, \sigma_j)$. Also we use the notation  $\phi(\hat \theta_{-j}; \theta) = \prod_{k \neq j }^{} \phi(\hat \theta_k; \theta, \sigma_k)$

Consumer utility is given by: $u_{ij} = \gamma(\theta_i) F_j(\theta) + \xi_j + \beta r_j +  \epsilon_{ij}$, where $F_j(\theta)$ is the flow payment of firm $j$ for consumer type $\theta$, $r_j$ is the risk-rating of firm $j$ and $\xi_j$ is a firm-specific unobserved characteristic. We assume that $\epsilon_{ij}$ is distributed Gumbel.

The consumer has search cost $s_i$. The joint distribution of search costs and consumer types is given by $F(s, \theta)$. We denote by $S = 1 $ if the consumer searches and $S=0$ otherwise.

Following \textcite{cosconati_competing_2025} we assume that firms pricing strategy is linear on its posterior expectation of the consumer type. But we allow for different pricing strategies for the first and second stage.  Specifically, firm $j$ sets prices according to: 

\begin{align}
    F_{j}^{1T}(\hat \theta_j) = \alpha_j + \beta_j E[\theta \mid \hat \theta_j, D = j,  S=0] \\ 
    F_{j}^{2T}(\hat \theta) = \alpha_j^{2T} + \beta_j^{2T} E[\theta \mid (\hat \theta_j, \hat\theta_{-j}), D = j,  S=1] 
\end{align}

note that prices in the second stage depend on the signals of all firms, since all firms observe the offers made in the first stage. 

Given signals which determine the offers, consumers of type $\theta$ who do not search have a demand for firm $j$ given by: 

\begin{align}
    Pr( D = j \mid \hat \theta, \theta, S=0) = \frac{\exp( \gamma(\theta) F_j^{1T}(\hat \theta_j) + \xi_j + \beta r_j )}{1+ \sum_{k =1 }^J \exp(\gamma(\theta) F_k^{1T}(\hat \theta_k) + \xi_k + \beta r_k )}
\end{align}

and the probability that the a consumer who searches and is of type $\theta$ chooses firm $j$ is given by: 

\begin{align}
    Pr( D = j \mid \hat \theta, \theta, S=1) = \frac{\exp( \gamma(\theta) F_j^{2T}(\hat \theta) + \xi_j + \beta r_j )}{1+ \sum_{k =1 }^J \exp(\gamma(\theta) F_k^{2T}(\hat \theta) + \xi_k + \beta r_k )}
\end{align}


 

The consumer decides whether to search. Define $U_{i0}(\theta, \hat \theta) = \max_j u_{ij}(F_j^{1T}(\hat \theta_j))$ and $U_{i1}(\theta, \hat \theta) = \max_j u_{ij}(F_j^{2T}(\hat \theta))$ he searches if: 

\begin{align}
    \underbrace{U_{i1}(\theta, \hat \theta)  - U_{i0}(\theta, \hat \theta)}_{\equiv G(\theta, \hat \theta)}> s_i
\end{align}

hence the search probability given the signals and his type is given by: 

\begin{align}
    \pi_1(\theta, \hat \theta) = F_{s\mid \theta}(G(\theta, \hat \theta))  
\end{align}

%%%%%%%%%%%%%%%%%%%%%%%%%%%%%%%%%%%%%%%%%%%%%%%%%%%%%%%%%%%%%%%%%%%%%%%%%


The conditional demand for non-searchers in the first stage is given by:
%\begin{align}
%    D_{j0}(\theta, \hat \theta_j) = Pr( D = j \mid \hat \theta_j, \theta, S=0) 
%    &= \int_{\hat \theta_{-j}} Pr( D = j \mid \hat \theta, \theta, S=0)  \phi(\hat \theta_{-j}; \theta) d\hat \theta_{-j} \\
%    &= \int_{\hat \theta_{-j}} \frac{\exp( \gamma(\theta) F_j^{1T}(\hat \theta_j) + \xi_j + \beta r_j )}{1+ \sum_{k =1 }^J \exp(\gamma(\theta) F_k^{1T}(\hat \theta_k) + \xi_k + \beta r_k )} \phi(\hat \theta_{-j}; \theta) d\hat \theta_{-j}
%\end{align}

\begin{align}
    D_{j0}(\theta, \hat \theta_j) &= Pr( D = j \mid \hat \theta_j, \theta, S=0) 
    = \int_{\hat \theta_{-j}} Pr( D = j \mid \hat \theta, \theta, S=0) [1-\pi_1(\theta, (\hat \theta_{-j}, \hat \theta_j))] \phi(\hat \theta_{-j}; \theta) d\hat \theta_{-j} \notag \\
    &= \int_{\hat \theta_{-j}} \frac{\exp( \gamma(\theta) F_j^{1T}(\hat \theta_j) + \xi_j + \beta r_j )}{1+ \sum_{k =1 }^J \exp(\gamma(\theta) F_k^{1T}(\hat \theta_k) + \xi_k + \beta r_k )} [1-\pi_1(\theta, (\hat \theta_{-j}, \hat \theta_j))] \phi(\hat \theta_{-j}; \theta) d\hat \theta_{-j}
\end{align}



Then the conditional demand for searchers in the first stage is given by: 

\begin{align}
    D_{j1} (\theta, \hat \theta) = Pr(D=j, S=1 \mid \theta, \hat \theta) =  \pi_1(\theta, \hat \theta) \cdot Pr( D = j \mid \hat \theta, \theta, S=1) 
\end{align}

and the unconditional demand is given by: 

\begin{align} 
    D_{j1}(\theta, \hat \theta_j) = Pr(D=j | S =1, \theta_j, \theta) =\int_{\hat \theta_{-j}}  D_{j1} (\theta, (\hat \theta_{-j}, \hat \theta_j)) \phi(\hat \theta_{-j}; \theta) f_0(\theta) d\theta d\hat \theta_{-j}
\end{align}

Then the firm posterior mean when the consumer does not search is gien by: 


\begin{align}
    \mathbb{E}(\theta \mid \hat \theta_j, D=j, S=0) 
    &= \int_\theta \theta f(\theta\mid \hat \theta_j, D=j, S=0) d\theta
    %= \int_\theta \theta \frac{f(D=j, s_0 \mid \theta, \hat \theta_j)     \phi(\hat \theta_j; \theta, \sigma_j) f_0(\theta)}{\int_{\theta'} f(D=j,  s_0 \mid \theta', \hat \theta_j ) \phi(\hat \theta_j; \theta', \sigma_j) f_0(\theta') d\theta'} d\theta \\
    =  \frac{ \int_\theta \theta f(D=j, S=0 \mid \theta, \hat \theta_j) 
    \phi(\hat \theta_j; \theta, \sigma_j) f_0(\theta)}{\int_{\theta'} f(D=j,  S=0 \mid \theta', \hat \theta_j) \phi(\hat \theta_j; \theta', \sigma_j) f_0(\theta') d\theta'} d\theta \notag \\
    &=  \frac{ \int_\theta \theta D_{j0}(\theta, \hat \theta_j) 
    \phi(\hat \theta_j; \theta, \sigma_j) f_0(\theta)}{\int_{\theta'} D_{j0}(\theta', \hat \theta_j) \phi(\hat \theta_j; \theta', \sigma_j) f_0(\theta') d\theta'} d\theta 
\end{align}

and the posterior mean given that the consumer searches is given by: 


%\begin{align}
%    \mathbb{E}(\theta \mid \hat \theta_j, D=j, S=1) 
%    = \int_\theta \theta f(\theta\mid \hat \theta_j, D=j, S=1) d\theta
%    =  \frac{ \int_\theta \theta f(D=j, S=1 \mid \theta, \hat \theta_j) \phi(\hat \theta_j; \theta, \sigma_j) f_0(\theta)}{\int_{\theta'} f(D=j,  S=1  \mid \theta', \hat \theta_j) \phi(\hat \theta_j; \theta', \sigma_j) f_0(\theta') d\theta'} d\theta
%\end{align}


\begin{align}
    \mathbb{E}(\theta \mid \hat \theta, D=j, S=1) 
    &= \int_\theta \theta f(\theta\mid \hat \theta, D=j, S=1) d\theta
    = \int_\theta \theta \frac{f(D=j, S=1 \mid \theta, \hat \theta) \phi(\hat \theta; \theta, \sigma_j) f_0(\theta)}{\int_{\theta'} f(D=j,  S=1 \mid \theta', \hat \theta ) \phi(\hat \theta; \theta', \sigma_j) f_0(\theta') d\theta'} d\theta \notag \\
    &=  \frac{ \int_\theta \theta f(D=j, S=1 \mid \theta, \hat \theta) 
    \phi(\hat \theta; \theta, \sigma_j) f_0(\theta)}{\int_{\theta'} f(D=j,  S=1  \mid \theta', \hat \theta) \phi(\hat \theta; \theta', \sigma_j) f_0(\theta') d\theta'} d\theta \\
    &=  \frac{ \int_\theta \theta D_{j1} (\theta, \hat \theta)
    \phi(\hat \theta; \theta, \sigma_j) f_0(\theta)}{\int_{\theta'} 
    D_{j1} (\theta', \hat \theta)     \phi(\hat \theta; \theta', \sigma_j) f_0(\theta') d\theta'} d\theta
\end{align}




\newpage 





\end{document}