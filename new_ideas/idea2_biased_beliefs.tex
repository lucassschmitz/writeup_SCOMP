 \documentclass[12pt]{article}
\usepackage{amsfonts}
\usepackage{eurosym}
\usepackage{geometry}
\usepackage{amsmath,amsthm,amssymb}
\usepackage{graphicx}
\usepackage{comment}
\usepackage[utf8]{inputenc}
\usepackage{setspace}
%\usepackage[sort,comma]{natbib}
\usepackage[backend=biber, style = apa]{biblatex}
\usepackage{placeins} % to separate sections

\usepackage{adjustbox}
\usepackage{array}
\usepackage{multirow}
\usepackage{graphicx}
\usepackage{subcaption}
\usepackage{pifont}
\usepackage{amssymb}
\usepackage{comment}
 
\usepackage[hang, flushmargin, bottom]{footmisc}
\usepackage{hyperref}

\usepackage{footnotebackref}
\usepackage{xcolor}
\usepackage{booktabs}
\usepackage{pifont}
\usepackage{caption}
\usepackage{float}
\setlength{\marginparwidth}{2cm} 

\usepackage{todonotes}
\setcounter{MaxMatrixCols}{10}
%TCIDATA{OutputFilter=LATEX.DLL}
%TCIDATA{Version=5.50.0.2960}
%TCIDATA{<META NAME="SaveForMode" CONTENT="1">}
%TCIDATA{BibliographyScheme=BibTeX}
%TCIDATA{LastRevised=Sunday, April 28, 2024 18:12:38}
%TCIDATA{<META NAME="GraphicsSave" CONTENT="32">}
%TCIDATA{Language=American English}

%\setlength{\bibsep}{0.3pt}
\setlength{\textfloatsep}{5pt}
\hypersetup{breaklinks=true,hypertexnames=false,colorlinks=true,citecolor = teal}
\captionsetup{font=normalsize}
\newcommand{\cmark}{\ding{51}}
\def\sym#1{\ifmmode^{#1}\else\(^{#1}\)\fi}
\renewcommand{\thetable}{\Roman{table}}
\geometry{verbose,tmargin=.9in,bmargin=1in,lmargin=1in,rmargin=.9in,nomarginpar}
\makeatletter

\DeclareTextSymbolDefault{\textquotedbl}{T1}
\theoremstyle{plain}
\newtheorem{thm}{Theorem}%[section] commented out to avoid numbering by section
\newtheorem{prop}[thm]{Proposition}
\newtheorem{ass}[thm]{Assumption}
\newtheorem{lemma}[thm]{Lemma}
\newtheorem{theorem}[thm]{Theorem}   % alias for \begin{theorem}
\newtheorem{definition}{Definition}
\newtheorem{corollary}[thm]{Corollary}
\newtheorem{proposition}[thm]{Proposition}
\makeatother



\newcommand{\sepline}{\par\bigskip\noindent\rule{\linewidth}{0.4pt}\par\medskip}

% \input{tcilatex}
\usepackage{enumitem} % allows custom labels
\usepackage{tikz}
\usetikzlibrary{shapes.geometric, arrows, positioning}

\addbibresource{../references.bib}
\begin{document}
 
\title{\Large Search and Leverage: With an Empirical Application to the Chilean Annuities Market\thanks{We thank...}}
\author{Lucas Schmitz\thanks{Yale University (email: \texttt{lucas.schmitz@yale.edu}). Corresponding author.} \,\,\,\,\,\, and \,\,\,\,\,\, Diego Cussen\thanks{New York University (email: \texttt{dc5004@nyu.edu}).}} 
\date{\today}
\maketitle
 

 
Some thoughts: 
\begin{itemize}
    \item Imagine that you have all the info observed by the firms ($x$) and all the private information $y$. Then you can calculate the survival probabilities for each individual $f(x,y)$. My intuition is that one can calculate the variation of $f()$ given by $y$ 

    The degree of adverse selection should tell you something about whether individuals are using $y$. 
\end{itemize}



There is this story where people have private information about their risk and there is selection into the market. But I highly doubt that people actually can select into the market based on their risk type. My prior is that people have private information about their risk but actually do not use it correctly to choose the most convenient products. 

To prove the point we would need to show that people do not use private information as much as they should, hence we would need to observe private information. I think Chile is a good setting since we could merge the annuities data with health records. 

This would go in the same line as Handel (2013) %\textcite{handel_adverse_2013} 
that shows that inertia decreases adverse selection, because people are not choosing optimally. In this case it would be similar in the sense that people are just not using their private information to choose the best product. 





\textbf{Empirical strategy}

We can follow the annuities literature and estimate a model of annuities choice with private information, but in our setting we would have access to part of the private information (health records). Using this, we can use a mortality model to estimate the mortality risk given our observed health records, and given the estimated mortality model we can compute the effect o


For example consider that firms observe the covariates $x$ and consumers observe $x$ and $y$. Denote by $\mu_t(x,y)$ the probability an individual dies at age t $x$ and $y$ and by $\mu(x)$ the probability conditional only on $x$.
Denote by $S(t|x,y)$ and $S(t|x)$ the corresponding survival functions.


Consider a demand for annuities that depends on the expected mortality risk. Which we can microfound. For example the utility of an annuity could be: 
\begin{equation}
    U = \mathbb{E}_T\left[\int_0^{T} e^{-\delta t} c_t^\alpha dt \right] = \int_0^\infty e^{-\delta t} S(t|x,y) c_t^\alpha dt
\end{equation}

We assume that firms only observe $x$, individuals observe $x$ and $y$ and the econometrician also observes $x$ and $y$.

Individuals face a choice set $\mathcal{C}$ of annuities contracts with different guaranteed periods and prices. We observe the choice of annuity $j \in \mathcal{C}$ that maximizes their utility given their information:


\textbf{Identification argument}

Given that firms observe only $x$, the choice set $\mathcal{C}$ depends only on $x$, variation in prices and average choices across different $x$ allows us to identify the 


Hence the choice of annuity $j$ reveals information about $y$ since individuals choose the annuity that maximizes their utility given their information $(x,y)$ over the choice set $\mathcal{C}(x)$.

Identification issue: if individuals are bad at translating their information ($x,y$) into mortality beliefs, then we cannot identify the model. 

\begin{itemize}
    \item Note that here the data from SCOMP is not better than the medicare data, it seems unlikely that no researcher has tried to estimate this behavioral bias before. 
\end{itemize}

\newpage


\subsection*{Two-period model with distorted survival beliefs}

\paragraph{1. Environment.}

Time is $t=1,2$. Each individual has initial wealth (savings) $W=0$ at $t=1$.

\begin{itemize}
    \item True probability of being alive at $t=2$ is $x\in(0,1)$.
    \item Individual beliefs about survival are distorted: the perceived survival probability is
    \[
        \hat x \equiv \theta x,
    \]
    where $\theta>0$ and $\theta=1$ corresponds to rational beliefs.
    \item Period utility from consumption is
    \[
        u(c) = c^{\alpha}, \qquad \alpha\in(0,1).
    \]
    \item Bequest utility is linear:
    \[
        v(B) = \beta_i B, \qquad \beta_i \ge 0.
    \]
    \item No time discounting.
\end{itemize}

At $t=1$ the individual chooses between:

\begin{enumerate}
    \item Not buying an annuity (self-insuring with savings).
    \item Buying an annuity that pays a fixed amount $F>0$ in each period in which the individual is alive (no bequest from the annuity).
\end{enumerate}

\paragraph{2. No annuity: optimal consumption and expected utility.}

If the individual does not buy the annuity, she chooses savings $s \in [0,W]$ at $t=1$:
\[
    c_1 = W - s.
\]

At $t=2$:
\begin{itemize}
    \item If alive, consumption is $c_2 = s$ and bequest $B=0$.
    \item If dead, consumption is $c_2=0$ and bequest is $B = s$.
\end{itemize}

Given beliefs $\hat x = \theta x$, expected utility from not buying the annuity, for a given $s$, is
\begin{align*}
    U^{N}(s; x,\theta,\beta_i)
    &= u(c_1)
       + \hat x \, u(c_2)
       + (1-\hat x)\, v(B) \\
    &= (W - s)^{\alpha}
       + \theta x \, s^{\alpha}
       + (1-\theta x)\, \beta_i s.
\end{align*}

The individual chooses $s$ to maximize $U^{N}(s; x,\theta,\beta_i)$ subject to $0 \le s \le W$.
Assuming an interior solution $s^\ast \in (0,W)$, the first-order condition is
\begin{align*}
    \frac{\partial U^{N}}{\partial s}
    &= -\alpha (W - s)^{\alpha-1}
       + \theta x \, \alpha s^{\alpha-1}
       + (1-\theta x)\,\beta_i
       \;=\;0.
\end{align*}

Denote the optimal choice by $s^\ast(x,\theta,\beta_i,W)$ (solution to the FOC and bounds).  
Optimal consumptions are then
\[
    c_1^\ast = W - s^\ast, 
    \qquad
    c_2^\ast = s^\ast,
    \qquad
    B^\ast = s^\ast.
\]

The maximal expected utility from not buying the annuity is
\[
    V^{N}(x,\theta,\beta_i,W)
    \;=\;
    (W - s^\ast)^{\alpha}
    + \theta x \, (s^\ast)^{\alpha}
    + (1-\theta x)\, \beta_i s^\ast.
\]

\paragraph{3. Buying the annuity: expected utility.}

If the individual buys the annuity, assume she uses all her wealth $W$ to buy an annuity that pays $F$ in each period she is alive, with no bequest from the annuity and no additional savings.

Then:
\begin{itemize}
    \item At $t=1$: consumption $c_1 = F$.
    \item At $t=2$: if alive, $c_2 = F$ and $B=0$; if dead, $c_2=0$ and $B=0$.
\end{itemize}

Given beliefs $\hat x = \theta x$, expected utility from buying the annuity is
\begin{align*}
    V^{A}(x,\theta)
    &= u(F) + \hat x \, u(F) + (1-\hat x)\, v(0) \\
    &= F^{\alpha} + \theta x \, F^{\alpha} \\
    &= (1 + \theta x)\, F^{\alpha}.
\end{align*}

\paragraph{4. Adoption condition for the annuity.}

The individual buys the annuity if and only if
\[
    V^{A}(x,\theta)
    \;\ge\;
    V^{N}(x,\theta,\beta_i,W),
\]
i.e.
\[
    (1 + \theta x)\, F^{\alpha}
    \;\ge\;
    (W - s^\ast)^{\alpha}
    + \theta x \, (s^\ast)^{\alpha}
    + (1-\theta x)\, \beta_i s^\ast,
\]
where $s^\ast = s^\ast(x,\theta,\beta_i,W)$ solves
\[
    -\alpha (W - s)^{\alpha-1}
    + \theta x \, \alpha s^{\alpha-1}
    + (1-\theta x)\,\beta_i = 0
\]
subject to $0 \le s \le W$.

\newpage 

\subsection*{Identification of $\theta$ and the distribution of $\beta_i$}

\paragraph{Setup.}

Recall the two-period model above. For each individual $i$:

\begin{itemize}
    \item True survival probability from period 1 to 2 is $x_i\in(0,1)$.
    \item Believed survival probability is $\hat x_i = \theta x_i$, with $\theta>0$ and $\theta x_i\in(0,1)$.
    \item Bequest weight $\beta_i$ is heterogeneous with CDF $F_\beta$ on $[0,\infty)$.
    \item Initial wealth $W>0$ and annuity payout $F>0$ are common and observed.
\end{itemize}

The expected utility from buying the annuity is
\[
    U^{A}(x_i,\theta) = (1+\theta x_i)F^\alpha.
\]

The expected utility from not buying the annuity, given bequest weight $\beta_i$, is
\[
    U^{N}(c_1;x_i,\theta,\beta_i) 
    = c_1^\alpha + \theta x_i (W-c_1)^\alpha + (1-\theta x_i)\beta_i (W-c_1),
\]
and the individual chooses $c_1^\ast(x_i,\theta,\beta_i)$ solving the FOC
\[
    \alpha c_1^{\alpha-1} 
    - \alpha \theta x_i (W-c_1)^{\alpha-1}
    - (1-\theta x_i)\beta_i = 0,
\]
with $0\le c_1^\ast\le W$.
Define the optimized no-annuity value
\[
    U^{N,\ast}(x_i,\theta,\beta_i) 
    \equiv U^{N}(c_1^\ast(x_i,\theta,\beta_i);x_i,\theta,\beta_i).
\]

The observed binary choice is
\[
    D_i = 
    \begin{cases}
        1, & \text{if the individual buys the annuity},\\
        0, & \text{otherwise}.
    \end{cases}
\]

We assume that for each $i$, the individual chooses the option with higher (perceived) expected utility:
\[
    D_i = \mathbf 1\!\left\{U^{A}(x_i,\theta)\ge U^{N,\ast}(x_i,\theta,\beta_i)\right\}.
\]

We observe $(x_i,D_i)$ in the data; the primitives of interest are the scalar $\theta$ and the CDF $F_\beta$.

\paragraph{Step 1: Threshold structure in $\beta_i$.}

\begin{lemma}[Single crossing in $\beta_i$]
For any fixed $(x,\theta)$, $U^{A}(x,\theta)$ is independent of $\beta$, and $U^{N,\ast}(x,\theta,\beta)$ is strictly increasing in $\beta$. Hence the difference
\[
    \Delta U(x,\theta,\beta) \equiv U^{A}(x,\theta)-U^{N,\ast}(x,\theta,\beta)
\]
is strictly decreasing in $\beta$.
\end{lemma}

\begin{proof}
By construction, $U^{A}(x,\theta)$ does not depend on $\beta$. In $U^{N}(c_1;x,\theta,\beta)$ the term $(1-\theta x)\beta (W-c_1)$ is strictly increasing in $\beta$ whenever $W-c_1>0$ and $\theta x<1$. Optimization over $c_1$ preserves monotonicity in $\beta$, so $U^{N,\ast}(x,\theta,\beta)$ is increasing in $\beta$. Therefore $\Delta U(x,\theta,\beta)$ is strictly decreasing in $\beta$.
\end{proof}

\begin{corollary}[Threshold rule]
For each $(x,\theta)$ there exists a (possibly infinite) cutoff $\beta^\ast(x;\theta)\in[0,\infty]$ such that
\[
    D_i = 1
    \iff
    \beta_i \le \beta^\ast(x_i;\theta),
\]
i.e.
\[
    \Delta U(x_i,\theta,\beta_i)\ge 0
    \iff
    \beta_i \le \beta^\ast(x_i;\theta).
\]
\end{corollary}

\paragraph{Step 2: Choice probabilities given $(\theta,F_\beta)$.}

Define the conditional annuitization probability
\[
    p(x) \equiv \Pr(D_i=1\mid x_i=x).
\]
Using the threshold result,
\[
    p(x) = \Pr(\beta_i\le \beta^\ast(x;\theta))
         = F_\beta\big(\beta^\ast(x;\theta)\big).
\]
Thus, for any given pair $(\theta,F_\beta)$, the model implies the choice-probability function
\begin{equation}
    p(x;\theta,F_\beta) = F_\beta\big(\beta^\ast(x;\theta)\big).
    \label{eq:prob_function}
\end{equation}

\paragraph{Step 3: Observables and non-identification.}

Assume that from the data we know the function $p(x)$ for all $x$ in the support of $x_i$.\footnote{Formally, we observe the joint distribution of $(x_i,D_i)$, so $p(x)=\Pr(D_i=1\mid x_i=x)$ is identified.}

Equation \eqref{eq:prob_function} shows that the observables pin down only the \emph{composition} of $F_\beta$ with the cutoff function $\beta^\ast(\cdot;\theta)$:
\[
    p(x) = F_\beta\big(\beta^\ast(x;\theta)\big).
\]

We now show that $(\theta,F_\beta)$ is \emph{not} identified from $p(x)$ alone.

\begin{ass}[Monotonic $\beta^\ast$ in $x$]
For each fixed $\theta>0$, the function $x\mapsto \beta^\ast(x;\theta)$ is strictly monotone and continuous on the support of $x$, hence invertible, with inverse $b\mapsto x^\theta(b)$.
\end{ass}

\begin{proposition}[Non-identification of $(\theta,F_\beta)$]
Suppose the data are generated by some true pair $(\theta_0,F_\beta^0)$, yielding choice probabilities $p(x) = F_\beta^0(\beta^\ast(x;\theta_0))$.
Then for any alternative $\theta_1>0$ with monotone $\beta^\ast(\cdot;\theta_1)$, there exists a CDF $F_\beta^1$ such that
\[
    p(x) = F_\beta^1\big(\beta^\ast(x;\theta_1)\big)
    \quad\text{for all }x.
\]
Hence $(\theta,F_\beta)$ is not point-identified.
\end{proposition}

\begin{proof}
Fix any $\theta_1>0$ and consider the function $\beta^\ast(x;\theta_1)$.
By monotonicity, it has an inverse $x^{\theta_1}(b)$ satisfying $\beta^\ast(x^{\theta_1}(b);\theta_1)=b$.

Define a new CDF $F_\beta^1$ on the range of $\beta^\ast(\cdot;\theta_1)$ by
\[
    F_\beta^1(b) \equiv p\big(x^{\theta_1}(b)\big), \qquad 
    b\in \mathrm{Range}\big(\beta^\ast(\cdot;\theta_1)\big).
\]
Because $p(\cdot)$ is nondecreasing in $x$ and $x^{\theta_1}(\cdot)$ is monotone in $b$, $F_\beta^1$ is nondecreasing in $b$ and can be completed to a valid CDF on $[0,\infty)$.

Then, for any $x$,
\[
    F_\beta^1\big(\beta^\ast(x;\theta_1)\big)
    = p\big(x^{\theta_1}(\beta^\ast(x;\theta_1))\big)
    = p(x).
\]
Thus the pair $(\theta_1,F_\beta^1)$ generates exactly the same choice probabilities $p(x)$ as the true pair $(\theta_0,F_\beta^0)$.

Therefore the mapping $(\theta,F_\beta)\mapsto p(x)$ is not injective: different pairs $(\theta,F_\beta)$ produce the same observable distribution of $(x_i,D_i)$, so $(\theta,F_\beta)$ is not identified.
\end{proof}

\paragraph{Conclusion.}

In this two-period model with one binary annuitization choice and heterogeneous bequest weights $\beta_i\sim F_\beta$, the observable choice probabilities by risk type $x$ satisfy
\[
    p(x) = F_\beta\big(\beta^\ast(x;\theta)\big),
\]
which depends on $(\theta,F_\beta)$ only through their composition. Without additional structure (e.g.\ parametric restrictions on $F_\beta$, extra product dimensions that shift the cutoff differently for beliefs vs.\ $\beta_i$, or direct information on $\beta_i$), the belief parameter $\theta$ and the distribution $F_\beta$ cannot be separately identified.

\newpage 

\subsection*{Three-period model with distorted survival beliefs and guaranteed annuity}

\paragraph{1. Environment.}

Time is $t=1,2,3$. Each individual has initial wealth (savings) $W>0$ at $t=1$.

\begin{itemize}
    \item True survival probability between any two consecutive periods is $x\in(0,1)$.
    \item Individual beliefs about survival are distorted: the perceived one-period survival probability is
    \[
        \hat x \equiv \theta x,
    \]
    with $\theta>0$ and $\hat x\in(0,1)$.
    \item Period utility from consumption is
    \[
        u(c) = c^{\alpha}, \qquad \alpha\in(0,1).
    \]
    \item Bequest utility is linear:
    \[
        v(B) = \beta_i B, \qquad \beta_i \ge 0.
    \]
    \item No time discounting.
\end{itemize}

The individual has three options at $t=1$:

\begin{enumerate}
    \item[\textbf{N}] \textbf{No annuity (self-insurance with savings).}
    \item[\textbf{A}] \textbf{Immediate annuity}: pays $F>0$ in each period the individual is alive, with no bequest.
    \item[\textbf{G}] \textbf{Guaranteed annuity}: pays $F_g>0$ in each period the individual is alive; in addition, if the individual dies between periods 1 and 2, the contract pays a guaranteed amount $F_g$ at $t=2$ as a pure bequest (no consumption).
\end{enumerate}

\paragraph{2. No annuity: consumption--savings problem and value.}

If the individual does not buy an annuity, she chooses a consumption--savings plan:
\[
    c_1, s_1 \ge 0,\quad c_2, s_2 \ge 0,\quad c_3 \ge 0,
\]
subject to the budget constraints
\[
    c_1 + s_1 = W, \qquad
    c_2 + s_2 = s_1 \ \text{(if alive at $t=2$)}, \qquad
    c_3 = s_2 \ \text{(if alive at $t=3$)}.
\]

If she dies between $t=1$ and $t=2$, she leaves bequest $B_1 = s_1$ at $t=2$.  
If she is alive at $t=2$ and dies between $t=2$ and $t=3$, she leaves bequest $B_2 = s_2$ at $t=3$.

Given beliefs $\hat x = \theta x$, the expected utility from a given plan
$(c_1,s_1,c_2,s_2,c_3)$ is
\begin{align*}
    U^{N}(c_1,s_1,c_2,s_2,c_3; x,\theta,\beta_i)
    &= u(c_1)
       + (1-\hat x)\, \beta_i s_1 \\
    &\quad + \hat x\Big[
            u(c_2)
            + (1-\hat x)\,\beta_i s_2
            + \hat x\, u(c_3)
        \Big].
\end{align*}

The value of self-insurance is
\[
    V^{N}(x,\theta,\beta_i,W)
    \equiv
    \max_{\substack{c_1,s_1,c_2,s_2,c_3\ge 0 \\ c_1+s_1=W \\ c_2+s_2=s_1 \\ c_3=s_2}}
    U^{N}(c_1,s_1,c_2,s_2,c_3; x,\theta,\beta_i).
\]

\paragraph{3. Immediate annuity (A).}

If the individual buys the immediate annuity, she uses all wealth $W$ to purchase a contract that pays $F$ in each period she is alive. There is no bequest from the annuity.

\begin{itemize}
    \item At $t=1$: alive with probability $1$, consumption $c_1 = F$.
    \item At $t=2$: alive with probability $\hat x$, consumption $c_2 = F$; dead with probability $1-\hat x$, no consumption and no bequest.
    \item At $t=3$: alive with probability $\hat x^2$, consumption $c_3 = F$; otherwise no consumption and no bequest.
\end{itemize}

Expected utility under the immediate annuity is
\begin{align*}
    V^{A}(x,\theta)
    &= u(F) + \hat x\, u(F) + \hat x^2\, u(F) \\
    &= \big(1 + \hat x + \hat x^2\big)\, F^{\alpha}.
\end{align*}

\paragraph{4. Guaranteed annuity (G).}

If the individual buys the guaranteed annuity, she uses all wealth $W$ to purchase a contract that:

\begin{itemize}
    \item pays $F_g$ each period she is alive (as with an immediate annuity),
    \item if she dies between $t=1$ and $t=2$, pays a guaranteed amount $F_g$ at $t=2$ as a bequest (no consumption).
\end{itemize}

Thus:

\begin{itemize}
    \item At $t=1$: always alive, consumption $c_1 = F_g$.
    \item At $t=2$:
    \begin{itemize}
        \item alive (probability $\hat x$): consumption $c_2 = F_g$;
        \item dead between 1 and 2 (probability $1-\hat x$): bequest $B_1 = F_g$, no consumption.
    \end{itemize}
    \item At $t=3$:
    \begin{itemize}
        \item alive (probability $\hat x^2$): consumption $c_3 = F_g$;
        \item otherwise: no consumption and no bequest (beyond the guaranteed payment already accounted for at $t=2$).
    \end{itemize}
\end{itemize}

Expected utility under the guaranteed annuity is
\begin{align*}
    V^{G}(x,\theta,\beta_i)
    &= u(F_g)
       + \hat x\, u(F_g)
       + \hat x^2\, u(F_g)
       + (1-\hat x)\, v(F_g) \\
    &= \big(1 + \hat x + \hat x^2\big)\, F_g^{\alpha}
       + (1-\hat x)\,\beta_i F_g.
\end{align*}

\paragraph{5. Choice among N, A, and G.}

Given $(x,\theta,\beta_i,W)$, the individual chooses the option with highest expected utility:
\[
    \text{Option chosen} 
    = \arg\max\big\{V^{N}(x,\theta,\beta_i,W),\, V^{A}(x,\theta),\, V^{G}(x,\theta,\beta_i)\big\}.
\]

Equivalently, the conditions for each choice are:

\begin{itemize}
    \item \textbf{Buys immediate annuity (A)} if
    \[
        V^{A}(x,\theta) \;\ge\; V^{G}(x,\theta,\beta_i)
        \quad\text{and}\quad
        V^{A}(x,\theta) \;\ge\; V^{N}(x,\theta,\beta_i,W).
    \]
    \item \textbf{Buys guaranteed annuity (G)} if
    \[
        V^{G}(x,\theta,\beta_i) \;\ge\; V^{A}(x,\theta)
        \quad\text{and}\quad
        V^{G}(x,\theta,\beta_i) \;\ge\; V^{N}(x,\theta,\beta_i,W).
    \]
    \item \textbf{Uses savings only (N)} if
    \[
        V^{N}(x,\theta,\beta_i,W) \;\ge\; V^{A}(x,\theta)
        \quad\text{and}\quad
        V^{N}(x,\theta,\beta_i,W) \;\ge\; V^{G}(x,\theta,\beta_i).
    \]
\end{itemize}


\newpage 


\subsection*{Identification of $\theta$ and $F_\beta$ in the three-period model}

\paragraph{Setup.}

For each individual $i$:

\begin{itemize}
    \item True one-period survival probability is $x_i\in(0,1)$.
    \item Believed survival is $\hat x_i = \theta x_i$, with $\theta>0$ and $\hat x_i\in(0,1)$.
    \item Bequest weight $\beta_i$ is heterogeneous with CDF $F_\beta$ on $[0,\infty)$.
    \item Wealth $W$, annuity payments $F$ and $F_g$, and preference parameter $\alpha$ are known.
\end{itemize}

Expected utilities (using the three-period model above) are:
\begin{align*}
    V^{A}(x_i,\theta)
    &= (1 + \hat x_i + \hat x_i^2)\,F^\alpha, \\
    V^{G}(x_i,\theta,\beta_i)
    &= (1 + \hat x_i + \hat x_i^2)\,F_g^\alpha
       + (1-\hat x_i)\,\beta_i F_g, \\
    V^{N}(x_i,\theta,\beta_i,W)
    &= \max_{(c_1,s_1,c_2,s_2,c_3)} U^{N}(c_1,s_1,c_2,s_2,c_3; x_i,\theta,\beta_i),
\end{align*}
where $U^{N}(\cdot)$ is as defined in the three-period savings problem and is strictly increasing in $\beta_i$.\footnote{Linearity of bequest utility and $\beta_i\ge 0$ imply $V^{N}$ is strictly increasing in $\beta_i$.}

The observed choice is
\[
    D_i \in \{A,G,N\}
    \quad\text{with}\quad
    D_i = \arg\max\{V^{A},V^{G},V^{N}\}.
\]
We observe the joint distribution of $(x_i,D_i)$; primitives of interest are $\theta$ and $F_\beta$.

\paragraph{Step 1: Threshold structure in $\beta_i$.}

\begin{lemma}[Monotone partition in $\beta_i$]
Fix $(x,\theta)$. Then:
\begin{enumerate}
    \item $V^{A}(x,\theta)$ does not depend on $\beta$.
    \item $V^{G}(x,\theta,\beta)$ is strictly increasing in $\beta$ (linear term $(1-\hat x)\beta F_g$).
    \item $V^{N}(x,\theta,\beta,W)$ is strictly increasing in $\beta$.
\end{enumerate}
Hence, for each $(x,\theta)$ there exist (possibly infinite) cutoffs
\[
    0 \le \beta_1(x;\theta) \le \beta_2(x;\theta) \le \infty
\]
such that, almost surely in $\beta_i$,
\[
    D_i =
    \begin{cases}
        A, & \text{if } \beta_i \le \beta_1(x_i;\theta),\\[3pt]
        G, & \text{if } \beta_1(x_i;\theta) < \beta_i \le \beta_2(x_i;\theta),\\[3pt]
        N, & \text{if } \beta_i > \beta_2(x_i;\theta).
    \end{cases}
\]
\end{lemma}

\begin{proof}[Sketch]
$V^{A}$ is constant in $\beta$. $V^{G}$ is affine and strictly increasing in $\beta$. In $V^{N}$, bequest enters linearly as $\beta$ times expected bequests, and the optimal plan preserves strict monotonicity in $\beta$ (more weight on bequests cannot reduce the value). Thus $V^{N}$ is strictly increasing in $\beta$. The three strictly ordered graphs $V^{A}$ (flat), $V^{G}$ (increasing) and $V^{N}$ (increasing, eventually dominating as $\beta\to\infty$) cross at most twice, yielding the partition by two thresholds.
\end{proof}

\paragraph{Step 2: Choice probabilities as compositions of $F_\beta$ and cutoffs.}

Define the conditional choice probabilities
\[
    p_k(x) \equiv \Pr(D_i = k \mid x_i = x),\qquad k\in\{A,G,N\}.
\]
By the lemma and the law of $\beta_i$,
\begin{align}
    p_A(x)
    &= \Pr(\beta_i \le \beta_1(x;\theta))
     = F_\beta\big(\beta_1(x;\theta)\big), \label{eq:pA}\\[4pt]
    p_A(x) + p_G(x)
    &= \Pr(\beta_i \le \beta_2(x;\theta))
     = F_\beta\big(\beta_2(x;\theta)\big), \label{eq:pAG}\\[4pt]
    p_N(x)
    &= 1 - \big[p_A(x)+p_G(x)\big]
     = 1 - F_\beta\big(\beta_2(x;\theta)\big). \label{eq:pN}
\end{align}
Thus, for any $(\theta,F_\beta)$, the model implies the system
\[
    p_A(x) = F_\beta\big(\beta_1(x;\theta)\big),
    \qquad
    p_A(x)+p_G(x) = F_\beta\big(\beta_2(x;\theta)\big)
\]
for all $x$ in the support.

\paragraph{Step 3: Non-identification of $(\theta,F_\beta)$ (nonparametric $F_\beta$).}

Assume:
\begin{itemize}
    \item[(A1)] For each $\theta>0$, the functions $x\mapsto \beta_1(x;\theta)$ and $x\mapsto \beta_2(x;\theta)$ are strictly monotone and continuous on the support of $x$, hence invertible with inverses $b\mapsto x_1^\theta(b)$ and $b\mapsto x_2^\theta(b)$.
    \item[(A2)] The images of $\beta_1(\cdot;\theta)$ and $\beta_2(\cdot;\theta)$ are disjoint (or can be made disjoint by restricting attention to suitable subsets of $x$ where each rank dominates).
\end{itemize}

Suppose the data are generated by some true pair $(\theta_0,F_\beta^0)$, yielding observed choice probabilities
\[
    p_A(x) = F_\beta^0\big(\beta_1(x;\theta_0)\big),
    \qquad
    p_A(x)+p_G(x) = F_\beta^0\big(\beta_2(x;\theta_0)\big),
\]
for all $x$.

\begin{proposition}[Non-identification]
Under (A1)–(A2), for any alternative $\theta_1>0$ there exists a CDF $F_\beta^1$ such that the pair $(\theta_1,F_\beta^1)$ generates the same conditional choice probabilities $\{p_A(x),p_G(x),p_N(x)\}$ for all $x$. Hence $(\theta,F_\beta)$ is not point-identified.
\end{proposition}

\begin{proof}
Fix $\theta_1>0$ and consider the cutoff functions $\beta_1(x;\theta_1)$ and $\beta_2(x;\theta_1)$.
By (A1), define inverses $x_1^{\theta_1}(b)$ and $x_2^{\theta_1}(b)$ on the images of $\beta_1(\cdot;\theta_1)$ and $\beta_2(\cdot;\theta_1)$, respectively, such that
\[
    \beta_1\big(x_1^{\theta_1}(b);\theta_1\big)=b, 
    \qquad
    \beta_2\big(x_2^{\theta_1}(b);\theta_1\big)=b.
\]

Define $F_\beta^1$ on these images by
\begin{align*}
    F_\beta^1(b)
    &\equiv p_A\big(x_1^{\theta_1}(b)\big),
    &&\text{for } b\in \mathrm{Im}\big(\beta_1(\cdot;\theta_1)\big),\\[4pt]
    F_\beta^1(b)
    &\equiv p_A\big(x_2^{\theta_1}(b)\big) + p_G\big(x_2^{\theta_1}(b)\big),
    &&\text{for } b\in \mathrm{Im}\big(\beta_2(\cdot;\theta_1)\big).
\end{align*}
Disjointness of the images (A2) guarantees that $F_\beta^1$ is well-defined. Monotonicity of $p_A$ and $p_A+p_G$ in $x$ and of $x_1^{\theta_1}$, $x_2^{\theta_1}$ in $b$ implies $F_\beta^1$ is nondecreasing in $b$ on the union of these images, and it can be extended arbitrarily (but monotonically) to a full CDF on $[0,\infty)$.

Now, for any $x$,
\[
    F_\beta^1\big(\beta_1(x;\theta_1)\big)
    = p_A(x),
\qquad
    F_\beta^1\big(\beta_2(x;\theta_1)\big)
    = p_A(x)+p_G(x).
\]
Hence the model-implied probabilities under $(\theta_1,F_\beta^1)$ satisfy
\[
    p_A(x;\theta_1,F_\beta^1)
    = F_\beta^1\big(\beta_1(x;\theta_1)\big)
    = p_A(x),
\]
and
\[
    p_A(x;\theta_1,F_\beta^1) + p_G(x;\theta_1,F_\beta^1)
    = F_\beta^1\big(\beta_2(x;\theta_1)\big)
    = p_A(x)+p_G(x),
\]
so $p_G(x;\theta_1,F_\beta^1)=p_G(x)$ and $p_N(x;\theta_1,F_\beta^1)=1-p_A(x)-p_G(x)=p_N(x)$ for all $x$.

Therefore the observed conditional choice probabilities do not uniquely determine $(\theta,F_\beta)$: for each $\theta_1>0$ we can construct a CDF $F_\beta^1$ that exactly reproduces the same $\{p_A(x),p_G(x),p_N(x)\}_{x}$.
\end{proof}

\paragraph{Conclusion.}

In the three-period model with three options (savings $N$, annuity $A$, guaranteed annuity $G$), and with a general (nonparametric) bequest distribution $F_\beta$, the observable choice probabilities
\[
    \{p_A(x),p_G(x),p_N(x)\}_{x}
\]
depend on $(\theta,F_\beta)$ only through the compositions
\[
    F_\beta\big(\beta_1(x;\theta)\big), \qquad F_\beta\big(\beta_2(x;\theta)\big).
\]
Without additional structure (e.g.\ parametric $F_\beta$, extra product dimensions or price variation that generate independent shifts in the cutoffs), $\theta$ and the full CDF $F_\beta$ cannot be separately identified.



\newpage 


\subsection*{Identification with price and wealth shifters}

\paragraph{1. Baseline: why $(\theta,F_\beta)$ is not identified without extra shifters.}

Recall the three-period model with options $k\in\{A,G,N\}$:
\[
V^{A}(x,\theta) = (1+\hat x+\hat x^2)F^\alpha,\qquad \hat x=\theta x,
\]
\[
V^{G}(x,\theta,\beta) = (1+\hat x+\hat x^2)F_g^\alpha + (1-\hat x)\beta F_g,
\]
\[
V^{N}(x,\theta,\beta,W) = \max_{(c_1,s_1,c_2,s_2,c_3)}U^{N}(\cdot; x,\theta,\beta),
\]
with $V^{N}$ strictly increasing in $\beta$. For each $(x,\theta)$ there exist cutoffs
\[
0\le \beta_1(x;\theta)\le \beta_2(x;\theta)\le\infty
\]
such that, almost surely in $\beta$,
\[
D=
\begin{cases}
A, & \beta\le \beta_1(x;\theta),\\
G, & \beta_1(x;\theta)<\beta\le \beta_2(x;\theta),\\
N, & \beta> \beta_2(x;\theta).
\end{cases}
\]

Let $p_k(x)=\Pr(D=k\mid x)$ and let $F_\beta$ be the CDF of $\beta$. Then
\begin{align*}
p_A(x) &= F_\beta(\beta_1(x;\theta)),\\
p_A(x)+p_G(x) &= F_\beta(\beta_2(x;\theta)).
\end{align*}
With fixed $(F,F_g,W)$, the observables $\{p_A(x),p_G(x),p_N(x)\}_x$ depend on
$(\theta,F_\beta)$ only through the \emph{compositions}
\[
F_\beta\big(\beta_1(x;\theta)\big),\qquad F_\beta\big(\beta_2(x;\theta)\big).
\]
As shown in the two-period case, for any alternative $\theta_1>0$ one can construct a
different CDF $F_\beta^1$ such that
\[
F_\beta^1\big(\beta_j(x;\theta_1)\big) = F_\beta^0\big(\beta_j(x;\theta_0)\big),\quad j=1,2,
\]
for all $x$, implying observational equivalence and non-identification of $(\theta,F_\beta)$
without further structure.

\paragraph{2. Adding exogenous price variation.}

Now allow the annuity payments to depend on an observable ``price state'' $r$ (e.g.\ an
interest rate or term-structure shifter):
\[
F=F(r),\qquad F_g=F_g(r),
\]
with $r$ observed by the econometrician and exogenous:
\[
r\perp (\beta,x,\theta),
\]
and with sufficient support on some compact interval $\mathcal R$.

For each $(x,r,\theta)$ define the payoffs
\[
V^{A}(x,r,\theta) = (1+\theta x + \theta^2 x^2)\, F(r)^\alpha,
\]
\[
V^{G}(x,r,\theta,\beta) = (1+\theta x + \theta^2 x^2)\,F_g(r)^\alpha
                         + (1-\theta x)\,\beta F_g(r),
\]
and $V^{N}(x,r,\theta,\beta,W)$ as before (note: with $W$ fixed, $V^N$ does not depend
on $r$ if the outside savings technology is unchanged).

For each $(x,r,\theta)$ we again obtain two cutoffs
\[
0\le \beta_1(x,r;\theta)\le \beta_2(x,r;\theta)\le\infty,
\]
with
\[
D=
\begin{cases}
A, & \beta\le \beta_1(x,r;\theta),\\
G, & \beta_1(x,r;\theta)<\beta\le \beta_2(x,r;\theta),\\
N, & \beta> \beta_2(x,r;\theta),
\end{cases}
\]
and choice probabilities
\begin{align}
p_A(x,r) &= F_\beta\big(\beta_1(x,r;\theta)\big), \label{eq:pA_xr}\\
p_A(x,r)+p_G(x,r) &= F_\beta\big(\beta_2(x,r;\theta)\big). \label{eq:pAG_xr}
\end{align}

\paragraph{3. Why price variation helps: breaking the simple relabeling.}

Without $r$, the previous non-identification argument rests on the fact that
\[
p_A(x) = F_\beta(\beta_1(x;\theta))
\]
only pins down $F_\beta$ on the \emph{image} of $x\mapsto \beta_1(x;\theta)$, and for any
alternative $\theta_1$ we can reparametrize $F_\beta$ along that image. With $r$, we now
observe $p_A(x,r)$ and $p_A(x,r)+p_G(x,r)$ for a \emph{two-dimensional} regressor
$z=(x,r)$, and the relevant composition is
\[
p_A(z)=F_\beta(\beta_1(z;\theta)),\qquad
p_A(z)+p_G(z)=F_\beta(\beta_2(z;\theta)).
\]

A sufficient condition to break the relabeling is:

\begin{description}
\item[(C1) Rich price support and invertibility.] For each $\theta>0$, the mappings
$z\mapsto \beta_j(z;\theta)$, $j=1,2$, are continuous and strictly monotone in $r$ for
all $x$, with images that overlap across different values of $(x,r)\in\mathcal X\times\mathcal R$.
Formally, for any two points $z_1,z_2$ there exist $z_3,z_4$ such that
\[
\beta_1(z_1;\theta) = \beta_2(z_3;\theta),\qquad
\beta_2(z_2;\theta) = \beta_1(z_4;\theta),
\]
and $z\mapsto \beta_j(z;\theta)$ are invertible on their images.
\end{description}

Intuitively, \((C1)\) says that as interest rates change, the thresholds
$\beta_1(x,r;\theta)$ and $\beta_2(x,r;\theta)$ sweep over the support of $\beta$ in a way
that generates \emph{overlapping evaluations} of $F_\beta$ at common $\beta$ values coming
from different $(x,r)$ pairs.

\paragraph{4. Identification sketch under (C1).}

Let $(\theta_0,F_\beta^0)$ be the true primitives generating the observed probabilities
$\{p_A(x,r),p_G(x,r),p_N(x,r)\}_{(x,r)}$ through
\eqref{eq:pA_xr}--\eqref{eq:pAG_xr}.

Suppose there is another pair $(\theta_1,F_\beta^1)$ that yields the \emph{same} choice
probabilities for all $(x,r)$. Then
\[
F_\beta^0\big(\beta_j(x,r;\theta_0)\big)
=
F_\beta^1\big(\beta_j(x,r;\theta_1)\big),
\qquad
j=1,2,
\]
for all $(x,r)$.

Because $z\mapsto \beta_j(z;\theta_k)$ are invertible on their images, we can reparametrize
these equalities as
\[
F_\beta^1(b)
=
F_\beta^0\big(\beta_j(\phi_j(b;\theta_0,\theta_1);\theta_0)\big),
\quad b\in \mathrm{Im}(\beta_j(\cdot;\theta_1)),\ j=1,2,
\]
for suitable maps $\phi_j$ taking a cutoff under $\theta_1$ back to a $(x,r)$ pair under
$\theta_0$. Under (C1), the images of $\beta_1(\cdot;\theta_1)$ and $\beta_2(\cdot;\theta_1)$
\emph{overlap} on a set of $\beta$ values with positive measure. On that overlap, the
two definitions of $F_\beta^1$ must coincide:
\[
F_\beta^0\big(\beta_1(\phi_1(b;\theta_0,\theta_1);\theta_0)\big)
=
F_\beta^0\big(\beta_2(\phi_2(b;\theta_0,\theta_1);\theta_0)\big)
\quad\text{for all such }b.
\]

Generically, this equality cannot hold for all $b$ unless $\theta_1=\theta_0$. (Any
$\theta_1\neq\theta_0$ implies a different way in which $(x,r)$ map into thresholds
$\beta_1,\beta_2$, so the overlap-induced constraints on $F_\beta^0$ are violated except on
a set of measure zero.) Thus, under (C1), the belief parameter $\theta$ is \emph{point
identified}.

Given $\theta=\theta_0$, equations \eqref{eq:pA_xr}--\eqref{eq:pAG_xr} identify $F_\beta$
nonparametrically on the union of the images of $\beta_1(\cdot;\theta_0)$ and
$\beta_2(\cdot;\theta_0)$ by
\[
F_\beta(b) = p_A\big(z_1^\ast(b)\big),\quad
F_\beta(b) = p_A\big(z_2^\ast(b)\big)+p_G\big(z_2^\ast(b)\big),
\]
for any $z_j^\ast(b)$ satisfying $\beta_j(z_j^\ast(b);\theta_0)=b$, $j=1,2$. With rich
support in $(x,r)$ these images can be made dense in the support of $\beta$, giving
(nonparametric) identification of $F_\beta$ up to standard interpolation.

\paragraph{5. Role of non-financial wealth (house value).}

Let $H$ denote non-annuitizable wealth (e.g.\ housing). If $H$ enters only as a pure
background bequest (same in all options), expected bequest from $H$ is
\[
\mathbb{E}[\beta_i H \,\mathbf 1\{\text{dead}\}]
\]
and is common to $A,G,N$, so it \emph{cancels out} of the choice problem and provides no
identification power.

To help identify $F_\beta$ separately from $\theta$, $H$ must affect utilities in a
\emph{choice-specific} way that loads only on $\beta$, not on $\theta$. For example:

\begin{itemize}
    \item $H$ can be used for consumption (and hence bequest) only if the individual does
          \emph{not} annuitize (option $N$), while $A$ and $G$ use only $W$.
    \item Or $H$ affects the collateral/borrowing constraint differently under $N$ vs.\ $A,G$.
\end{itemize}

In such cases the value functions take the form
\[
V^{k}(x,r,\theta,\beta,H)
=
V^{k}_0(x,r,\theta,H) + \beta \cdot B_k(x,H),
\quad k\in\{A,G,N\},
\]
with $B_k$ not all equal. Then $H$ generates an additional shifter that moves the
thresholds $\beta_1(x,r,H;\theta)$ and $\beta_2(x,r,H;\theta)$ in a way that is
\emph{independent} of $\theta$ (because $H$ multiplies only $\beta$). Under the same kind
of overlapping-support condition as (C1), variation in $(r,H)$ allows one to trace out
$F_\beta$ at many points for a fixed $\theta$, and the cross-equations across $(x,r,H)$
then pin down $\theta$ as above.



\section{literature review}

\begin{itemize}
    \item Cormac has a RESTAT that studies the same but is not really the interaction between private info and beliefs, rather that people are pessimistic, but do not study the interaction of beliefs with private information.
\end{itemize}

\newpage 

a

\newpage 

\section{Identification with Biased Beliefs and Bequest Heterogeneity}

This section uses the three-period model with options $N$ (self-insurance), $A$ (plain annuity), and $G$ (guaranteed annuity) to (i) state conditions under which the belief parameter $\theta$ and the bequest heterogeneity $F_\beta$ are identified nonparametrically, and (ii) show how the model can be used to quantify the impact of biased beliefs on selection into annuities. The underlying primitives and value functions are as in the previous note.

\subsection{Setup}

Individuals live for up to three periods $t=1,2,3$. Let $x\in(0,1)$ denote the true one-period survival probability, common across periods. Beliefs are distorted by a scalar parameter
\[
\hat x \;=\; \theta x, \qquad \theta>0,
\]
with $\theta=1$ corresponding to rational beliefs and $\theta<1$ meaning that individuals under-react to their survival prospects.

Preferences are
\[
u(c) = c^\alpha,\quad \alpha\in(0,1), \qquad v(B)=\beta_i B, \quad \beta_i\ge 0,
\]
with heterogeneous bequest weights $\beta_i\sim F_\beta$ (independent of $x_i$).

At $t=1$ each individual chooses one of three options:
\begin{itemize}
    \item $A$: plain annuity paying $F>0$ in each period alive, no bequest.
    \item $G$: guaranteed annuity paying $F_g>0$ each period alive and an additional pure bequest $F_g$ at $t=2$ if the individual dies between $t=1$ and $t=2$.
    \item $N$: no annuity, self-insurance with initial financial wealth $W>0$.
\end{itemize}

Let $V^k(x,\theta,\beta_i,W)$ denote expected utility under choice $k\in\{A,G,N\}$ as defined in the model section. In particular,
\[
V^{A}(x,\theta)
= \big(1+\hat x + \hat x^2\big) F^\alpha,
\]
\[
V^{G}(x,\theta,\beta_i)
= \big(1+\hat x + \hat x^2\big) F_g^\alpha + (1-\hat x)\,\beta_i F_g,
\]
and $V^N(x,\theta,\beta_i,W)$ is the value of the optimal three-period savings plan, strictly increasing in $\beta_i$.

The observed choice is
\[
D_i \in\{A,G,N\}, \qquad 
D_i = \arg\max\{V^{A}(x_i,\theta),\,V^{G}(x_i,\theta,\beta_i),\,V^{N}(x_i,\theta,\beta_i,W)\}.
\]

\subsection{Threshold structure and basic non-identification}

Fix $(x,\theta)$. Since $V^A$ is independent of $\beta$, while $V^G$ and $V^N$ are strictly increasing in $\beta$, there exist cutoffs
\[
0\le \beta_1(x;\theta)\le \beta_2(x;\theta)\le\infty
\]
such that, almost surely in $\beta_i$,
\[
D_i =
\begin{cases}
A, & \beta_i \le \beta_1(x_i;\theta),\\[0.2em]
G, & \beta_1(x_i;\theta) < \beta_i \le \beta_2(x_i;\theta),\\[0.2em]
N, & \beta_i > \beta_2(x_i;\theta).
\end{cases}
\]

Let $p_k(x)=\Pr(D_i=k\mid x_i=x)$ for $k\in\{A,G,N\}$. Independence between $x_i$ and $\beta_i$ implies
\[
p_A(x) = F_\beta\big(\beta_1(x;\theta)\big),\qquad
p_A(x)+p_G(x) = F_\beta\big(\beta_2(x;\theta)\big).
\]
With a fixed annuity menu $(F,F_g,W)$, the data therefore identify only the compositions
\[
F_\beta\circ\beta_1(\cdot;\theta),\qquad F_\beta\circ\beta_2(\cdot;\theta),
\]
which means that $(\theta,F_\beta)$ is not identified: for any alternative $\theta_1$ one can re-label the quantiles of $F_\beta$ so that $F_\beta(\beta_j(x;\theta_0))$ is matched by a different CDF evaluated at $\beta_j(x;\theta_1)$, for $j=1,2$. This is the analogue of the standard non-identification result in the two-period case.:contentReference[oaicite:1]{index=1}

\subsection{Identification with price variation}

To separately identify $\theta$ and $F_\beta$ without parametric assumptions on $F_\beta$, we exploit exogenous variation in prices induced by interest rate shifts. Let $r$ denote an observed price state (e.g.\ a short-rate or term-structure index), and let
\[
F=F(r), \qquad F_g=F_g(r),
\]
be the annuity payments associated with $r$. The value functions become
\[
V^{A}(x,\theta;r) = \big(1+\theta x + \theta^2 x^2\big) F(r)^\alpha,
\]
\[
V^{G}(x,\theta,\beta;r) 
= \big(1+\theta x + \theta^2 x^2\big) F_g(r)^\alpha + \big(1-\theta x\big)\beta F_g(r),
\]
while $V^N(x,\theta,\beta,W;r)$ may change with $r$ only through the return on savings (if any).

\paragraph{Exclusion and richness assumptions.}
\begin{enumerate}
    \item[(IR1)] $r$ is exogenous and independent of $(x_i,\beta_i)$, and it affects payoffs only through $(F(r),F_g(r))$ and the savings technology. In particular, $r$ does not affect the belief parameter $\theta$ or the distribution $F_\beta$.
    \item[(IR2)] For the true $\theta_0$, the induced cutoffs
    \[
    \beta_1(x,r;\theta_0),\qquad \beta_2(x,r;\theta_0)
    \]
    are continuous and strictly monotone in $(x,r)$, and their images across different $r\in\mathcal{R}$ overlap on sets of positive measure.
\end{enumerate}

Given $(x,r)$ and $\theta$, there exist cutoffs $\beta_1(x,r;\theta)\le \beta_2(x,r;\theta)$ partitioning the bequest space exactly as before, so that
\[
D_i =
\begin{cases}
A, & \beta_i \le \beta_1(x_i,r_i;\theta),\\[0.2em]
G, & \beta_1(x_i,r_i;\theta) < \beta_i \le \beta_2(x_i,r_i;\theta),\\[0.2em]
N, & \beta_i > \beta_2(x_i,r_i;\theta).
\end{cases}
\]
Let $p_k(x,r)=\Pr(D_i=k\mid x_i=x,r_i=r)$. Then
\begin{align*}
p_A(x,r) &= F_\beta\big(\beta_1(x,r;\theta)\big),\\
p_A(x,r)+p_G(x,r) &= F_\beta\big(\beta_2(x,r;\theta)\big).
\end{align*}

\paragraph{Identification idea.}

With one fixed price schedule, one can always construct a CDF $F_\beta$ to absorb any change in $\theta$ while keeping $\{p_k(x)\}_x$ unchanged, because $F_\beta$ is only evaluated at the one-dimensional image of $x\mapsto\beta_j(x;\theta)$. Once $r$ varies, the same \emph{single} CDF $F_\beta$ must rationalize:
\[
p_A(x,r) = F_\beta\big(\beta_1(x,r;\theta)\big),\qquad
p_A(x,r)+p_G(x,r) = F_\beta\big(\beta_2(x,r;\theta)\big)
\]
for all $(x,r)$. Under (IR2), the images of $(x,r)\mapsto\beta_j(x,r;\theta)$ for different $r$ overlap: the same $\beta$ value is reached at multiple $(x,r)$ pairs. For the true $\theta_0$, the implied values $F_\beta(\beta)$ at those overlapping points coincide by construction. For any alternative $\theta_1\neq \theta_0$, the pattern of overlapping images is different, and the system
\[
p_A(x,r) = F_\beta^1\big(\beta_1(x,r;\theta_1)\big),\qquad
p_A(x,r)+p_G(x,r) = F_\beta^1\big(\beta_2(x,r;\theta_1)\big)
\]
typically has no solution for a \emph{single} CDF $F_\beta^1$ defined on $[0,\infty)$, because it would require contradictory values at the same $\beta$ coming from different $(x,r)$ pairs.

Formally, conditions (IR1)–(IR2) ensure that the operator
\[
(\theta,F_\beta)\ \longmapsto\ \{p_A(x,r),p_G(x,r)\}_{(x,r)}
\]
is injective on the admissible parameter space: the belief parameter $\theta$ is uniquely pinned down by the way choice probabilities react to exogenous price shifts, and once $\theta$ is known, $F_\beta$ is recovered nonparametrically by inverting
\[
F_\beta(\beta_1(x,r;\theta)) = p_A(x,r),\quad
F_\beta(\beta_2(x,r;\theta)) = p_A(x,r)+p_G(x,r)
\]
over the support of $(x,r)$.

\subsection{Using background wealth to strengthen identification}

Let $H_i$ denote observable non-annuitizable wealth (e.g.\ housing) that is \emph{not} used to purchase $A$ or $G$, but affects the bequest and consumption components of $N$ and $G$; for example, $H_i$ may be fully bequeathed if the individual dies, or may relax borrowing constraints only under $N$. Assume:
\begin{enumerate}
    \item[(H1)] $(x_i,\beta_i)$ are independent of $H_i$; $H_i$ does not affect $(F(r),F_g(r))$ or $\theta$.
    \item[(H2)] For given $(x,r,\theta)$, the differences $V^G(x,\theta,\beta,H)-V^A(x,\theta;r)$ and $V^N(x,\theta,\beta,W,H)-V^A(x,\theta;r)$ are strictly increasing in both $\beta$ and $H$, and the way $H$ enters these differences is not proportional across $G$ and $N$.
\end{enumerate}

Under (H1)–(H2), the cutoffs become functions $\beta_1(x,r,H;\theta)$ and $\beta_2(x,r,H;\theta)$, and we observe
\[
p_A(x,r,H) = F_\beta\big(\beta_1(x,r,H;\theta)\big),\qquad
p_A(x,r,H)+p_G(x,r,H) = F_\beta\big(\beta_2(x,r,H;\theta)\big).
\]

Variation in $H$ shifts these cutoffs purely through the bequest channel (multiplying $\beta$), while variation in $r$ shifts them through prices and beliefs. In combination, $(r,H)$ generate a rich two-dimensional set of arguments at which $F_\beta$ is evaluated. Under the same type of overlap and monotonicity conditions as above, the joint map
\[
(\theta,F_\beta)\ \longmapsto\ \{p_A(x,r,H),p_G(x,r,H)\}_{(x,r,H)}
\]
is one-to-one: $\theta$ is identified from the \emph{price} dimension, while $F_\beta$ is traced out from the \emph{wealth} dimension, conditional on $\theta$.

\subsection{Biased beliefs and the magnitude of selection}

Once $(\theta,F_\beta)$ are identified, the model can be used to quantify how biased beliefs about survival attenuate selection into annuities. Define a simple selection measure:
\[
S(\theta) \;\equiv\; \mathbb{E}\big[x_i \mid D_i\in\{A,G\}\big]
\;-\;
\mathbb{E}\big[x_i \mid D_i = N\big],
\]
the difference between average true survival probability among annuitizers and non-annuitizers, holding the annuity menu $(F(r),F_g(r))$, the wealth distribution, and $F_\beta$ fixed.

In this model, $\theta$ controls how strongly $x_i$ enters perceived values:
\[
\hat x_i = \theta x_i,\quad V^A(x_i,\theta;r)\propto (1+\theta x_i+\theta^2 x_i^2),
\]
and the marginal bequest term in $G$ is proportional to $(1-\theta x_i)\beta_i$. When $\theta$ decreases toward zero, individual decisions become less sensitive to $x_i$:
\begin{itemize}
    \item The differences $V^A(x_i,\theta;r)-V^N(x_i,\theta,\beta_i)$ and $V^G(x_i,\theta,\beta_i;r)-V^N(x_i,\theta,\beta_i)$ become flatter in $x_i$.
    \item The cutoffs $\beta_1(x,r,H;\theta)$ and $\beta_2(x,r,H;\theta)$ vary less with $x$, so the conditional choice probabilities $p_k(x,r,H)$ become less dispersed in $x$.
\end{itemize}
Under mild regularity conditions (single-crossing in $x$), this implies that $S(\theta)$ is strictly increasing in $\theta$: when individuals are closer to rational ($\theta$ larger), high-$x$ types are more likely to buy annuities and low-$x$ types to self-insure; when $\theta<1$, beliefs compress heterogeneity in survival toward the mean, and selection is mechanically weaker.

Empirically, once we have estimated $(\hat\theta,\hat F_\beta)$ and calibrated the annuity menu and wealth distribution from the data, we can compute:
\begin{enumerate}
    \item The \emph{actual} selection $S(\hat\theta)$ implied by the estimated biased-belief model.
    \item A \emph{counterfactual} selection $S(1)$ under rational beliefs ($\theta=1$) but the same $F_\beta$, prices, and wealth distribution.
\end{enumerate}
The difference $S(1)-S(\hat\theta)$ then measures how much observed selection into annuities is dampened by biased beliefs relative to a benchmark with the same bequest heterogeneity but fully rational use of information. If, as you anticipate, $\hat\theta<1$, the model will predict that adverse selection in the Chilean annuity market is significantly smaller than what would arise under rational beliefs, because individuals are effectively more myopic about their survival.

\printbibliography

\end{document}