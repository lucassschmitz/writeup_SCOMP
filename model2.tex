\documentclass[12pt]{article}
%%%%%%%%%%%%%%%%%%%%%%%%%%%%%%%%%%%%%%%%%%%%%%%%%%%%%%%%%%%%%%%%%%%%%%%%%%%%%%%%%%%%%%%%%%%%%%%%%%%%%%%%%%%%%%%%%%%%%%%%%%%%%%%%%%%%%%%%%%%%%%%%%%%%%%%%%%%%%%%%%%%%%%%%%%%%%%%%%%%%%%%%%%%%%%%%%%%%%%%%%%%%%%%%%%%%%%%%%%%%%%%%%%%%%%%%%%%%%%%%%%%%%%%%%%%%
\usepackage{amsfonts}
\usepackage{eurosym}
\usepackage{geometry}
\usepackage{amsmath,amsthm,amssymb}
\usepackage{ulem} 
\usepackage{graphicx}
\usepackage{comment}
%\usepackage[sort,comma]{natbib}
\usepackage[backend=biber, style = apa]{biblatex}
\usepackage{placeins} % to separate sections

\usepackage{adjustbox}
\usepackage{array}
\usepackage{multirow}
\usepackage{graphicx}
\usepackage{subcaption}
\usepackage{pifont}
\usepackage{amssymb}
\usepackage{comment}
\usepackage[utf8]{inputenc}
\usepackage{setspace}
\usepackage[hang, flushmargin, bottom]{footmisc}
\usepackage{footnotebackref}
\usepackage{xcolor}
\usepackage{hyperref}
\usepackage{booktabs}
\usepackage{pifont}
\usepackage{caption}
\usepackage{float}
\usepackage{todonotes}
\setcounter{MaxMatrixCols}{10}
%TCIDATA{OutputFilter=LATEX.DLL}
%TCIDATA{Version=5.50.0.2960}
%TCIDATA{<META NAME="SaveForMode" CONTENT="1">}
%TCIDATA{BibliographyScheme=BibTeX}
%TCIDATA{LastRevised=Sunday, April 28, 2024 18:12:38}
%TCIDATA{<META NAME="GraphicsSave" CONTENT="32">}
%TCIDATA{Language=American English}

%\setlength{\bibsep}{0.3pt}
\setlength{\textfloatsep}{5pt}
\hypersetup{breaklinks=true,hypertexnames=false,colorlinks=true,citecolor = teal}
\captionsetup{font=normalsize}
\newcommand{\cmark}{\ding{51}}
\def\sym#1{\ifmmode^{#1}\else\(^{#1}\)\fi}
\renewcommand{\thetable}{\Roman{table}}
\geometry{verbose,tmargin=.9in,bmargin=1in,lmargin=1in,rmargin=.9in,nomarginpar}
\makeatletter
\DeclareTextSymbolDefault{\textquotedbl}{T1}
\theoremstyle{plain}
\newtheorem{thm}{\protect\theoremname}
\theoremstyle{plain}
\newtheorem{prop}[thm]{\protect\propositionname}
\providecommand{\propositionname}{Proposition}
\providecommand{\theoremname}{Theorem}
\makeatother
\providecommand{\propositionname}{Proposition}
\providecommand{\theoremname}{Theorem}
\newtheorem{ass}[thm]{Assumption}
% \input{tcilatex}
\usepackage{tikz}
\usetikzlibrary{shapes.geometric, arrows, positioning}





\addbibresource{references.bib}
\begin{document}
 
% \title{{\Large Centralized annuities marketplace}}
%\author{Lucas Condeza\thanks{Yale University %\texttt{lucas.schmitz@yale.edu}}} 
%\date{}
%\maketitle
\section*{Literature on search and some comments}

Models of search in \textcite{gavazza_frictions_2021}
\begin{enumerate}
    \item Informed vs uninformed: Varian (1980). Also referred as to shoppers and non-shoppers. 
    \item Simultaneous search:  Burdett and Judd (1983). One decides ex-ante the number of searches 
    \item Sequential search: Rothschild (1973) 
\end{enumerate}


Models of search (Ellison) 
\begin{enumerate}
    \item Diamond (1971)
    \item Heterogenous search
    
    Stahl (89) shoppers and non-shoppers as in Varian but the non-shoppers have a search cost hence their search is endogenous, moreover there is heterogeneity among the non shoppers. 

    \item Clearinghouse: you pay a fixed cost to know all the alternatives and if you do not pay it you are directed to a default option. (Baye and Morgan 2001)

\end{enumerate}


\subsection*{Comments}
\begin{itemize}
    \item Undirected search is the idea that the consumer requests quotes randomly, I think it simplifies the problem. 
    
    \item why is it that we do not observe the Diamond paradox in Varian(1980)? 
    
    - because buyers do not optimally select how much to search. some individuals search all the options and others none. a way of justifying it is that some have an infinite search cost and others zero search cost, hence neither of the two groups changes their search behavior when prices change. 
\end{itemize}




\section*{Varian (1980) --- Extension with Initial Offers}

Let $N$ sellers. $c=0$ makes initial offers $b_i$.

The consumer values the good at $v$.

In stage 2:
\[
\bar{v} = v - (v - \min(b_i)) = \min(b_i)
\]
where $v - \min(b_i)$ is the outside option. 

Sellers set second stage prices $p_i$, which are latent prices revealed only if the consumer asks them for an external offer.

With probability $\Pr = 1- \lambda$: the consumer knows $p_i\ \forall i$, which can be thought of as having no search costs.

There is no equilibrium in pure strategies. The Bertrand logic applies.


 Firms set prices according to $F(p)$, being
\begin{enumerate}
    \item smooth
    \item upper bound $\bar{v} $ (monopoly price)
\end{enumerate}

For any $p$ in the support, $p \cdot Q(p) = \bar{v} \cdot Q(\bar{v})$ applies, where
\[
Q(p) = \frac{\lambda}{N} + (1 - \lambda) \left(1 - F(p)\right)^{N-1}
\]

where the first term is the revenue from uninformed (who choose randomly) and the second term is revenue from informed. 


Since $F(\bar{v}) = 1$, then
\[
F(p) = 1 - \left( \frac{ \lambda}{N (1- \lambda)} \cdot \left( \frac{\bar{v}}{p} - 1 \right) \right)^{\frac{1}{N-1}}
\]

Then the expected profits are: 

\[
\pi =  \bar{v} \cdot Q(\bar{v})=  \bar{v} \cdot \frac{\lambda}{N}
\]

Note that 1) nobody buys in the first stage and 2) since profits are increasing on first stage prices then when doing backwards induction we get that $b_i \geq v, \forall i$. 

Essentially the problem is that all the buyers in the second stage can observe the first stage bids and make better bids because otherwise the price $p_i$ is greater than the valuation increase $\bar{v}$. 



\textbf{Comments}
\begin{itemize}
    \item The problem of this model is that everyone buys in the second stage. We need the possibility that $\max(c_i) > \bar{V}$ so that some people take the outside option.

    \item Search costs are not explicitly modeled, buyers have either 0 or $\infty$ search costs. 
\end{itemize}





 

 \section*{Varian (1980) --- extension with assymetric costs. }


 
Let $N$ sellers with marginal cost $c_i$ and $c_i < c_{i+1}$  makes initial offers $b_i$. 

The consumer values the good at $v$.

In stage 2:
\[
\bar{v} = v - (v - \min(b_i)) = \min(b_i)
\]
where $v - \min(b_i)$ is the outside option. Assume $\bar{v}>c_N$

Sellers set second stage prices $p_i$, which are latent prices revealed only if the consumer asks them for an external offer.

With probability $\Pr = 1- \lambda$: the consumer knows $p_i\ \forall i$, which can be thought of as having no search costs.

Assume there is an equilibrium where all firms play mixed strategies. 

 Firms set prices according to $F_i(p)$, being
\begin{enumerate}
    \item smooth
    \item upper bound $\bar{v} $ (monopoly price)
\end{enumerate}
then 
\[
Q_i(p) = \frac{\lambda}{N} +  (1 - \lambda) \prod_{j\neq i} \left(1 - F_j(p)\right)
\]
where the first term is the revenue from uninformed (who choose randomly) and the second term is revenue from informed. 

For any $p$ in the support, $(p - c_i) \cdot Q_i(p) = (\bar{v}-c_i) \cdot Q_i(\bar{v})$ applies. 


Moreover since $F_i(\bar{v}) = 1, \forall i$ we have: 
\begin{align*}
    (p - c_i) \cdot Q_i(p) = (\bar{v}-c_i) \cdot Q_i(\bar{v}) \\
     (p - c_i) \cdot \left[\frac{\lambda}{N} +  (1 - \lambda) \prod_{j\neq i} \left(1 - F_j(p)\right)\right] = (\bar{v}-c_i) \cdot \frac{\lambda}{N} \\
     (p- c_i) \cdot \left[ (1 - \lambda) \prod_{j\neq i} \left(1 - F_j(p)\right)\right] = \frac{\lambda}{N}(\bar{v}-p) \\
      \prod_{j\neq i} \left(1 - F_j(p)\right) = \frac{\lambda}{(1 - \lambda)N}\frac{\bar{v}-p}{p- c_i} 
\end{align*}
Take the previous equation for firms $m$ and $n$ and divide them, then: 
\begin{align*}
    \frac{1 - F_m(p)}{1 - F_n(p)}  = \frac{p-c_m}{p- c_n}
\end{align*}

Note that if 


\vspace{3cm}


There is no equilibrium in pure strategies. The Bertrand logic applies.




Since $F(\bar{v}) = 1$, then
\[
F(p) = 1 - \left( \frac{ \lambda}{N (1- \lambda)} \cdot \left( \frac{\bar{v}}{p} - 1 \right) \right)^{\frac{1}{N-1}}
\]

Then the expected profits are: 

\[
\pi =  \bar{v} \cdot Q(\bar{v})=  \bar{v} \cdot \frac{\lambda}{N}
\]

Note that 1) nobody buys in the first stage and 2) since profits are increasing on first stage prices then when doing backwards induction we get that $b_i \geq v, \forall i$. 

Essentially the problem is that all the buyers in the second stage can observe the first stage bids and make better bids because otherwise the price $p_i$ is greater than the valuation increase $\bar{v}$. 


\vspace{3cm}

First let's solve Varian with assymetric costs for a duopoly with $c_1 < c_2$.

\textbf{Claim 1: $F_i(p)$ has support $[c_2, \bar{v}]$ and is smooth}

Since $F_i(\bar{v}) =1$ we have: 

\begin{align*}
    (p - c_i) \cdot Q_i(p) = (\bar{v}-c_i) \cdot Q_i(\bar{v}) \\
     (p - c_i) \cdot \left[\frac{\lambda}{2} +  (1 - \lambda) (1 - F_j(p))\right] = (\bar{v}-c_i) \cdot \frac{\lambda}{2} \\
     (p- c_i)  \left[ (1 - \lambda)  (1 - F_j(p))\right] = \frac{\lambda}{2}(\bar{v}-p) \\
     1 - F_j(p) = \frac{\lambda}{(1 - \lambda)2}\frac{\bar{v}-p}{p- c_i} 
\end{align*}

\vspace{2cm}

\textbf{Claim: $\bar{p}_1 =\bar{p}_2 =\bar{v}$ } 

Assume $\bar{p}_1 < \bar{p}_2$

\textcolor{red}{I THINK THAT SOLVING THE MODEL FOR ASSYMETRIC COSTS IS EQUIVALENT TO SOLVING THE MODEL FOR FIRM SPECIFIC $v_i$, meaning that putting the heterogeneity in the firm cost or in the the buyer preferences is isomorphic. }


 
 


  

 \end{document}
