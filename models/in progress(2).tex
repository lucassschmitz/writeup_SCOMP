\documentclass[12pt]{article}
%%%%%%%%%%%%%%%%%%%%%%%%%%%%%%%%%%%%%%%%%%%%%%%%%%%%%%%%%%%%%%%%%%%%%%%%%%%%%%%%%%%%%%%%%%%%%%%%%%%%%%%%%%%%%%%%%%%%%%%%%%%%%%%%%%%%%%%%%%%%%%%%%%%%%%%%%%%%%%%%%%%%%%%%%%%%%%%%%%%%%%%%%%%%%%%%%%%%%%%%%%%%%%%%%%%%%%%%%%%%%%%%%%%%%%%%%%%%%%%%%%%%%%%%%%%%
\usepackage{amsfonts}
\usepackage{eurosym}
\usepackage{geometry}
\usepackage{amsmath,amsthm,amssymb}
\usepackage{ulem} 

\usepackage{comment}
%\usepackage[sort,comma]{natbib}
\usepackage[backend=biber, style = apa]{biblatex}
\usepackage{placeins} % to separate sections

\usepackage{adjustbox}
\usepackage{array}
\usepackage{multirow}
\usepackage{graphicx}
\usepackage{subcaption}
\usepackage{pifont}
\usepackage{amssymb}
\usepackage{comment}
\usepackage[utf8]{inputenc}
\usepackage{setspace}
\usepackage[hang, flushmargin, bottom]{footmisc}
\usepackage{footnotebackref}
\usepackage{xcolor}
\usepackage{booktabs}
\usepackage{pifont}
\usepackage{caption}
\usepackage{float}
\usepackage{todonotes}
\usepackage{hyperref}

\setcounter{MaxMatrixCols}{10}

%\setlength{\bibsep}{0.3pt}
\setlength{\textfloatsep}{5pt}
\hypersetup{breaklinks=true,hypertexnames=false,colorlinks=true,citecolor = teal}
\captionsetup{font=normalsize}
\newcommand{\cmark}{\ding{51}}
\def\sym#1{\ifmmode^{#1}\else\(^{#1}\)\fi}
\renewcommand{\thetable}{\Roman{table}}
\geometry{verbose,tmargin=.9in,bmargin=1in,lmargin=.8in,rmargin=.8in,nomarginpar}
\makeatletter




\addbibresource{references.bib}
\begin{document}

Here we explore what determines whether prices are higher/lower in a model with price uncertainty. The simplest model is that firms draw cost $c_j$ from a distribution $F_j$ and then make an offer $p_j$. The consumer then decides whether to accept one of the offers, their demand is given by $D_j(p)$. 

Then firm prices satisfy: 
\begin{align} \label{eq:base_equilibrium} %eq 1 
    p_j(c_j) = \arg \max_{p_j} \int_{\mathbb{R}^{N-1}}^{} (p_j - c_j) D_j(p_j, p_{-j}(c_{-j})) dF(c_{-j})
\end{align} 
where $p_{-j}(c_{-j}) = (p_n(c_n))_{n \neq j}$ and $F(c_{-j} )$ is the distribution of costs for the rival firms. 


The FOC is: 
\begin{align} % eq 1b
    \int \left[ D_j(\cdot) - (p_j - c_j) \frac{\partial D_j(\cdot)}{\partial p_j } \right] dF(c_{-j}) = 0  
\end{align}


Define $ \pi(p_j, p_{-j}, c_j) = (p_j - c_j) D_j(p_j, p_{-j})$ the profits of firm $j$ given prices and costs. 

Assume that $\pi(p_j, p_{-j}, c_j)$ is concave in $p_j$\footnote{Standard assumption, required for the FOC to provide the optimal price.}
then we have that 



\subsection{Example: logit demand}
With logit demand we have that

\vspace{3cm}

\end{document}