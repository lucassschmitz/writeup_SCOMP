%\documentclass[notes,10pt,aspectratio=169]{beamer}

%\documentclass[notes, 10pt,aspectratio=169]{beamer}
\documentclass[10pt,aspectratio=169]{beamer}


% Add this line to your preamble
%\setbeameroption{show notes on second screen=right}

%\usetheme{Singapore} %Boadilla, Madrid, default, etc. 
\usetheme[progressbar=frametitle]{metropolis}
\usecolortheme{rose} %beaver, dolphin, crane, 


%\setbeamersize{text margin left=4mm, text margin right=4mm}


\usecolortheme{default}

\usepackage[utf8]{inputenc}
\usepackage[T1]{fontenc}
\usepackage{lmodern}
\usepackage{xcolor}
\usepackage{tikz}
\usepackage{booktabs} % Required for \toprule, \midrule, \bottomrule
\usetikzlibrary{shapes.geometric, arrows, positioning}

\tikzstyle{block} = [rectangle, draw, text width=4cm, align=center, rounded corners, minimum height=1cm]
\tikzstyle{decision} = [rectangle, draw, text width=5cm, align=center, fill=blue!10, rounded corners, minimum height=1cm]
\tikzstyle{terminal} = [rectangle, draw, text width=4.5cm, align=center, fill=yellow!30, rounded corners, minimum height=1cm]
\tikzstyle{end} = [rectangle, draw, text width=5cm, align=center, fill=green!30, rounded corners, minimum height=1cm]
\tikzstyle{arrow} = [->, thick]



\usepackage{adjustbox}
%2. change the bullets 
\setbeamertemplate{itemize item}[triangle] %circle, square,... 


% 1. Define custom colors and set colors 
%\definecolor{myblue}{HTML}{003366}
\definecolor{accent}{RGB}{78,205,196}

%\setbeamercolor{title}{fg=white,bg=myblue}
\setbeamercolor{frametitle}{fg=black,bg=white}
%\setbeamercolor{normal text}{fg=mygray}
\setbeamercolor{block title}{fg=black,bg=blue}
%\setbeamercolor{block body}{fg=black,bg=white}

\setbeamercolor{item}{fg= orange!80} % Change bullet color
\setbeamercolor{button}{bg=orange, fg=white}





% 3. BibLaTeX settings
\usepackage[
  backend=biber,
  style=apa,
  citestyle=authoryear
]{biblatex}
\addbibresource{references.bib}

\title{Meetin with SPoints to discuss with Steve}
%\subtitle{A Mini Literature Overview}

\author{%
 Lucas Condeza
\inst{1} \and
   %\and
%  Coauthor Three\inst{3}
}
\institute{
  \inst{1} Yale University \\
}

\date{\today}

\begin{document}

%\begin{frame}
%  \titlepage
%\end{frame}



\begin{frame}{Last meeting}
Last meeting we talked about:
\begin{itemize}
    \item Institutional setting: centralized annuities market with two stages 1. initial offers and 2. external offers: bargaining (?)  
    \begin{itemize}
        \item Actually NO bargaining [conversations with intermediaries]
    \end{itemize}
    \item Possible information disclosure between first and second stage 
    \begin{itemize}
        \item Not the case [conversations with intermediaries]
    \end{itemize}

    \item Previous research question: what is the impact of the second stage [policy concern, was replaced by an auction]. 
    \begin{itemize}
        \item  many platforms combine posted prices with bargaining (e.g. Zillow)
    \end{itemize}
\end{itemize}
\end{frame}


\begin{frame}{This meeting: research question}
Possible research questions are: 



 \begin{enumerate}
        \item Optimality of firm pricing 

        \begin{itemize}
            \item Firm pricing: 

            $S_{i}-\sum_{t}\hat{p}_{j}(t|x_{i}) \frac{\bar{F}}{(1+\bar{r}_{j})^{t}}=0$
            
            $S$: stock of savings, $F$: per period annuity payment, $x_i$: individual mortality factors 
            \item Firms do not consider 1.demand elasticity, 2. winner's curse 3. selection. 

            \item How important is the miss-pricing?
        \end{itemize}
        
        \item Impact of changing granularity of the informaion ($x_i$)

         \begin{itemize}
            \item E.g. what are the equilibrium effects of adding health information to $x_i$? It could reduce adverse selection 
        \end{itemize}
        
        \item Impact of the seconds stage?  [Varian mdoel of shoppers vs non-shoppers in a differentiated product industry] -> could inform design of insurance markets.

    \end{enumerate}
    




\begin{itemize}
   
   
    \item Research question: is it interesting? 


     
\end{itemize}
\end{frame}



\begin{frame}{This meeting: identification of supply}

\begin{itemize}
    \item  Recursive cost identity plus boundary condition permit identification of $(q_j(\bar{x}, a), z_j)$
    
    \begin{align}
    C_j(a) = z_j \cdot q_j(\bar{x}, a)[1+C_j(a+1).]
    \end{align}

$q_j(\bar{x}, a)$: probability firm $j$ assigns to consumer with characteristics $\bar{x}$ and age $a$ of surviving at least a period.  $ z_j = 1/(1+\bar{r}_j)$

    
    \item How to identify $r_j$? 
     
\end{itemize}
\end{frame}






\end{document}