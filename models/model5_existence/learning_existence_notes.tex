\documentclass[12pt]{article}
%%%%%%%%%%%%%%%%%%%%%%%%%%%%%%%%%%%%%%%%%%%%%%%%%%%%%%%%%%%%%%%%%%%%%%%%%%%%%%%%%%%%%%%%%%%%%%%%%%%%%%%%%%%%%%%%%%%%%%%%%%%%%%%%%%%%%%%%%%%%%%%%%%%%%%%%%%%%%%%%%%%%%%%%%%%%%%%%%%%%%%%%%%%%%%%%%%%%%%%%%%%%%%%%%%%%%%%%%%%%%%%%%%%%%%%%%%%%%%%%%%%%%%%%%%%%
\usepackage{amsfonts}
\usepackage{eurosym}
\usepackage{geometry}
\usepackage{amsmath,amsthm,amssymb}
\usepackage{graphicx}
\usepackage{comment}
\usepackage[utf8]{inputenc}
\usepackage{setspace}
%\usepackage[sort,comma]{natbib}
\usepackage[backend=biber, style = apa]{biblatex}
\usepackage{placeins} % to separate sections

\usepackage{adjustbox}
\usepackage{array}
\usepackage{multirow}
\usepackage{graphicx}
\usepackage{subcaption}
\usepackage{pifont}
\usepackage{amssymb}
\usepackage{comment}
 
\usepackage[hang, flushmargin, bottom]{footmisc}
\usepackage{footnotebackref}
\usepackage{xcolor}
\usepackage{hyperref}
\usepackage{booktabs}
\usepackage{pifont}
\usepackage{caption}
\usepackage{float}
\setlength{\marginparwidth}{2cm} 
\usepackage{todonotes}
\setcounter{MaxMatrixCols}{10}
%TCIDATA{OutputFilter=LATEX.DLL}
%TCIDATA{Version=5.50.0.2960}
%TCIDATA{<META NAME="SaveForMode" CONTENT="1">}
%TCIDATA{BibliographyScheme=BibTeX}
%TCIDATA{LastRevised=Sunday, April 28, 2024 18:12:38}
%TCIDATA{<META NAME="GraphicsSave" CONTENT="32">}
%TCIDATA{Language=American English}

%\setlength{\bibsep}{0.3pt}
\setlength{\textfloatsep}{5pt}
\hypersetup{breaklinks=true,hypertexnames=false,colorlinks=true,citecolor = teal}
\captionsetup{font=normalsize}
\newcommand{\cmark}{\ding{51}}
\def\sym#1{\ifmmode^{#1}\else\(^{#1}\)\fi}
\renewcommand{\thetable}{\Roman{table}}
\geometry{verbose,tmargin=.9in,bmargin=1in,lmargin=1in,rmargin=.9in,nomarginpar}
\makeatletter

\DeclareTextSymbolDefault{\textquotedbl}{T1}
\theoremstyle{plain}
\newtheorem{thm}{Theorem}%[section] commented out to avoid numbering by section
\newtheorem{prop}[thm]{Proposition}
\newtheorem{ass}[thm]{Assumption}
\newtheorem{lemma}[thm]{Lemma}
\newtheorem{theorem}[thm]{Theorem}   % alias for \begin{theorem}
\newtheorem{definition}{Definition}
\makeatother

% \newtheorem{thm}{\protect\theoremname}
% \theoremstyle{plain}
% \newtheorem{prop}[thm]{\protect\propositionname}
% \providecommand{\propositionname}{Proposition}
% \providecommand{\theoremname}{Theorem}
% \makeatother
% \providecommand{\propositionname}{Proposition}
% \providecommand{\theoremname}{Theorem}
% \newtheorem{thm}[thm]{Theorem}
% \newtheorem{ass}[thm]{Assumption}
% \newtheorem{lemma}[thm]{Lemma}  
% \newtheorem{definition}{Definition}
% \newtheorem{theorem}[thm]{Theorem}   % alias so \begin{theorem} works
% \makeatother


% \input{tcilatex}
\usepackage{tikz}
\usetikzlibrary{shapes.geometric, arrows, positioning}





%\addbibresource{../references.bib}
\begin{document}
 
% \title{{\Large Centralized annuities marketplace}}
%\author{Lucas Condeza\thanks{Yale University %\texttt{lucas.schmitz@yale.edu}}} 
%\date{}
%\maketitle

\textcolor{red}{This document has just some notes on the proof of existence, for a proper proof look at other document which uses Athey 2001. }


Consider a simple Cournot game with number of firms $N = 2$, where demand is linear given by $q = 2-p$ and costs are drawn from a uniform distribution $c_i \sim U[0,1]$. Each firm $i$ chooses quantity $q_i$ simultaneously. We look for a symmetric equilibrium. 

Firm $i$ chooses $q_i(c)$ to maximize: 

\begin{align*}
    q_i(c) &= \arg \max_{q} \int_{0}^{1} q \cdot (2-q - q(c_j) - c) dc_j = \arg \max_{q} 2q - q^2 - q\cdot c -  q E(q_j)
\end{align*}
Then the FOC is: 
\begin{align}
    2 - 2q - E(q_j) - c &= 0 \implies q = 1 - \frac{E(q_j) + c}{2}
\end{align}

Applying expectation over the costs we have that: 
\begin{align*}
    E(q_i) &=1 - \frac{E(q_j)}{2} - \frac{E(c)}{2}  \implies E(q_i) = \frac{2}{3} - E(c)/ 3
\end{align*}



\subsection{case 2 }

Previously we took a given demand, now we will consider the case of discrete choice where demand is logit. 

\subsection{Proof of existence of pricing game}

Assumptions:
\begin{enumerate}
    \item Continuous demand function $D_i(p_i, p_{-i})$ that is decreasing in own price and increasing in competitors prices.
    \item Profits are quasi-concave in own price.\footnote{For example for logit demand we have that    
    \[
    \frac{\partial \pi_i}{\partial p_i} =s_i(p) - \alpha(p_i - c_i)s_i(p)(1 - s_i(p))
    \]
    \[
    \frac{\partial^2 \pi_i}{\partial p_i^2} =\frac{\partial s_i}{\partial p_i} - \alpha(1 - s_i)s_i - \alpha(p_i - c_i)\frac{\partial}{\partial p_i}(s_i(1 - s_i))
    \]
    
    Using $\frac{\partial s_i}{\partial p_i} = -\alpha s_i(1 - s_i)$ and $\frac{\partial}{\partial p_i}[s_i(1 - s_i)] = -\alpha s_i(1 - s_i)(1 - 2s_i)$, one obtains $\frac{\partial^2 \pi_i}{\partial p_i^2} < 0$ on $P$. Thus $\pi_i(\cdot, p_{-i})$ is strictly concave, so firm $i$'s best reply $b_i(p_{-i})$ exists and is unique.
    } 
\end{enumerate}


The set of strategies is a compact convex set in a close interval. The price is chosen to be in the interval $[c_i, p^m_i]$. Meaning it is greater than the cost and lower than the monopoly price.

The payoff function is: 
\begin{align*}
    \pi_i(p_i, p_{-i}) &= (p_i - c_i) \cdot D_i(p_i, p_{-i}) 
\end{align*}
which is continuous given the assumption of continuous demand. 


By assumption 2 the best response function is non-empty and convex. If we had assumed that the profit function is strictly concave then the best response correspondence would actually be a function.

Then by Kakutani fixed point theorem there exists a fixed point in the best response correspondence, which is a Nash equilibrium in this pricing game.



\newpage 

\section{Adapt Milgrom and Weber (1985)}
Here we adapt the proof of MW. Types are transformed into costs, actions into prices and payoffs into profits.

In their setting the model consists of the following elements: 
\begin{enumerate}
    \item the set of players: $N = \{1, 2, \ldots, n\}$.
    \item the set of types for each player: $\{C_i\}_{i \in N}$. Each $C_i$ is a complete, separable metric space.
    \item the set of actions available to each player: $\{P_i\}_{i \in N}$. Each $P_i$ is a compact metric space\footnote{In our particular case we assume prices are chosen from the set of positive integers lower than the monopoly price.}. 
    \item the set of possible states: $C_0$, a complete, separable metric space.
Let $C \equiv C_0 \times \cdots \times C_n$ and let $P \equiv P_1 \times \cdots \times P_n$.

    \item the payoff functions: $\{\pi_i\}_{i \in N}$. Each $\pi_i$ is a bounded, measurable function from $C \times P$ into $\mathbb{R}$.
    \item the information structure: $\eta$, a probability measure on the Borel subsets of $C$. Associated with the information structure $\eta$ is a marginal distribution on each $C_i$ which we denote by $\eta_i$. Thus, if $S$ is a Borel subset of $C_i$, then $\eta_i(S) = \eta(C_0 \times S \times C_2 \times \cdots \times C_n)$.
\end{enumerate}


The regularity conditions are: 
\begin{enumerate}
    \item Equicontinuous payoffs: For each firm $i$, and every $\epsilon> 0$ there is a subset $E$ of $C$ such that $\eta(E) > 1- \epsilon. $ and such that the family of functions $\{\pi_i(t, \cdot ) \mid c\in E\}$ is equicontinuous.\footnote{Note that if the cost distribution is continuous and bounded by $1, \overline c_i$ then the first condition is met because there is always some $\eta([\underline c_i + \delta, \overline c_i - \delta]) = 1 - \epsilon/2$. 
    Note that if $\delta =\epsilon \frac{1}{\max_i \overline c_i+ 1 }$ then for any $|x-y|< \delta$ we have that $|\pi_i(t,x)-\pi_i(t,y)|<|D_i x - D_i y| < |x-y| D_i < |x-y| < \epsilon \frac{1}{\max_i \overline c_i+ 1 } < \epsilon$ }

    \item Absolutely continous information: the measure $\eta$ is absolutely continuous with respect to the product measure $\hat \eta \equiv \eta_0 \times \eta_1 \times \cdots \times \eta_n$. We denote the density of $\eta$ with respect to $\hat \eta$ by $f$.
\end{enumerate}

We assume $D_i(p)$ is continuous. 


We can use proposition 1 to show that R1 is satisfied. Particularly, (a) is satisfied because prices are chosen from a set of integers with upper and lower bounds, (b) is 
satisfied since profits are continuous on prices and costs, and the domain of prices and costs is compact, thus by Heine-Cantor\footnote{Heine-Cantor: if $f: M \rightarrow N$ is a continuous function between two metric spaces $M$ and $N$, and $M$ is compact, then $f$ is uniformly continuous. } theorem profits are uniformly continuous. To provoe (c), consider the $\ell_\infty$ metric on $P \subset \mathbb{Z}^N$. Take $\delta(c, \epsilon) = \frac{1}{2}$ for all $c \in C$. If $d_P(p, p') < \delta(c, \epsilon)$ then $p = p'$ and thus $|\pi_i(c, p) - \pi_i(c, p')| = 0 < \epsilon$. Thus $\pi_i(c, \cdot)$ is uniformly continuous with modulus $\delta(c, \epsilon) = 1/2$.  (c) is satisfied. 


Finally, we assume the $c_i$ are independent draws from a continuous distribution with bounded support, hence hence R2 is satisfied. 

By theorem 1 in MW, the pricing game has an equilibrium point in distributional strategies.s 


Then we can use theorem 4 to prove that there is an equilibrium in pure strategies. We show that the conditions are satisfied: 
\begin{enumerate}
\renewcommand{\labelenumi}{(\roman{enumi})}
\item Satisfied since nature does not play in this game ($c_0$ is a singleton) and that the types are independent draws 
\item Given that the cost distributions are continuous they are atomless. 
\item Since other firms' costs do not enter directly into firm $i$'s profit function, 
\item Given by our assumption that prices are chosen from a set of integers 
\item R1 already proved 
\item $C_0$ is a singleton. 
\end{enumerate}

\newpage


\end{document}
