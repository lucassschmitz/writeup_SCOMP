 \documentclass[12pt]{article}
\usepackage{amsfonts}
\usepackage{eurosym}
\usepackage{geometry}
\usepackage{amsmath,amsthm,amssymb}
\usepackage{graphicx}
\usepackage{comment}
\usepackage[utf8]{inputenc}
\usepackage{setspace}
%\usepackage[sort,comma]{natbib}
\usepackage[backend=biber, style = apa]{biblatex}
\usepackage{placeins} % to separate sections

\usepackage{adjustbox}
\usepackage{array}
\usepackage{multirow}
\usepackage{graphicx}
\usepackage{subcaption}
\usepackage{pifont}
\usepackage{amssymb}
\usepackage{comment}
 
\usepackage[hang, flushmargin, bottom]{footmisc}
\usepackage{hyperref}

\usepackage{footnotebackref}
\usepackage{xcolor}
\usepackage{booktabs}
\usepackage{pifont}
\usepackage{caption}
\usepackage{float}
\setlength{\marginparwidth}{2cm} 

\usepackage{todonotes}
\setcounter{MaxMatrixCols}{10}
%TCIDATA{OutputFilter=LATEX.DLL}
%TCIDATA{Version=5.50.0.2960}
%TCIDATA{<META NAME="SaveForMode" CONTENT="1">}
%TCIDATA{BibliographyScheme=BibTeX}
%TCIDATA{LastRevised=Sunday, April 28, 2024 18:12:38}
%TCIDATA{<META NAME="GraphicsSave" CONTENT="32">}
%TCIDATA{Language=American English}

%\setlength{\bibsep}{0.3pt}
\setlength{\textfloatsep}{5pt}
\hypersetup{breaklinks=true,hypertexnames=false,colorlinks=true,citecolor = teal}
\captionsetup{font=normalsize}
\newcommand{\cmark}{\ding{51}}
\def\sym#1{\ifmmode^{#1}\else\(^{#1}\)\fi}
\renewcommand{\thetable}{\Roman{table}}
\geometry{verbose,tmargin=.9in,bmargin=1in,lmargin=1in,rmargin=.9in,nomarginpar}
\makeatletter

\DeclareTextSymbolDefault{\textquotedbl}{T1}
\theoremstyle{plain}
\newtheorem{thm}{Theorem}%[section] commented out to avoid numbering by section
\newtheorem{prop}[thm]{Proposition}
\newtheorem{ass}[thm]{Assumption}
\newtheorem{lemma}[thm]{Lemma}
\newtheorem{theorem}[thm]{Theorem}   % alias for \begin{theorem}
\newtheorem{definition}{Definition}

\makeatother



\newcommand{\sepline}{\par\bigskip\noindent\rule{\linewidth}{0.4pt}\par\medskip}

% \input{tcilatex}
\usepackage{enumitem} % allows custom labels
\usepackage{tikz}
\usetikzlibrary{shapes.geometric, arrows, positioning}

\addbibresource{../references.bib}
\begin{document}
 
\title{\Large Beliefs identification\thanks{We thank...}}
\author{Lucas Schmitz\thanks{Yale University (email: \texttt{lucas.schmitz@yale.edu}). Corresponding author.} \,\,\,\,\,\, and \,\,\,\,\,\, Diego Cussen\thanks{New York University (email: \texttt{dc5004@nyu.edu}).}} 
\date{\today}
\maketitle
 
 

 

\subsection*{Three-period model with distorted survival beliefs and guaranteed annuity}

\paragraph{1. Environment.}

Time is $t=1,2,3$. Each individual has initial wealth (savings) $W>0$ at $t=1$.

\begin{itemize}
    \item True survival probability between any two consecutive periods is $x\in(0,1)$.
    \item Individual beliefs about survival are distorted: the perceived one-period survival probability is
    \[
        \hat x \equiv \theta x,
    \]
    with $\theta>0$ and $\hat x\in(0,1)$.
    \item Period utility from consumption is 
    \[
        u(c) = c^{\alpha}, \qquad \alpha\in(0,1).
    \]
    \item Bequest utility is linear:
    \[
        v(B) = \beta_i B, \qquad \beta_i \ge 0.
    \]
    \item No time discounting.
\end{itemize}

The individual has three options at $t=1$:

\begin{enumerate}
    \item[\textbf{N}] \textbf{No annuity (self-insurance with savings).}
    \item[\textbf{A}] \textbf{Immediate annuity}: pays $F>0$ in each period the individual is alive, with no bequest.
    \item[\textbf{G}] \textbf{Guaranteed annuity}: pays $F_g>0$ in each period the individual is alive; in addition, if the individual dies between periods 1 and 2, the contract pays a guaranteed amount $F_g$ at $t=2$ as a pure bequest (no consumption).
\end{enumerate}

\paragraph{2. No annuity: consumption--savings problem and value.}

If the individual does not buy an annuity, she chooses a consumption--savings plan:
\[
    c_1, c_2, c_3 \ge 0 
\]
subject to $c_1 + c_2 + c_3 =  W$  and $c_1, c_2, c_3 \geq 0 $. 

Given beliefs $\hat x = \theta x$, the expected utility from a given plan
$(c_1,c_2,c_3)$ is
\begin{align*}
    U^{N}(c_1,c_2,c_3; x,\theta,\beta_i)
    &= u(c_1)
       + (1-\hat x)\, \beta_i ( W- c_1)  + \hat x\Big[
            u(c_2)
            + (1-\hat x)\,\beta_i (W- c_1 - c_2)  + \hat x\, u(c_3)
        \Big].
\end{align*}

The value of self-insurance is
\[
    V^{N}(x,\theta,\beta_i,W)
    \equiv
    \max_{\substack{c_1,c_2,c_3\ge 0 \\ c_1+ c_2 + c_3 = W }}
    U^{N}(c_1,c_2,c_3; x,\theta,\beta_i).
\]
Sometimes we also use $s_t$ for savings at the beginning of period $t$, for instance $s_1 = W - c_1$. 

\paragraph{3. Immediate annuity (A).}

If the individual buys the immediate annuity, she uses all wealth $W$ to purchase a contract that pays $F$ in each period she is alive. There is no bequest from the annuity.

Expected utility under the immediate annuity is
\begin{align*}
    V^{A}(x,\theta)
    %&= u(F) + \hat x\, u(F) + \hat x^2\, u(F) \\
    &= \big(1 + \hat x + \hat x^2\big)\, F^{\alpha}.
\end{align*}

\paragraph{4. Guaranteed annuity (G).}

If the individual buys the guaranteed annuity, she uses all wealth $W$ to purchase a contract that:

\begin{itemize}
    \item pays $F_g$ each period she is alive (as with an immediate annuity),
    \item if she dies between $t=1$ and $t=2$, pays a guaranteed amount $F_g$ at $t=2$ as a bequest (no consumption).
\end{itemize}

Hence, the expected utility under the guaranteed annuity is
\begin{align*}
    V^{G}(x,\theta,\beta_i)
    &= u(F_g)
       + \hat x\, u(F_g)
       + \hat x^2\, u(F_g)
       + (1-\hat x)\, v(F_g) \\
    &= \big(1 + \hat x + \hat x^2\big)\, F_g^{\alpha}
       + (1-\hat x)\,\beta_i F_g.
\end{align*}

\paragraph{5. Choice among N, A, and G.}

Given $(x,\theta,\beta_i,W)$, the individual chooses the option with highest expected utility:
\[
    \text{Option chosen} 
    = \arg\max\big\{V^{N}(x,\theta,\beta_i,W),\, V^{A}(x,\theta),\, V^{G}(x,\theta,\beta_i)\big\}.
\]



\newpage 


\subsection*{Non-Identification of $\theta$ and $F_\beta$ in the three-period model}

\paragraph{Setup.}

For each individual $i$:

\begin{itemize}
    \item True one-period survival probability is $x_i\in(0,1)$.
    \item Believed survival is $\hat x_i = \theta x_i$, with $\theta>0$ and $\hat x_i\in(0,1)$.
    \item Bequest weight $\beta_i$ is heterogeneous with CDF $F_\beta$ on $[0,\infty)$.
    
\end{itemize}


The observed choice is
\[
    D_i \in \{A,G,N\}
    \quad\text{with}\quad
    D_i = \arg\max\{V^{A},V^{G},V^{N}\}.
\]
We observe the joint distribution of $(x_i,D_i)$; primitives of interest are $\theta$ and $F_\beta$.

In this case we can prove a non-identification resul, see section \ref{sec:appendix0}. 

\vspace{2cm}
 

\subsection*{Identification with price and wealth shifters}

Previously we showed that data on $(x_i, D_i)$ is not sufficient to identify  $(\theta,F_\beta)$ 





\paragraph{2. Adding exogenous price variation.}

Now allow the annuity payments to depend on an observable ``price state'' $r$ (e.g.\ an
interest rate or term-structure shifter):
\[
F=F(r),\qquad F_g=F_g(r),
\]
with $r$ observed by the econometrician and exogenous:
\[
r\perp (\beta,x,\theta),
\]
and with sufficient support on some compact interval $\mathcal R$.

For each $(x,r,\theta)$ define the payoffs
\[
V^{A}(x,r,\theta) = (1+\theta x + \theta^2 x^2)\, F(r)^\alpha,
\]
\[
V^{G}(x,r,\theta,\beta) = (1+\theta x + \theta^2 x^2)\,F_g(r)^\alpha
                         + (1-\theta x)\,\beta F_g(r),
\]
and $V^{N}(x,r,\theta,\beta,W)$ as before (note: with $W$ fixed, $V^N$ does not depend
on $r$ if the outside savings technology is unchanged).

For each $(x,r,\theta)$ we again obtain two cutoffs
\[
0\le \beta_1(x,r;\theta)\le \beta_2(x,r;\theta)\le\infty,
\]
with
\[
D=
\begin{cases}
A, & \beta\le \beta_1(x,r;\theta),\\
G, & \beta_1(x,r;\theta)<\beta\le \beta_2(x,r;\theta),\\
N, & \beta> \beta_2(x,r;\theta),
\end{cases}
\]
and choice probabilities
\begin{align}
p_A(x,r) &= F_\beta\big(\beta_1(x,r;\theta)\big), \label{eq:pA_xr}\\
p_A(x,r)+p_G(x,r) &= F_\beta\big(\beta_2(x,r;\theta)\big). \label{eq:pAG_xr}
\end{align}

\paragraph{3. Why price variation helps: breaking the simple relabeling.}

Without $r$, the previous non-identification argument rests on the fact that
\[
p_A(x) = F_\beta(\beta_1(x;\theta))
\]
only pins down $F_\beta$ on the \emph{image} of $x\mapsto \beta_1(x;\theta)$, and for any
alternative $\theta_1$ we can reparametrize $F_\beta$ along that image. With $r$, we now
observe $p_A(x,r)$ and $p_A(x,r)+p_G(x,r)$ for a \emph{two-dimensional} regressor
$z=(x,r)$, and the relevant composition is
\[
p_A(z)=F_\beta(\beta_1(z;\theta)),\qquad
p_A(z)+p_G(z)=F_\beta(\beta_2(z;\theta)).
\]

A sufficient condition to break the relabeling is:

\begin{description}
\item[(C1) Rich price support and invertibility.] For each $\theta>0$, the mappings
$z\mapsto \beta_j(z;\theta)$, $j=1,2$, are continuous and strictly monotone in $r$ for
all $x$, with images that overlap across different values of $(x,r)\in\mathcal X\times\mathcal R$.
Formally, for any two points $z_1,z_2$ there exist $z_3,z_4$ such that
\[
\beta_1(z_1;\theta) = \beta_2(z_3;\theta),\qquad
\beta_2(z_2;\theta) = \beta_1(z_4;\theta),
\]
and $z\mapsto \beta_j(z;\theta)$ are invertible on their images.
\end{description}

Intuitively, \((C1)\) says that as interest rates change, the thresholds
$\beta_1(x,r;\theta)$ and $\beta_2(x,r;\theta)$ sweep over the support of $\beta$ in a way
that generates \emph{overlapping evaluations} of $F_\beta$ at common $\beta$ values coming
from different $(x,r)$ pairs.

\paragraph{4. Identification sketch under (C1).}

Let $(\theta_0,F_\beta^0)$ be the true primitives generating the observed probabilities
$\{p_A(x,r),p_G(x,r),p_N(x,r)\}_{(x,r)}$ through
\eqref{eq:pA_xr}--\eqref{eq:pAG_xr}.

Suppose there is another pair $(\theta_1,F_\beta^1)$ that yields the \emph{same} choice
probabilities for all $(x,r)$. Then
\[
F_\beta^0\big(\beta_j(x,r;\theta_0)\big)
=
F_\beta^1\big(\beta_j(x,r;\theta_1)\big),
\qquad
j=1,2,
\]
for all $(x,r)$.

Because $z\mapsto \beta_j(z;\theta_k)$ are invertible on their images, we can reparametrize
these equalities as
\[
F_\beta^1(b)
=
F_\beta^0\big(\beta_j(\phi_j(b;\theta_0,\theta_1);\theta_0)\big),
\quad b\in \mathrm{Im}(\beta_j(\cdot;\theta_1)),\ j=1,2,
\]
for suitable maps $\phi_j$ taking a cutoff under $\theta_1$ back to a $(x,r)$ pair under
$\theta_0$. Under (C1), the images of $\beta_1(\cdot;\theta_1)$ and $\beta_2(\cdot;\theta_1)$
\emph{overlap} on a set of $\beta$ values with positive measure. On that overlap, the
two definitions of $F_\beta^1$ must coincide:
\[
F_\beta^0\big(\beta_1(\phi_1(b;\theta_0,\theta_1);\theta_0)\big)
=
F_\beta^0\big(\beta_2(\phi_2(b;\theta_0,\theta_1);\theta_0)\big)
\quad\text{for all such }b.
\]

Generically, this equality cannot hold for all $b$ unless $\theta_1=\theta_0$. (Any
$\theta_1\neq\theta_0$ implies a different way in which $(x,r)$ map into thresholds
$\beta_1,\beta_2$, so the overlap-induced constraints on $F_\beta^0$ are violated except on
a set of measure zero.) Thus, under (C1), the belief parameter $\theta$ is \emph{point
identified}.

Given $\theta=\theta_0$, equations \eqref{eq:pA_xr}--\eqref{eq:pAG_xr} identify $F_\beta$
nonparametrically on the union of the images of $\beta_1(\cdot;\theta_0)$ and
$\beta_2(\cdot;\theta_0)$ by
\[
F_\beta(b) = p_A\big(z_1^\ast(b)\big),\quad
F_\beta(b) = p_A\big(z_2^\ast(b)\big)+p_G\big(z_2^\ast(b)\big),
\]
for any $z_j^\ast(b)$ satisfying $\beta_j(z_j^\ast(b);\theta_0)=b$, $j=1,2$. With rich
support in $(x,r)$ these images can be made dense in the support of $\beta$, giving
(nonparametric) identification of $F_\beta$ up to standard interpolation.



\section{literature review}

\begin{itemize}
    \item Cormac has a RESTAT that studies the same but is not really the interaction between private info and beliefs, rather that people are pessimistic, but do not study the interaction of beliefs with private information.
\end{itemize}
 

\vspace{5cm}

\section{Identification with Biased Beliefs and Bequest Heterogeneity}



\paragraph{Exclusion and richness assumptions.}


 
    


\subsection{Biased beliefs and the magnitude of selection}

Once $(\theta,F_\beta)$ are identified, the model can be used to quantify how biased beliefs about survival attenuate selection into annuities. Define a simple selection measure:
\[
S(\theta) \;\equiv\; \mathbb{E}\big[x_i \mid D_i\in\{A,G\}\big]
\;-\;
\mathbb{E}\big[x_i \mid D_i = N\big],
\]
the difference between average true survival probability among annuitizers and non-annuitizers, holding the annuity menu $(F(r),F_g(r))$, the wealth distribution, and $F_\beta$ fixed.

In this model, $\theta$ controls how strongly $x_i$ enters perceived values:
\[
\hat x_i = \theta x_i,\quad V^A(x_i,\theta;r)\propto (1+\theta x_i+\theta^2 x_i^2),
\]
and the marginal bequest term in $G$ is proportional to $(1-\theta x_i)\beta_i$. When $\theta$ decreases toward zero, individual decisions become less sensitive to $x_i$:
\begin{itemize}
    \item The differences $V^A(x_i,\theta;r)-V^N(x_i,\theta,\beta_i)$ and $V^G(x_i,\theta,\beta_i;r)-V^N(x_i,\theta,\beta_i)$ become flatter in $x_i$.
    \item The cutoffs $\beta_1(x,r,H;\theta)$ and $\beta_2(x,r,H;\theta)$ vary less with $x$, so the conditional choice probabilities $p_k(x,r,H)$ become less dispersed in $x$.
\end{itemize}
Under mild regularity conditions (single-crossing in $x$), this implies that $S(\theta)$ is strictly increasing in $\theta$: when individuals are closer to rational ($\theta$ larger), high-$x$ types are more likely to buy annuities and low-$x$ types to self-insure; when $\theta<1$, beliefs compress heterogeneity in survival toward the mean, and selection is mechanically weaker.

Empirically, once we have estimated $(\hat\theta,\hat F_\beta)$ and calibrated the annuity menu and wealth distribution from the data, we can compute:
\begin{enumerate}
    \item The \emph{actual} selection $S(\hat\theta)$ implied by the estimated biased-belief model.
    \item A \emph{counterfactual} selection $S(1)$ under rational beliefs ($\theta=1$) but the same $F_\beta$, prices, and wealth distribution.
\end{enumerate}
The difference $S(1)-S(\hat\theta)$ then measures how much observed selection into annuities is dampened by biased beliefs relative to a benchmark with the same bequest heterogeneity but fully rational use of information. If, as you anticipate, $\hat\theta<1$, the model will predict that adverse selection in the Chilean annuity market is significantly smaller than what would arise under rational beliefs, because individuals are effectively more myopic about their survival.

\printbibliography

\vspace{5cm}

\section{Appendix}

\subsection{Non-identification result}\label{sec:appendix0}
 

\begin{ass}[Monotone partition in $\beta_i$]\label{ass:threshold}
    
    Assume that there exist  cutoffs $     0 \le \beta_1(x;\theta) \le \beta_2(x;\theta) $    such that, 
\[
    D_i =
    \begin{cases}
        A, & \text{if } \beta_i \le \beta_1(x_i;\theta),\\[3pt]
        G, & \text{if } \beta_1(x_i;\theta) < \beta_i \le \beta_2(x_i;\theta),\\[3pt]
        N, & \text{if } \beta_i > \beta_2(x_i;\theta).
    \end{cases}
\]
\end{ass}

Assumption \ref{ass:threshold} is intuitive, it says that individuals with a higher bequest motive will buy less insurance since they do not mind leavin savings to their beneficiaries. A sufficient condition is that  $s_1(\beta_i = 0;x)> F_g$. \footnote{Given that $\frac{\partial V^G }{\partial \beta_i} = (1-x) \beta_i F_g $ and that $\frac{\partial V^N}{\partial \beta_i} > \beta_i (1-x)[s_1(\beta_i) + xs_2(\beta_i) ]$ if $s_1(\beta_i = 0;x)> F_g$ then, given that savings are increasing on $\beta_i$, we have that $s_1(\beta_i;x)> F_g$. Then $ \frac{\partial V^G }{\partial \beta_i}(1-x) =\beta_i F_g < \beta_i (1-x)[s_1(\beta_i) + xs_2(\beta_i) ] < \frac{\partial V^N}{\partial \beta_i}$
 
}

Define the conditional choice probabilities
\[
    p_k(x) \equiv \Pr(D_i = k \mid x_i = x),\qquad k\in\{A,G,N\}.
\]
By Assumption \ref{ass:threshold}, the model implies: 
\begin{align}
    p_A(x)
    &= \Pr(\beta_i \le \beta_1(x;\theta))
     = F_\beta\big(\beta_1(x;\theta)\big), \label{eq:pA}\\[4pt]
    p_A(x) + p_G(x)
    &= \Pr(\beta_i \le \beta_2(x;\theta))
     = F_\beta\big(\beta_2(x;\theta)\big), \label{eq:pAG}\\[4pt]
    p_N(x)
    &= 1 - \big[p_A(x)+p_G(x)\big]
     = 1 - F_\beta\big(\beta_2(x;\theta)\big). \label{eq:pN}
\end{align}


\paragraph{ Non-identification of $(\theta,F_\beta)$ (nonparametric $F_\beta$).}

Assume:
\begin{itemize}
    \item[(A1)] For each $\theta>0$, the functions $x\mapsto \beta_1(x;\theta)$ and $x\mapsto \beta_2(x;\theta)$ are strictly monotone and continuous on the support of $x$, hence invertible with inverses $b\mapsto x_1^\theta(b)$ and $b\mapsto x_2^\theta(b)$.
    \item[(A2)] The images of $\beta_1(\cdot;\theta)$ and $\beta_2(\cdot;\theta)$ are disjoint (or can be made disjoint by restricting attention to suitable subsets of $x$ ).
\end{itemize}

In section \ref{sec:appendix1} we provide intuitive conditions under which this assumptions are satisfied. 


Suppose the data are generated by some true pair $(\theta_0,F_\beta^0)$, yielding observed choice probabilities
\[
    p_A(x) = F_\beta^0\big(\beta_1(x;\theta_0)\big),
    \qquad
    p_A(x)+p_G(x) = F_\beta^0\big(\beta_2(x;\theta_0)\big),
\]
for all $x$.

\begin{prop}[Non-identification]
Under (A1)–(A2), for any alternative $\theta_1>0$ there exists a CDF $F_\beta^1$ such that the pair $(\theta_1,F_\beta^1)$ generates the same conditional choice probabilities $\{p_A(x),p_G(x),p_N(x)\}$ for all $x$. Hence $(\theta,F_\beta)$ is not point-identified.
\end{prop}

\begin{proof}
Fix $\theta_1>0$ and consider the cutoff functions $\beta_1(x;\theta_1)$ and $\beta_2(x;\theta_1)$.
By (A1), define inverses $x_1^{\theta_1}(b)$ and $x_2^{\theta_1}(b)$ on the images of $\beta_1(\cdot;\theta_1)$ and $\beta_2(\cdot;\theta_1)$, respectively, such that
\[
    \beta_1\big(x_1^{\theta_1}(b);\theta_1\big)=b, 
    \qquad
    \beta_2\big(x_2^{\theta_1}(b);\theta_1\big)=b.
\]

Define $F_\beta^1$ on these images by
\begin{align*}
    F_\beta^1(b)
    &\equiv p_A\big(x_1^{\theta_1}(b)\big),
    &&\text{for } b\in \mathrm{Im}\big(\beta_1(\cdot;\theta_1)\big),\\[4pt]
    F_\beta^1(b)
    &\equiv p_A\big(x_2^{\theta_1}(b)\big) + p_G\big(x_2^{\theta_1}(b)\big),
    &&\text{for } b\in \mathrm{Im}\big(\beta_2(\cdot;\theta_1)\big).
\end{align*}
Disjointness of the images (A2) guarantees that $F_\beta^1$ is well-defined. Monotonicity of $p_A$ and $p_A+p_G$ in $x$ and of $x_1^{\theta_1}$, $x_2^{\theta_1}$ in $b$ implies $F_\beta^1$ is nondecreasing in $b$ on the union of these images, and it can be extended arbitrarily (but monotonically) to a full CDF on $[0,\infty)$.

Now, for any $x$,
\[
    F_\beta^1\big(\beta_1(x;\theta_1)\big)
    = p_A(x),
\qquad
    F_\beta^1\big(\beta_2(x;\theta_1)\big)
    = p_A(x)+p_G(x).
\]
Hence the model-implied probabilities under $(\theta_1,F_\beta^1)$ satisfy
\[
    p_A(x;\theta_1,F_\beta^1)
    = F_\beta^1\big(\beta_1(x;\theta_1)\big)
    = p_A(x),
\]
and
\[
    p_A(x;\theta_1,F_\beta^1) + p_G(x;\theta_1,F_\beta^1)
    = F_\beta^1\big(\beta_2(x;\theta_1)\big)
    = p_A(x)+p_G(x),
\]
so $p_G(x;\theta_1,F_\beta^1)=p_G(x)$ and $p_N(x;\theta_1,F_\beta^1)=1-p_A(x)-p_G(x)=p_N(x)$ for all $x$.

Therefore the observed conditional choice probabilities do not uniquely determine $(\theta,F_\beta)$: for each $\theta_1>0$ we can construct a CDF $F_\beta^1$ that exactly reproduces the same $\{p_A(x),p_G(x),p_N(x)\}_{x}$.
\end{proof}

\subsection{Microfoundations for (A1) and (A2)}\label{sec:appendix1}

In this subsection we show how the cutoff properties used in Step~3 follow from the
underlying value functions, under mild assumptions on primitives, instead of being imposed
directly on the reduced-form objects \(\beta_1(\cdot;\theta)\), \(\beta_2(\cdot;\theta)\).

\paragraph{Explicit expression and monotonicity of \(\beta_1(x;\theta)\).}

Recall that the \(A\)–\(G\) cutoff \(\beta_1(x;\theta)\) is defined implicitly by
\[
V^A(x,\theta) \;=\; V^G(x,\theta,\beta_1(x;\theta)).
\]
Replacing for the value functions, and rearranging we have that the cutoff solves: 

\begin{align}\label{eq:beta1-explicit}
%    (1+\hat x+\hat x^2) F^\alpha =
%    (1+\hat x+\hat x^2) F_g^\alpha + (1-\hat x)\beta_1(x;\theta) F_g. \notag  \\
%    (1-\hat x)\beta_1(x;\theta) F_g =
%    (1+\hat x+\hat x^2)\bigl(F^\alpha - F_g^\alpha\bigr), \notag \\
    \beta_1(x;\theta) =
    \frac{1+\hat x+\hat x^2}{1-\hat x}\cdot \frac{F^\alpha - F_g^\alpha}{F_g},
\end{align}

 
Under our maintained primitives \(\hat x\in(0,1)\), \(F>0\), \(F_g>0\), and
\(u(c)=c^\alpha\) with \(\alpha\in(0,1)\), \eqref{eq:beta1-explicit} is well-defined and continuous
in \(x\) for any \(\theta>0\). In addition, if we assume that the guaranteed annuity pays a (weakly) lower
flow than the plain annuity,
\begin{equation}
F>F_g>0,
\label{eq:F-Fg-order}
\end{equation}
then \(F^\alpha - F_g^\alpha>0\) and the factor \((F^\alpha-F_g^\alpha)/F_g\) is strictly positive.

To study monotonicity, define
\[
f(\hat x)\;\equiv\;\frac{1+\hat x+\hat x^2}{1-\hat x},\qquad \hat x\in(0,1).
\]
Then
\[
f'(\hat x)
\;=\;
\frac{(1+2\hat x)(1-\hat x) + (1+\hat x+\hat x^2)}{(1-\hat x)^2}
\;=\;
\frac{2+2\hat x - \hat x^2}{(1-\hat x)^2}.
\]
For \(\hat x\in(0,1)\), the numerator satisfies \(2+2\hat x-\hat x^2>2>0\) and the denominator is positive,
so \(f'(\hat x)>0\) on \((0,1)\). Since \(\hat x=\theta x\) with \(\theta>0\), the composite \(x\mapsto f(\theta x)\)
is strictly increasing and continuous on the support of \(x\). Combining this with
\eqref{eq:beta1-explicit} and \eqref{eq:F-Fg-order}, we obtain:

\begin{lemma}[Microfoundations for the \(A\)–\(G\) cutoff]
\label{lem:beta1-micro}
Under \(\hat x=\theta x\in(0,1)\), \(F>F_g>0\), and \(u(c)=c^\alpha\) with \(\alpha\in(0,1)\),
the \(A\)–\(G\) cutoff \(\beta_1(x;\theta)\) is given by \eqref{eq:beta1-explicit} and is
continuous and strictly increasing in \(x\) for any fixed \(\theta>0\).
\end{lemma}

This shows that the monotonicity and continuity of \(\beta_1(\cdot;\theta)\) in Assumption~(A1) are direct implications of the primitives. 

A sufficient condition for the monotonicity of \(\beta_2(x;\theta)\) is that  $s_1(\beta_i = 0;x)> F_g$. Given that $\frac{\partial V^G }{\partial \beta_i} = (1-x) \beta_i F_g $ and that $\frac{\partial V^N}{\partial \beta_i} > \beta_i (1-x)[s_1(\beta_i) + xs_2(\beta_i) ]$ if $s_1(\beta_i = 0;x)> F_g$ then, given that savings are increasing on $\beta_i$, we have that $s_1(\beta_i;x)> F_g$. Then $ \frac{\partial V^G }{\partial \beta_i}(1-x) =\beta_i F_g < \beta_i (1-x)[s_1(\beta_i) + xs_2(\beta_i) ] < \frac{\partial V^N}{\partial \beta_i}$

 

\paragraph{A simple sufficient condition for (A2).}

Recall that (A2) requires the images of the cutoff functions $\beta_1(\cdot;\theta)$ and
$\beta_2(\cdot;\theta)$ to be disjoint (or made disjoint by restricting the range of $x$). A
convenient and transparent sufficient condition is a \emph{uniform separation} of the two
cutoffs in $\beta$.

Let $\mathcal{X}$ denote the support of $x$ and assume $\mathcal{X}$ is compact.

\begin{ass}[Uniform separation of cutoffs]\label{ass:A2prime}
For a given $\theta>0$, suppose there exists a constant $\delta>0$ such that
    \[
        \beta_2(x;\theta) - \beta_1(x;\theta) \;\ge\; \delta
        \qquad \text{for all } x \in \mathcal{X}.
    \]
%\begin{enumerate}
%    \item The cutoff functions $x \mapsto \beta_1(x;\theta)$ and $x \mapsto \beta_2(x;\theta)$
%    are continuous on $\mathcal{X}$.
%    \item There exists a constant $\delta>0$ such that
%    \[
%        \beta_2(x;\theta) - \beta_1(x;\theta) \;\ge\; \delta
%        \qquad \text{for all } x \in \mathcal{X}.
%    \]
%\end{enumerate}
\end{ass}

%Assumption~\ref{ass:A2prime} strengthens the pointwise ordering $\beta_1(x;\theta) < \beta_2(x;\theta)$ by requiring a \emph{uniform} gap $\delta>0$ between the two cutoffs for all $x$. Together with continuity in $x$, this guarantees that the ranges of $\beta_1$ and $\beta_2$ do not overlap.

\begin{lemma}[Assumption~\ref{ass:A2prime} implies (A2)]\label{lem:A2prime-implies-A2}
Fix $\theta>0$ and suppose Assumption~\ref{ass:A2prime} holds. Then the images
$\mathrm{Im}(\beta_1(\cdot;\theta))$ and $\mathrm{Im}(\beta_2(\cdot;\theta))$ are disjoint. In
particular, condition {\rm(A2)} holds.
\end{lemma}

\begin{proof}
Because $\mathcal{X}$ is compact and $\beta_j(\cdot;\theta)$ is continuous for $j=1,2$, each
image $\mathrm{Im}(\beta_j(\cdot;\theta))$ is a compact interval in $\mathbb{R}$. Define
\[
    \overline{\beta}_1 \equiv \sup_{x\in\mathcal{X}} \beta_1(x;\theta),
    \qquad
    \underline{\beta}_2 \equiv \inf_{x\in\mathcal{X}} \beta_2(x;\theta).
\]
By Assumption~\ref{ass:A2prime},
\[
    \beta_2(x;\theta) \;\ge\; \beta_1(x;\theta) + \delta
    \quad \text{for all } x.
\]
Taking the infimum over $x$ on the left-hand side and the supremum over $x$ on the right-hand
side yields
\[
    \underline{\beta}_2
    \;\ge\; \overline{\beta}_1 + \delta.
\]
In particular, $\overline{\beta}_1 < \underline{\beta}_2$. Hence
\[
    \mathrm{Im}(\beta_1(\cdot;\theta)) \;\subseteq\; (-\infty,\overline{\beta}_1],
    \qquad
    \mathrm{Im}(\beta_2(\cdot;\theta)) \;\subseteq\; [\underline{\beta}_2,\infty),
\]
and the two images are disjoint because the right endpoint of the first is strictly smaller
than the left endpoint of the second. This is exactly the content of (A2).
\end{proof}

\end{document}