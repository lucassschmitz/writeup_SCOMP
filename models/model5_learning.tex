\documentclass[12pt]{article}
%%%%%%%%%%%%%%%%%%%%%%%%%%%%%%%%%%%%%%%%%%%%%%%%%%%%%%%%%%%%%%%%%%%%%%%%%%%%%%%%%%%%%%%%%%%%%%%%%%%%%%%%%%%%%%%%%%%%%%%%%%%%%%%%%%%%%%%%%%%%%%%%%%%%%%%%%%%%%%%%%%%%%%%%%%%%%%%%%%%%%%%%%%%%%%%%%%%%%%%%%%%%%%%%%%%%%%%%%%%%%%%%%%%%%%%%%%%%%%%%%%%%%%%%%%%%
\usepackage{amsfonts}
\usepackage{eurosym}
\usepackage{geometry}
\usepackage{amsmath,amsthm,amssymb}
\usepackage{ulem} 
\usepackage{graphicx}
\usepackage{comment}
%\usepackage[sort,comma]{natbib}
\usepackage[backend=biber, style = apa]{biblatex}
\usepackage{placeins} % to separate sections

\usepackage{adjustbox}
\usepackage{array}
\usepackage{multirow}
\usepackage{graphicx}
\usepackage{subcaption}
\usepackage{pifont}
\usepackage{amssymb}
\usepackage{comment}
\usepackage[utf8]{inputenc}
\usepackage{setspace}
\usepackage[hang, flushmargin, bottom]{footmisc}
\usepackage{footnotebackref}
\usepackage{xcolor}
\usepackage{booktabs}
\usepackage{pifont}
\usepackage{caption}
\usepackage{float}
\usepackage{todonotes}
\usepackage{hyperref}

\setcounter{MaxMatrixCols}{10}

%\setlength{\bibsep}{0.3pt}
\setlength{\textfloatsep}{5pt}
\hypersetup{breaklinks=true,hypertexnames=false,colorlinks=true,citecolor = teal}
\captionsetup{font=normalsize}
\newcommand{\cmark}{\ding{51}}
\def\sym#1{\ifmmode^{#1}\else\(^{#1}\)\fi}
\renewcommand{\thetable}{\Roman{table}}
\geometry{verbose,tmargin=.9in,bmargin=1in,lmargin=.8in,rmargin=.8in,nomarginpar}
\makeatletter




\addbibresource{references.bib}
\begin{document}



This model tries to rationalize the increase in the amount offered between the initial offer and the external offers. Broadly the idea is that there will be some learning\footnote{We take a stance on what is generating uncertainty about the other firms' offers. If it is the mortality tables there is a winner's curse, if it is interest rate variation there's no curse. We assume the latter. }. 

We first start by deriving a model without external offers. 

\subsection{Base model }
There is a consumer and $J$ firms. 

Each firm $j$ receives a signal about the cost of consumer $i$, denoted as $\tilde{c}_{j}$.The firms update their beliefs about the cost of consumer $i$ based on this signal.
This signal is drawn from a distribution with pdf $F_j$ mean $c_i$ and variance $\sigma^2$\footnote{The variance can come due to commissions or interest rates} , after observing the signal the firms makes an offer $p_j$.  Denote by $p = (p_1, ..., p_J)$ the vector of offers received by the buyer.  The consumer faces a discrete choice problem where we denote the probability of him choosing $j$ as $D_j(p)$. 

An equilibrium of the pricing game are functions $p_j(c_i)$ such that for all $j$:
\begin{align}
\label{eq:base_equilibrium}
    p_j(c_i) = \arg \max_{p_j} \int_{\mathbb{R}^{N-1}}^{} (p_j - c_j) D_j(p_j, p_{-j}(c_{-j})) dF(p_{-j} \mid c_j)
\end{align} 

where $p_{-j}(c_{-j}) = (p_n(c_n))_{n \neq j}$ and $F(p_{-j} \mid c_j)$ is the distribution of the offers of the other firms given firm's $j$ cost. Note that firm profits in this case are: $\pi_j^{O}(p_j, p_{-j}, c_j) = \int (p_j - c_j) D_j(p_j, p_{-j}(c_{-j})) dF(p_{-j} \mid c_j)$

Where the $O$ supraindex denote the game of only one stage. 

\subsection{Model with external offers}

Now suppose that firms are in the same situation as before, but after making their offers with probability $\lambda$ the consumer chooses to request an external offer. In which case he chooses randomly one firm from which to request the external offer, then the firm can observe the initial offers made by the other firms and decide whether to update its offer \footnote{The firms are not always able to observe the prior offers. In this model we are assuming $\lambda$ is exogenous and the firm observes initial offers. One possible extension is to endogeneize the search behavior (searches if certain conditions are met) and also to assume that sometimes the firm does not observe the prior offers. Note that this could considerably complicate the model because the disclosure of the initial offers is strategic and by itself could reveal somethin, but we do not observe whether the buyer discloses. }.

We will use backward induction to solve the new game. 

In the second stage a firm chooses an offer $p_{j}^{T2}$ equal to the minimum between the initial offer and the optimal offer after observing all the initial offers, i.e. 
\begin{align}
\label{eq:base_equilibrium2}
    p_{j}^{T2}(c_j, p) = \min(p_j^{T1}, p^*)
\end{align}
where the optimal offer is given by: 
\begin{align}
    p^* = \arg \max_{p_j'} (p_j' - c_j) D_j(p_j', p_{-j}) 
\end{align}

Then the expected profits of firm $j$ in the case that the consumer requests an external offer are: 
\begin{align}\label{eq:profits_external}
    \pi_j^{T2}(p_j, p_{-j}, c_j) =  \frac{1}{J} \left( (p_{j,2}(c_j, p) - c_j) D_j(p_{j,2}(c_j, p), p_{-j}) + \sum_{j'\neq j} (p_j - c_j) D_j(p_j, p_{j',2}(c_{j'}, p), p_{-j, -j'}) \right)
\end{align}

where $p_{-j, -j'}$ is the vector of offers excluding firms $j$ and $j'$. And $T2$ denotes that are the second stage profits of the two stage game. 

Define the profits of firm $j$ when the buyer asks them for the external offers: 
\begin{align}
    \pi_j^{(j)}(p_j, p_{-j}, c_j) =  (p_{j,2}(c_j, p) - c_j) D_j(p_{j,2}(c_j, p), p_{-j})
\end{align}
and when the buyer asks another firm $j'$ for the external offer:
\begin{align}
    \pi_j^{(j')}(p_j, p_{-j}, c_j) =  (p_j - c_j) D_j(p_j, p_{j',2}(c_{j'}, p), p_{-j, -j'})
\end{align}
Then we have: 
\begin{align}\label{eq:profits_external2}
    \pi_j^{T2}(p_j, p_{-j}, c_j) =  \frac{1}{J} \left( \pi_j^{(j)}(p_j, p_{-j}, c_j)  + \sum_{j'\neq j} \pi_j^{(j')}(p_j, p_{-j}, c_j) \right)
\end{align}


Then the expected profits of firm $j$ in the first stage are: 
\begin{align}
    \pi_j^{T1}(p_j, p_{-j}, c_j) = (1-\lambda) \pi_j^O(p_j, p_{-j}, c_j) + \lambda \pi_j^{T2}(p_j, p_{-j}, c_j)
\end{align}

An equilibrium of game with external offers are pricing functions  $p_j^T(c_i)$ such that for all $j$:
\begin{align}\label{eq:base_equilibrium3}
    p_j^T(c_i) = \arg \max_{p_j} \int_{\mathbb{R}^{N-1}}^{}  \pi_j^{T1}(p_j, p_{-j}, c_j) dF(p_{-j} \mid c_j)   
\end{align}



\subsection{Comparison of the two models }

Define $g(p_j, p_{-j}) =  D_j + (p_j - c_j) \frac{\partial D_j}{\partial p_j}$ 
Then the FOC of condition \ref{eq:base_equilibrium} is:
\begin{align}
\label{eq:FOC_base} 
    \mathbb{E}\left[ \frac{\partial \pi_j^{O}}{\partial p_j} \right] = \mathbb{E}\left[ g(p_j, p_{-j}) \right] = 0
\end{align}
where the expectation is with respect to $F(p_{-j} \mid c_j)$

In the case with the external offer the FOC condition is: 
\begin{align}
\label{eq:FOC_external} 
    \mathbb{E}\left[ \frac{\partial \pi_j^{T1}}{\partial p_j} \right] = (1-\lambda) \mathbb{E}\left[ g(p_j, p_{-j}) \right] + \lambda \mathbb{E}\left[ \frac{\partial \pi_j^{T2}}{\partial p_j} \right] = 0   
\end{align}


%\vspace{3cm}



From equation  \ref{eq:profits_external2}
\begin{align}
    \frac{\partial \pi_j^{T2}}{\partial p_j} = \underbrace{\frac{1}{J} \frac{\partial}{\partial p_j} \left[ (p_{j,2} - c_j) D_j(p_{j,2}, p_{-j}) \right]}_{\text{Option Value Effect} \geq 0 } + \underbrace{\frac{1}{J} \sum_{j' \ne j} g(p_j, p_{-j}^{(j')})}_{\text{Competitive Threat Effect}}.
\end{align}

The first term is non-negative, since when the buyer requests an external offer to firm $j$, the first stage price acts as an upper bound on the price the firm can set, hence the feasible set is bigger the bigger the initial price.  Essentially the firm has an option value when it sets a higher price. 


We refer to the second term as competitive threat effect because it represents the cases where the competitor's are allowed to update their prices. 

Evaluating the FOC of the two stage game in the prices of the one stage game  we get: 
\begin{align}
    E \left[  \frac{\partial \pi_j^{T1}}{\partial p_j} \mid _{p=p^O} \right] & = (1-\lambda) \underbrace{E[g(p^0)]}_{=0} + \frac{\lambda}{J} E \left[  \frac{\partial(\text{Option Value Profit})}{\partial p_j} \mid_{p=p^{base}} \right] + \frac{\lambda}{J} \sum_{j' \ne j} E[g(p_j^{O}, p_{-j}^{(j')})] \notag \\ 
    & =\frac{\lambda}{J} \left( E \left[  \frac{\partial(\text{Option Value Profit})}{\partial p_j} \mid_{p=p^O} \right] +  \sum_{j' \ne j} E[g(p_j^O, p_{-j}^{(j')})] \right)    
\end{align}

My intuition is that the competitive threat effect reduces prices, because if everyone sets prices according to $p^O$ then in some cases the competitor (upon low realization of costs) will decrease their price in the external offer, and since prices are strategic complements then the initial price set by $j$ will be higher than the optimal. 


\textcolor{red}{I THINK THAT FOR THE CASE with no differentiation we can use \textcite{spulber_bertrand_1995} to show that the prices with the external offer are lower. }





\vspace{3cm}


 
Note that the option value term is always positive, particulary evaluated in $p_j^{T1}$
\begin{align}
    \frac{\partial \pi_j^{T2}}{\partial p_j} =\frac{1}{J} \frac{\partial}{\partial p_j} \left[ (p_{j,2} - c_j) D_j(p_{j,2}, p_{-j}) \right]
\end{align}
 
\begin{align}
    \frac{\partial \pi_j^{T2}}{\partial p_j} = \underbrace{\frac{1}{J} \frac{\partial}{\partial p_j} \left[ (p_{j,2} - c_j) D_j(p_{j,2}, p_{-j}) \right]}_{\text{Option Value Effect} \geq 0 } + \underbrace{\frac{1}{J} \sum_{j' \ne j} g(p_j, p_{-j}^{(j')})}_{\text{Competitive Threat Effect}}.
\end{align}








\section{Simulation}
The parameters to do this are: 
\begin{itemize}
    \item Demand parameters $(\delta_j)_{j=1}^J, \alpha$
    \item Cost distributions $(F_j)_{j=1}^J$
    \item Probability of requesting external offer $\lambda$
\end{itemize}

The solution are firm-specific pricing functions. 

\medskip

We simulate a scenario with $J$ firms, with mean utility $\delta_j$ which is the same for all of them. Costs are drawn from a normal distribution $(\mu_c, \sigma_c)$. %We look for symmetric equilibria. 
Moreover we assume logit demand. 
\subsection{Base model}
Note that the equilibrium consists on $J$ functions, essentially we have to find $(p_j(c_j))_{j=1}^J$. We start with the symmetric case (mean utilities are the same and costs drawn from the same distribution) and search for a parametric pricing function $p(c; \theta)$ which we initially assume has a linear parametrization: $p(c; \theta)= \theta_0 + \theta_1 c$. We also assume that the cost draws are independent. 



\vspace{2cm}

Given that we are looking for a symmetric equilibria our algorithm assumes that $J-1$ firms are playing a strategy parametrized by $\theta$ and then 
the we aim to minimize the distance: 
\begin{align}
    d(\theta, c_j) = p_j(c_j; \theta) -\underbrace{\arg \max_{p_j} \int_{\mathbb{R}^{N-1}}^{} (p_j - c_j) D_j(p_j, p_{-j}(c_{-j}; \theta_0)) dF(c_{-j} \mid c_j)}_{p_j(c_j)}
\end{align} 
where the first term represent the strategy implied by our guess and the second term represents optimal pricing 


We consider $C$ points in a grid in the range $[\underline{c}, \bar{c}] $

\begin{enumerate}
    \item Make an initial guess $\theta_0$. 
    \item For each cost in the grid we calculate the optimal response by solving: 
    \begin{align}
        p_j(c_j) = \arg \max_{p_j} \int_{\mathbb{R}^{N-1}}^{} (p_j - c_j) D_j(p_j, p_{-j}(c_{-j}; \theta_0)) dF(c_{-j})
    \end{align}  
    where the integral is calculated over the grid of costs. 
    \item We calculate a distance 
    \begin{align}
        \int d(\theta, c_j) dF(c_j)
    \end{align}
    
\end{enumerate}
Strategy to solve for the model. 

 
\begin{align}
\label{eq:base_equilibrium}
    p_j(c_j) = \arg \max_{p_j} \int_{\mathbb{R}^{N-1}}^{} (p_j - c_j) D_j(p_j, p_{-j}(c_{-j})) dF(c_{-j} \mid c_j)
\end{align} 









\end{document}