\documentclass[12pt]{article}
%%%%%%%%%%%%%%%%%%%%%%%%%%%%%%%%%%%%%%%%%%%%%%%%%%%%%%%%%%%%%%%%%%%%%%%%%%%%%%%%%%%%%%%%%%%%%%%%%%%%%%%%%%%%%%%%%%%%%%%%%%%%%%%%%%%%%%%%%%%%%%%%%%%%%%%%%%%%%%%%%%%%%%%%%%%%%%%%%%%%%%%%%%%%%%%%%%%%%%%%%%%%%%%%%%%%%%%%%%%%%%%%%%%%%%%%%%%%%%%%%%%%%%%%%%%%
\usepackage{amsfonts}
\usepackage{eurosym}
\usepackage{geometry}
\usepackage{amsmath,amsthm,amssymb}
\usepackage{ulem} 
\usepackage{graphicx}
\usepackage{comment}
%\usepackage[sort,comma]{natbib}
\usepackage[backend=biber, style = apa]{biblatex}
\usepackage{placeins} % to separate sections

\usepackage{adjustbox}
\usepackage{array}
\usepackage{multirow}
\usepackage{graphicx}
\usepackage{subcaption}
\usepackage{pifont}
\usepackage{amssymb}
\usepackage{comment}
\usepackage[utf8]{inputenc}
\usepackage{setspace}
\usepackage[hang, flushmargin, bottom]{footmisc}
\usepackage{footnotebackref}
\usepackage{xcolor}
\usepackage{hyperref}
\usepackage{booktabs}
\usepackage{pifont}
\usepackage{caption}
\usepackage{float}
\usepackage{todonotes}
\setcounter{MaxMatrixCols}{10}
%TCIDATA{OutputFilter=LATEX.DLL}
%TCIDATA{Version=5.50.0.2960}
%TCIDATA{<META NAME="SaveForMode" CONTENT="1">}
%TCIDATA{BibliographyScheme=BibTeX}
%TCIDATA{LastRevised=Sunday, April 28, 2024 18:12:38}
%TCIDATA{<META NAME="GraphicsSave" CONTENT="32">}
%TCIDATA{Language=American English}

%\setlength{\bibsep}{0.3pt}
\setlength{\textfloatsep}{5pt}
\hypersetup{breaklinks=true,hypertexnames=false,colorlinks=true,citecolor = teal}
\captionsetup{font=normalsize}
\newcommand{\cmark}{\ding{51}}
\def\sym#1{\ifmmode^{#1}\else\(^{#1}\)\fi}
\renewcommand{\thetable}{\Roman{table}}
\geometry{verbose,tmargin=.9in,bmargin=1in,lmargin=.8in,rmargin=.8in,nomarginpar}
\makeatletter
\DeclareTextSymbolDefault{\textquotedbl}{T1}
\theoremstyle{plain}
\newtheorem{thm}{\protect\theoremname}
\theoremstyle{plain}
\newtheorem{prop}[thm]{\protect\propositionname}
\providecommand{\propositionname}{Proposition}
\providecommand{\theoremname}{Theorem}
\makeatother
\providecommand{\propositionname}{Proposition}
\providecommand{\theoremname}{Theorem}
\newtheorem{ass}[thm]{Assumption}
% \input{tcilatex}
\usepackage{tikz}
\usetikzlibrary{shapes.geometric, arrows, positioning}





\addbibresource{references.bib}
\begin{document}
Section \ref{sec:suboptimality} and \ref{sec:suboptimality2} present the pricing formula of insurers operating in the annuities market and show -under different assumptions- that the pricing formula is not optimal. 

Section \ref{sec:identification} discusses the identification of the different parameters that enter the pricing formula.

\begin{itemize}
    \item Individuals are indexed by $i$ and firms by $j$ 
    \item Firms, when making offers, have the same information\footnote{This information is provided by the centralized system, firms can not acquire additional information.} about the buyers.\\
    $x_{i}=\{a_i,g_i,family \  composition,S_{i}\}$ where $S_{i}$ are savings, $g_i$ is gender, and $a_i$ is age. 
    \item Given $x_{i}$, there is a probability that the individual lives at least $t$ additional periods, which we denote by $p(t|x_{i})$ 
    \item $p(t|x_{i})$  is not observed by firms, they aim to estimate it by constructing their own  mortality tables\footnote{Less sophisticated (smaller) firms rely on the tables of the regulator. }. They are conditional probabilities of being alive in $t$ periods given the type $x_{i}.$ We denote this conditional probability as $\hat{p}_{j}(t|x_{i})$.
     Firms also have different financing costs, given by $r_{j}$. 

    \item Expected firm profits when selling an annuity that pays a flow F to individual $i$ is: 

\begin{equation}\label{eq:profit}
\pi_{ij}(F)=S_{i}-\sum_{t}p(t|x_{i})\frac{F}{(1+r_{j})^{t}}=S_{i}-\mathbb{E}\left[\sum_{t}\frac{F}{(1+r_{j})^{t}}\right]=S_{i}-Fc_{ij}
\end{equation}
\end{itemize}

where we defined $c_{ij}=\mathbb{E}\left[\sum_{t}\frac{1}{(1+r_{j})^{t}}\right]$the
cost of an unitary annuity. $F$ is the flow payment offered by the firm. I will refer to $F$ as the price, because
$S_{i}/F$ is the price paid by the consumer for an unitary annuity\footnote{A unitary annuity pays $1$ every period.}.
But $c_{ij}$ is unknown by the firm, since it is constructed using the
actual survival probabilities. We define $\hat{c}_{ij}=\sum_{t}\hat{p_{j}}(t|x_{i})\frac{1}{(1+r_{j})^{t}}=\hat{\mathbb{E}_{j}}\left[\sum_{t}\frac{1}{(1+r_{j})^{t}}\right]$
as the unitary cost given the firm's information.


\section{Optimal pricing vs. used pricing}\label{sec:suboptimality}

In this section we present the optimal pricing under different assumptions and compare the optimal pricing under each assumption with the pricing formula used in the industry. 


\subsection{Assumption 1: Independent private costs and known $F_{-j}$ }\label{sec:independent_private}

Our initial assumption is that costs are independent ($\mathbb{E}[c_{ij}|\hat{c}_{ij}, \hat{c}_{ij'}] = \mathbb{E}[c_{ij}|\hat{c}_{ij}]$) and that the firm knows the offers made by the other firms\footnote{One case that would satisfy our assumption is the case where the firms know $p(t|x_i)$ and the pricing formula of the competitors.}. 



The per customer profits are:  
\begin{equation}\label{eq:profit3}
\hat{\pi}_{ij}(F)= S_{i}-F\hat{c}_{ij}
\end{equation}

\begin{itemize}

\item A profit maximizer chooses:  
\begin{align}\label{eq:optimal_pricing}
F_j&=\arg\max_{F}s_{ij}(F;F_{-j})\hat{\pi}_{ij}(F;F_{-j})
%&\implies s_{j}'(F)\left(S_{i}-F\hat{c}_{ij}\right)=s_{j}(F)\hat{c}_{ij}
\implies F_j=\frac{S_{i}}{\hat{c}_{ij}}-\frac{s_{ij}(F)}{s_{ij}'(F)}
\end{align}
where $s_{ji}$ is the market share among consumers type $i$. 



\item But from conversations with practitioners, what firms actually do
is to define a firm-wide Internal Rate of Return (IRR), let's call
it $\bar{r}_{j}$ and choose $\bar{F}$ such that: 

\begin{equation}\label{eq:actual_pricing}
S_{i}-\sum_{t}\hat{p}_{j}(t|x_{i}) \frac{\bar{F}}{(1+\bar{r}_{j})^{t}}=0
\implies S_{i}-\hat{\mathbb{E}}_j\left[\sum_{t}\frac{\bar{F}}{(1+\bar{r_{j}})^{t}}\right]=0
\implies\frac{S_{i}}{\bar{c}_{ij}}=\bar{F}
\end{equation}

where we defined $\bar{c}_{ij}=\hat{\mathbb{E}}_{j}\left[\sum_{t}\frac{1}{(1+\bar{r}_{j})^{t}}\right]$.

 
Note that $\bar{r}_{j}\geq r_{j}$ because the firm includes the markups in the implicit interest rate.  

\item Replacing the actual price (equation \ref{eq:actual_pricing}) in the optimal pricing condition (equation \ref{eq:optimal_pricing}) we have
that the price is optimal iff: 
\begin{equation}\label{eq:comparison}
%\frac{S_{i}}{\bar{c}_{ij}} =\frac{S_{i}}{\hat{c}_{ij}}-\frac{s_{j}\left(\frac{S_{i}}{\bar{c}_{ij}}\right)}{s_{j}'\left(\frac{S_{i}}{\bar{c}_{ij}}\right)} \implies
\frac{S_{i}}{\hat{c}_{ij}}-\frac{S_{i}}{\bar{c}_{ij}}=\frac{s_{j}\left(\frac{S_{i}}{\bar{c}_{ij}}\right)}{s_{j}'\left(\frac{S_{i}}{\bar{c}_{ij}}\right)}
\end{equation}
\item From equation \ref{eq:comparison} we see that if there is perfect competition (demand perfectly elastic) and the firm sets the IRR equal to the financing cost,  then the actual pricing scheme
is optimal. 
\end{itemize}

The suboptimality of the pricing of equation (\ref{eq:actual_pricing}) comes from not considering the price elasticity of type $i$ individuals into the pricing decision. 
The way firms are making a profit is by introducing the markup in the IRR, which does not allow them to choose the optimal markup based on the consumer type $i$ price elasticity. 

 



\subsection{Assumption 2: Independent private costs with unknown $F_{-j}$ }\label{sec:interdependent}

Our second assumption is  that the bids made by the other firms are not known, but knowing them does not change the expected unitary cost($\mathbb{E}[c_{ij}|\hat{c}_{ij}, \hat{c}_{ij'}] = \mathbb{E}[c_{ij}|\hat{c}_{ij}]$). 
This assumption deserves an explanation. Since offers are constructed using the mortality tables and the IRR, observing the rival's offers reveals information about this objects. But the mortality tables are constructed on a yearly basis and the IRR changes on a monthly or even weekly basis \footnote{From conversation with Renato Aburto, CFO of Consorcio, the second biggest insurer.}, hence we assume that in equilibrium the rival's offers only provide additional information about the IRR and not about the mortality tables used by other firms\footnote{We do not assume that the mortality tables of the competitors are know, just that the bids do not provide additional information. Given that less sophisticated firms do not even invest in producing their own mortality tables, it is plausible that they also are not investing in estimating the mortality tables the other firms' are using, which is more challenging since it involves separately identifying the IRR and the mortality tables. }. 

Define $G(F_{-j})$, the cdf of the other firms offers. Then, the firm when setting prices for customers of type $i$ maximizes: 
\begin{align*}
    \int_{F_{-j}} s_{ij}(F; F_{-j})\hat{\pi}_{ij}(F;F_{-j}) dG(F_{-j})
\end{align*}


then the price FOC is: 
\begin{align*}
    \int_{F_{-j}} \left[s_{ij}'(F; F_{-j}) \left(S_{i}-F\hat{c}_{ij}\right)- s_{ji}(F; F_{-j}) \hat{c}_{ij}  \right]dG(F_{-j})=0
\end{align*} 


Using, the actual prices from equation (\ref{eq:actual_pricing}), the firm is acting optimally iff: 
\begin{align}\label{eq:comparison2}
    \int_{F_{-j}} \left[s_{ij}'(\bar{F}; F_{-j}) \left(S_{i}-\frac{S_{i}}{\bar{c}_{ij}}\hat{c}_{ij}\right)- s_{ij}(\bar{F}; F_{-j}) \hat{c}_{ij}  \right]dG(F_{-j})=0
\end{align}   
 
 
Under the assumption of independent private  costs with unknown $F_{-j}$, the firm's simple pricing rule is suboptimal for two distinct reasons. First, the critique from the previous section still holds: even if the firm knew the competitors' offers, its formula would be incorrect because it  ignores the price elasticity of demand. Second, the pricing is flawed because it does not incorporate the information about the way competitors are pricing (i.e. $G(F_{-j})$). 



 
 


\section{Selection}\label{sec:suboptimality2}



Previously we assumed that $p(t|x_{i})$ does not depend on prices and is not firm specific, which are the assumptions followed by the firms when pricing their products\footnote{To construct their mortality tables, firms use the sample of annuity buyers. [Conversation with Ulises Rubio, former Director of Actuarial department at Metlife, biggest insuer.] }. Generally, there could be selection across firms, which could depend on prices, in which case survival probabilities would have to be written as: $p_j(t|x_{i}, (F_j, F_{-j}))$. 

Conditioning survival probabilities on $(j, F)$  allows for selection across firms and into the annuities market. \textcite{illanes_retirement_2019}, using data of the same centralized annuities market, find selection into the market\footnote{Across firm selection implies that the survival probabilities depend on prices. Take the case where all the annuities prices go to zero, then all consumers choose annuities and the survival probabilities among annuity holders equal the ones in the population. Since with the current prices the annuities holders do not have the same survival probabilities as the population, then the survival probabilities depend on prices.}, which is expected if buyers have private information about their health. But is not obvious what would generate selection across firms. 

In what follows we define what we mean by selection across firms and into the market and propose a stylized model that generates only the latter. 

We denote by $j=0$ the outside option of not buying an annuity. 

\textbf{Def: selection across firms} 
  $\exists \ j,j'\neq 0$ and a vector $(t,x_{i},F)$ such that  $p_j(t|x_{i}, F) \neq p_{j'}(t|x_{i}, F)$

\textbf{Def: selection into the market} 
for all $(j,j'\neq 0 )$ and $(t,x_{i},F)$ we have: 
$p_j(t|x_{i}, F) = p_{j'}(t|x_{i}, F)$ but there exists a vector $(j, t,x_{i},F)$ such that $p_j(t|x_{i}, F) = p_{0}(t|x_{i}, .)$


\textbf{Def: no selection} 
There is no selection  when $\forall (t,x_{i}) $ there exists a 
survival probability $p(t|x_{i})$ such that :   $p_j(t|x_{i}, (F_j, F_{-j})) = p(t|x_{i}), \forall F,j $



\smallskip

The three definitions of selection presented are not exhaustive. In what follows we present what the different types of selection would look like in a simple nested logit model. 

\medskip






\subsection*{Nested Logit Model for Pension Choice}

Here we specify a nested logit model for an individual's choice between an Annuity($A$) and the outside option ($0$). The choice of an Annuity provider represents the inner nest, while the  decision between the Annuity and $j=0$ constitutes the outer nest.

%\textbf{The Inner Nest}

Denote by $V_{ij} = f(h_i, F_j)$ the mean utility of a buyer of type $i$ when buying from insurer $j$, where $h_i$ is private health information. The probability of choosing $j$ given selection of $A$ is: 
\begin{equation}\label{eq:inner_prob}
    P(j|A) = \frac{e^{V_{ij} / \lambda}}{\sum_{k=1}^{J} e^{V_{ik} / \lambda}} = \frac{1}{\sum_{k=1}^{J} e^{(f(h_i, F_k)-f(h_i, F_j)) / \lambda}}
\end{equation}

where $\lambda\in[0,1]$ measures the correlation of unobserved utility among insurers. 

%\textbf{The Outer Nest}

Assume that the utility of the outside option is: $V_{i, 0 } + \epsilon_{i, 0}$ and define the inclusive value of the annuities: 

%Moreover, the inclusive value of the annuity option is: 
\begin{equation}\label{eq:inclusive}
    I_{i, \text{A}} = \ln\left(\sum_{k=1}^{J} e^{V_{ik} / \lambda}\right)
\end{equation}

%The utility for the Annuity nest is then:
%\begin{equation}
%    U_{i, A} = V_{i, A} + \lambda I_{i, A} + \epsilon_{i, A}
%\end{equation}
where $V_{i, \text{A}}$ depends on attributes common to all annuities.


The probability that individual $i$ chooses the Annuity nest is given by:

\begin{equation}
    P(A) = \frac{e^{V_{i,A} + \lambda I_{i, A}}}{e^{V_{i, 0}} + e^{V_{i, A} + \lambda I_{i, A}}}
\end{equation}


The unconditional probability of choosing firm $j$ is calculated as the product of the probability of choosing the Annuity nest and the conditional probability of choosing firm $j$ within that nest.
\begin{equation}
    P_{ij} = P(A) \times P(j|A) =
     \left( \frac{e^{V_{i, A} + \lambda I_{i, A}}}{e^{V_{i, 0}} + e^{V_{i, A} + \lambda I_{i, A}}} \right) \times \left( \frac{e^{V_{ij} / \lambda}}{\sum_{k=1}^{J} e^{V_{ik} / \lambda}} \right)
\end{equation}

\textbf{Commentary}


\begin{itemize}
    \item In equation (\ref{eq:inner_prob}) a simple condition to avoid having selection across firms ($h_i$ independent of choice probability) is to assume a utility function that is additively separable in health and price (e.g. $f(h_i, F_j) = m_1(h_i) + m_2(F_j)$). In this case we have that the utility difference between firms is: 
    \begin{align*}
    f(h_i, F_k) - f(h_i, F_j) = ( m_1(h_i) + m_2(F_k)) - ( m_1(h_i) + m_2(F_j)) = m_2(F_k) - m_2(F_j)
    \end{align*}

    Replacing in equation (\ref{eq:inner_prob}) we have: 
    
    \begin{equation}\label{eq:inner_prob2}
    P(j|A, h_i) = \frac{1}{\sum_{k=1}^{J} e^{(f(h_i, F_k)-f(h_i, F_j)) / \lambda}} =  \frac{1}{\sum_{k=1}^{J} e^{(m_2(F_k) - m_2(F_j)) / \lambda}} = P(j|A)
    \end{equation}

    \item The additively separable assumption above, will generally generate selection into the market since the inclusive value of the annuities (equation \ref{eq:inclusive} will depend on health status). 

    \item With adverse selection, since the marginal buyer might be switching from the outside option, its cost will be lower than the average buyer's. Since   mortality tables are estimated on the population of annuity buyers, the costs will be overestimated.  
\end{itemize}


\section{Identification}\label{sec:identification}


In this section we discuss how to identify $\hat{p}_{j}(t|x_{i})$  and  $\bar{r_{j}}$ from our data. 


\textbf{Data}

Our data consists on offers made by the firms. For each individual $i$ we have their characteristics ($x_i$) and the vector of offers they received. Hence our data is:  $(i, x_i, (\bar{F}_1, ..., \bar{F}_J))_{i=1}^I$ 

 

\textbf{Identification}

With our data we want to estimate 
$((\hat{p}_{j}(t|x_{i}))_{x_i,t},  \bar{r_{j}})_{j=1}^J$ 

Define the partition of   $x_{i}$ into age and the other variables $x_{i}=\{\bar{x}_i, a_i\}$. Moreover define $q_{j}(\bar{x}_i, a_i) \equiv \hat{p}_{j}(1|x_i) $, the one-period survival probability, then: 

\begin{align}\label{eq:mort_form}
    \hat{p}_{j}(t|x_i)
    %&= q(\bar{x}_i, a_i)\cdot q(\bar{x}_i, a_i+1) \cdot q(\bar{x}_i, a_i+2) \cdot ... \cdot q(\bar{x}_i, t-1) \notag \\
    &= \prod_{a= a_i}^{t-1}q(\bar{x}_i, a)
\end{align}

Define $z_j \equiv (1+\bar{r}_j)^{-1}$. In what follows we focus only on one group of $\bar{x}_i$, hence is not included as argument in the function. Denote by $C_j(a)$ the  cost of an unitary annuity given buyer age $a$\footnote{The cost calculation assumes the annuity starts paying next period and discounts flow by the IRR.}: 
\begin{align}\label{eq:ident_costs}
    C_j(a) &= \sum_{t=1}^{\infty} z_j^t \prod_{u=0}^{t-1} q_j(\bar{x}, a+u) 
\end{align}

Since $C_j(a) =\sum_{t}\hat{p}_{j}(t|x_{i}) \frac{1}{(1+\bar{r}_{j})^{t}} = \hat{\mathbb{E}}_{j}\left[\sum_{t}\frac{1}{(1+\bar{r}_{j})^{t}}\right] = \bar{c}_{ij}$ using the  actual pricing equation (\ref{eq:actual_pricing}) we have  $\frac{S_i}{C_j(a)} = \bar{F}$. Hence we can recover $C_j(a)$ from the observed offers.


It is possible to prove that 


\begin{align}\label{eq:identification1}
    C_j(a) = z_j \cdot q_j(\bar{x}, a)[1+C_j(a+1).]
\end{align}

From equation \label{eq:identification1} we can identify the product $m_j(a) = z_j\cdot q_j(\bar{x}, a)$ from: 

\begin{align}\label{eq:identification2}
     m_j(a) =z_j \cdot q_j(\bar{x}, a)=\frac{C_j(a)}{1+C_j(a+1)}
\end{align}
but we can not identify the probabilities and $z_j$ separately.




From our institutional context we know  that   $ q_j(\bar{x},110) =0$, but this condition is not useful to identify the probabilities separately from $z_j$ because it implies that $m_j(110) =0$. 

But if we knwe $ q_j(\bar{x},a_0)$ for a non-terminal $a_0$ then we would be able to identify all the parameters. One possibility is to assume that for certain age the firms use the same tables as the regulator. 

In this case, from equation (\ref{eq:identification1}), we could identify $z_j$ from 
\begin{align}\label{eq:identification3}
     z_j  =\frac{C_j(a)}{ q_j(\bar{x}, a_0)[1+C_j(a+1)]}
\end{align}

And once we have $z_j$ we can use equation \ref{eq:identification2} to estimate the probabilities. 

\end{document}