\documentclass[12pt]{article}
%%%%%%%%%%%%%%%%%%%%%%%%%%%%%%%%%%%%%%%%%%%%%%%%%%%%%%%%%%%%%%%%%%%%%%%%%%%%%%%%%%%%%%%%%%%%%%%%%%%%%%%%%%%%%%%%%%%%%%%%%%%%%%%%%%%%%%%%%%%%%%%%%%%%%%%%%%%%%%%%%%%%%%%%%%%%%%%%%%%%%%%%%%%%%%%%%%%%%%%%%%%%%%%%%%%%%%%%%%%%%%%%%%%%%%%%%%%%%%%%%%%%%%%%%%%%
\usepackage{amsfonts}
\usepackage{eurosym}
\usepackage{geometry}
\usepackage{amsmath,amsthm,amssymb}
\usepackage{ulem} 
\usepackage{graphicx}
\usepackage{comment}
%\usepackage[sort,comma]{natbib}
\usepackage[backend=biber, style = apa]{biblatex}
\usepackage{placeins} % to separate sections

\usepackage{adjustbox}
\usepackage{array}
\usepackage{multirow}
\usepackage{graphicx}
\usepackage{subcaption}
\usepackage{pifont}
\usepackage{amssymb}
\usepackage{comment}
\usepackage[utf8]{inputenc}
\usepackage{setspace}
\usepackage[hang, flushmargin, bottom]{footmisc}
\usepackage{footnotebackref}
\usepackage{xcolor}
\usepackage{hyperref}
\usepackage{booktabs}
\usepackage{pifont}
\usepackage{caption}
\usepackage{float}
\usepackage{todonotes}
\setcounter{MaxMatrixCols}{10}
%TCIDATA{OutputFilter=LATEX.DLL}
%TCIDATA{Version=5.50.0.2960}
%TCIDATA{<META NAME="SaveForMode" CONTENT="1">}
%TCIDATA{BibliographyScheme=BibTeX}
%TCIDATA{LastRevised=Sunday, April 28, 2024 18:12:38}
%TCIDATA{<META NAME="GraphicsSave" CONTENT="32">}
%TCIDATA{Language=American English}

%\setlength{\bibsep}{0.3pt}
\setlength{\textfloatsep}{5pt}
\hypersetup{breaklinks=true,hypertexnames=false,colorlinks=true,citecolor = teal}
\captionsetup{font=normalsize}
\newcommand{\cmark}{\ding{51}}
\def\sym#1{\ifmmode^{#1}\else\(^{#1}\)\fi}
\renewcommand{\thetable}{\Roman{table}}
\geometry{verbose,tmargin=.9in,bmargin=1in,lmargin=.8in,rmargin=.8in,nomarginpar}
\makeatletter
\DeclareTextSymbolDefault{\textquotedbl}{T1}
\theoremstyle{plain}
\newtheorem{thm}{\protect\theoremname}
\theoremstyle{plain}
\newtheorem{prop}[thm]{\protect\propositionname}
\providecommand{\propositionname}{Proposition}
\providecommand{\theoremname}{Theorem}
\makeatother
\providecommand{\propositionname}{Proposition}
\providecommand{\theoremname}{Theorem}
\newtheorem{ass}[thm]{Assumption}
% \input{tcilatex}
\usepackage{tikz}
\usetikzlibrary{shapes.geometric, arrows, positioning}





\addbibresource{references.bib}
\begin{document}
\textbf{Some notes}


\todo[inline]{Note that in insurance markets (just as in markets where there is a fixed cost) marginal cost is decreasing. In markets with a fixed cost firms can still make a profit because the fixed creates barriers to entry, hence equilibrium profits equal the fixed cost. But in an insurance market average cost is also decreasing, the difference is that there are no barriers to entry, hence I think that you do not even need many problesm with adverse selection. There could be a market where a monopolist charging their actual cost could survive but the }

\newpage

Assume that buyers differ in their health status, all sellers receive the same signal about the status. We study competition given a signal, which in our institutional context is equivalent to study the market of a particular group (e.g. women with a certain amount of savings and a certain age). 



\section{Perfect information}

We first present the demand side then we present the costs of insurers and the optimal pricing strategy. Finally we present equilibrium market shares. 

\subsection{Consumer choice}

Following \textcite[p. 21 and 22]{hortacsu_structural_2023} consider a standard nested logit with two modules (outer nest): the outside option (Phased Withdrawal) and buying an annuity. And the inner nest consists on choosing among the insurers. 

Denote by $\delta_j$ the mean utility of each option, where $j =0$ represents the outside option. 
Then the probability of choosing insurer $j$ conditional on buying an annuity is 

\begin{equation}\label{eq:inner_prob}
    s_{j\mid A} = \frac{\exp(\delta_{j} / \rho)}{\sum_{k=1}^{J} \exp(\delta_{k} / \rho)}  
\end{equation}
where $\rho \in [0,1]$ measures the degree of correlation among the alternatives, if $\rho =1 $ the model is a standard logit (without nests)\footnote{See \href{https://eml.berkeley.edu/choice2/ch4.pdf}{here}.}. 
And the probability of buying an annuity is: 


\begin{equation}\label{eq:outer_nest}
    s_A = \frac{\alpha_A\left(\sum_{j=1}^J \exp(\delta_j/\rho)\right)^\rho }{\alpha_A\left(\sum_{j=1}^J \exp(\delta_j/\rho)\right)^\rho +  \exp \left(\delta_0 \right)}
\end{equation}
Hence, the probability of choosing a particular $j\neq 0$ is: 

\begin{equation}\label{eq:choice_prob}
    s_j = s_{j\mid A} \cdot s_A = 
    \frac{\exp(\delta_{j} / \rho)}{\sum_{k=1}^{J} \exp(\delta_{k} / \rho)}  \cdot \left( \frac{\alpha_A\left(\sum_{j=1}^J \exp(\delta_j/\rho)\right)^\rho }{\alpha_A\left(\sum_{j=1}^J \exp(\delta_j/\rho)\right)^\rho +  \exp \left(\delta_0 \right)}\right)
\end{equation}

Define $h_i$ the health status of buyer $i$, which is private information. Given that $h_i$ affects the expected number of payments when buying the annuity the mean utility of the insurers will depend on it, $\delta_j(h_i)$, where we assume that the mean utility is increasing on the health information. 

Assume that, for any $j\neq 0$,  $\delta_j = f(h_i) + \tilde{\delta}_j$, with $f(\cdot)$ being strictly increasing,  but health status does not affect the outside option. 


Then we have:
\begin{equation}\label{eq:no_selection}
    s_{j\mid A}(h_i) = \frac{\exp((f(h_i) + \tilde{\delta}_j) / \rho)}{\sum_{k=1}^{J} \exp((f(h_i) + \tilde{\delta}_j)/ \rho)} = \frac{\exp( \tilde{\delta}_j / \rho)}{\sum_{k=1}^{J} \exp( \tilde{\delta}_j/ \rho)} \equiv s_{j\mid A}
\end{equation}
implying that, conditional on buying an annuity the health information does not affect the insurer choice. 

\todo[inline]{Note that although there is no selection (equation \ref{eq:no_selection}) the marginal costs can still be different among the firms, see equation \ref{eq:mg_cost}. Because the diversion from the outside option is different. }

Moreover 
\begin{align}\label{eq:into_selection}
    s_A(h_i)
    %= \frac{\alpha_A\left(\sum_{j=1}^J \exp((f(h_i) + \tilde{\delta}_j)/\rho)\right)^\rho }{\alpha_A\left(\sum_{j=1}^J \exp((f(h_i) + \tilde{\delta}_j)/\rho)\right)^\rho +  \exp \left(\delta_0 \right)} \\
    = \frac{\alpha_A\exp(f(h_i)/\delta)^\rho\left(\sum_{j=1}^J \exp(\tilde{\delta}_j/\rho)\right)^\rho }{\alpha_A\exp(f(h_i)/\delta)^\rho\left(\sum_{j=1}^J \exp(\tilde{\delta}_j/\rho)\right)^\rho  +  \exp \left(\delta_0 \right)}
\end{align}
implying that the probability of buying an annuity is increasing on health status. 

Given that the  probability of buying an annuity is increasing on $h_i$, there is adverse selection into the annuities market. The cost of selling an annuity to an annuities buyer is on average higher than the cost of providing an annuity to a buyer choosing the outside option. For a formal proof see section \ref{sec:appendix2}.  



\subsection{Cost of the average and marginal buyers}


Define $c_j(h_i)$ the cost for insurer $j$ of selling an annuity to buyer with health status $h_i$. Then the average cost of insurer $j$ is\footnote{Remember that we are studying a particular group of the population, hence is the average cost for individuals who have the same observables. }: 

\begin{align}\label{eq:avg_cost}
    \bar{c}_j = \frac{\int c_j(h)\cdot s_j(h) \cdot f_h(h) dh}{\int  s_j(h) \cdot f_h(h) dh}
\end{align}
where $f_h(\cdot)$ is the marginal distribution of the health status. 

To obtain the marginal cost, when increasing prices we need to obtain the marginal consumer, for tractability take the case when $\tilde{\delta}_j = \hat{\delta}_j - \alpha p_j$. Then: 

\begin{align}\label{eq:price_der}
    \frac{\partial s_j(h)}{\partial p_j} 
    %= -\alpha \ s_j \left[ \left(\frac{1}{\rho}(1-s_{j|A})\right)  + s_{j|A}  (1-s_A) \right]  
    = -\alpha \ s_A(h) s_{j\mid A}(h) \left[ \left(\frac{1}{\rho}(1-s_{j|A}(h))\right)  + s_{j|A}(h)  (1-s_A(h)) \right]  
\end{align}

For the derivation see section \ref{sec:appendix1}.
Note that $\frac{1}{\rho}(1-s_{j|A}(h))$ represents the consumers that are substituting from other firms and $s_{j|A}(h)  (1-s_A(h))$ represents the consumers that are substituting from the outside option. 
If the shocks within the nest are correlated ( $\rho \rightarrow 0 $) then most of the marginal consumers are susbstituting from other firms, whereas if the outside option is very relevant $s_A \rightarrow 1 $ then most of the additional consumers come from the outside option. 

The average cost of the consumers gained with a price decrease (marginal cost) is: 
\begin{align}\label{eq:mg_cost}
    c^m_j = \frac{\int c_j(h) \frac{\partial s_j(h)}{\partial p_j}  f_h(h) dh }{\int \frac{\partial s_j(h)}{\partial p_j}  f_h(h) dh }
\end{align}

In section \ref{sec:appendix3} we prove that 
$ \bar{c}_j \geq c_j^m$, the intuition is simple, non-buyers have a lower $h_i$ and when reducing the price the marginal buyer is a linear combination of consumers diverted from other insurers and from consumers that were not buying an annuity, since the first group of consumers is not selected (see equation \ref{eq:no_selection}) and the latter group has a lower $h_i$ the marginal consumer tends to be less healthy than the average consumer. 


\subsection{Firm pricing}
The firm profits are:

\begin{align}
    \pi_j &= \int ( p_j -c(h) )s_j(h) f(h) dh  
\end{align}

Then the FOC is: 
\begin{align}\label{eq:FOC}
    \int \left[s_j(h)+ ( p_j -c(h) )\frac{\partial s_j(h)}{\partial p_j}\right] f(h) dh =0  \\
    %\int ( p_j -c(h) )\frac{\partial s_j(h)}{\partial p_j} f(h) dh = - \int s_j(h) f(h) dh \notag \\
    %\int  p_j \frac{\partial s_j(h)}{\partial p_j} f(h) dh -  \int c(h) \frac{\partial s_j(h)}{\partial p_j} f(h) dh = - \int s_j(h) f(h) dh \\
    %\int  p_j \frac{\partial s_j(h)}{\partial p_j} f(h) dh - c^m_j \cdot \left(\int \frac{\partial s_j(h)}{\partial p_j}  f(h) dh \right)= - \int s_j(h) f(h) dh \\
    %[ p_j- c^m_j] \int  \frac{\partial s_j(h)}{\partial p_j} f(h) dh = - \int s_j(h) f(h) dh \\
    p_j- c^m_j  = - \frac{\int s_j(h) f(h) dh}{\int  \frac{\partial s_j(h)}{\partial p_j} f(h) dh} = - \frac{ s_j}{\int  \frac{\partial s_j(h)}{\partial p_j} f(h) dh} = \frac{1}{\bar{\eta}_j}
\end{align}
where we used the definition of equation \ref{eq:mg_cost} and we defined $\bar{\eta}_j \equiv -\int \frac{\partial s_j(h)}{\partial p_j}f(j)dh / \int s_j(h) f(h)dh$  the average own-price elasticity. 

As in standard FOC conditions, the margin equals the ratio of market share over a derivative term, but in this particular case the price derivative is the average derivative over health status. 

Using equation \ref{eq:price_der}, and defining the competition weight: 

\begin{align}\label{eq:comp_weight}
    w(h) \equiv s_A(h)s_{j|A}(h)\left[\frac{1}{\rho}(1-s_{j|A}(h)) + s_{j|A}(h)(1-s_A(h))\right]   = -\frac{1}{\alpha} \frac{\partial s_j(h)}{\partial p_j} 
\end{align}

Then we have: 
\begin{align}
    \int \frac{\partial s_j(h)}{\partial p_j} f(h) dh = - \alpha \int w(h) f(h) dh
\end{align}
Then the margin can be written as: 
\begin{align}
    p_j- c^m_j  = \frac{ s_j}{\alpha \int w(h) f(h) dh} = 
\end{align}


The highest the market share the higher the margin since raising prices increases profits by increasing revenue from more infra-marginal consumers. 

The higher the competitive pressure the lower the margin, where the competitive pressure is given by the magnitude of marginal consumers. 
For example the smaller the $\rho$ the stronger the correlation of idiosyncratic shocks making competition stronger and decreasing margins.. Or the higher $s_{j\mid A} s_0$ the more likely a consumer switches to the outside option, also decreasing margins. 






Using the FOC and the functional forms previously assumed we can obtain the share of consumers who do not buy an annuity and is given by: 

\begin{align}\label{eq:outside_share2}
    s_0 = \int s_0(h) f_h(h) dh
\end{align}

where: \begin{align}
    s_0(h) = \frac{1}{\alpha_A e^{f(h) -\delta_0}\left[\sum_{j=1}^{J}e^{(\hat{\delta}_j  - \alpha p_j)/\rho}\right]^\rho + 1 } 
\end{align}

For the steps see section \ref{sec:appendix4}. 


\subsection{Effect of increasing information}

In the model we presented, $h_i$ is the private information of the buyer. Providing more information to the sellers implies reducing the role of private information, which we interpret as a decrease in the variance of $h_i$. Assuming that $h_i$ is normally distributed with mean $0$ and standard deviation $\sigma$, we can use our previous results to study the impact of information in various outcomes.

\begin{itemize}
    \item From equation \ref{eq:outside_share2} we can calculate the market share of the outside product and determine what happens when we reduce the role of private information. 

    \item From equations \ref{eq:avg_cost} and \ref{eq:mg_cost} we can calculate the average cost and the marginal cost as a function of $\sigma$ and from them determine the degree of adverse selection. 

    \item 
\end{itemize}


Note that given that we do not have a close form solution for the equilibrium prices, and as a consequence neither for the market shares and costs, we will perform simulations. In the next section we present the functional forms used and the results obtained. 




\section{Simulation}

For the simulations we assume that the cost function of the firm is the same for all of them and is given by: 

$c(h) = c_0 + \gamma \cdot h$

\newpage
 

 \section{Appendix}

\subsection{Own elasticity in nested logit}\label{sec:appendix1}
From equation \ref{eq:choice_prob} we derive the derivative of the probability with respect to price. 

First we have: 

\begin{align}\label{eq:a1}
\frac{\partial s_{j\mid A}}{\partial \delta_j}
%&=\frac{(1/\rho)\cdot e^{\delta_j/\rho}\cdot \sum_{k\in A} e^{\delta_k/\rho}-e^{\delta_j/\rho}\cdot (1/\rho)\cdot e^{\delta_j/\rho}}{\left(\sum_{k\in A} e^{\delta_k/\rho}\right)^2} \\
%&=\frac{1}{\rho}\left(\frac{e^{\delta_j/\rho}}{\sum_{k\in A} e^{\delta_k/\rho}}\right)\frac{  \sum_{k\in A} e^{\delta_k/\rho}-  e^{\delta_j/\rho}}{\sum_{k\in A} e^{\delta_k/\rho}} \\
%&=\frac{1}{\rho}\left(\frac{e^{\delta_j/\rho}}{\sum_{k\in A} e^{\delta_k/\rho}}\right)\left(1-\frac{    e^{\delta_j/\rho}}{\sum_{k\in A} e^{\delta_k/\rho}} \right)\\
&=\frac{1}{\rho}\,s_{j\mid A}\big(1 - s_{j\mid A}\big), 
\end{align}

Define $D= \sum_{k\in A} e^{\delta_k/\rho}$ then $\frac{\partial D}{\partial \delta_j} =\frac{1}{\rho} e^{\delta_j/\rho} $

\begin{align}\label{eq:a2}
\frac{\partial s_A}{\partial \delta_j} &=
\frac{\partial}{\partial \delta_j}\left(\frac{\alpha_A D^\rho}{\alpha_A D^\rho + e^{\delta_0}}  \right) 
= \frac{\alpha_A \rho D^{\rho-1} D' (\alpha_A D^\rho + e^{\delta_0})-\alpha_A \rho D^{\rho-1} D' (\alpha_A D^\rho)}{(\alpha_A D^\rho + e^{\delta_0})^2} \notag \\ 
%=\alpha_A \cdot   \rho  \cdot D^{\rho-1} \cdot D'\frac{   (\alpha_A D^\rho + e^{\delta_0})- (\alpha_A D^\rho)}{(\alpha_A D^\rho + e^{\delta_0})^2}  \\
%=\rho  \cdot D^{-1} \cdot D'\frac{ e^{\delta_0}}{\alpha_A D^\rho + e^{\delta_0}} \frac{\alpha_AD^{\rho}}{\alpha_A D^\rho + e^{\delta_0}} \\
%=\rho  \cdot D^{-1} \cdot D'\frac{ e^{\delta_0}}{\alpha_A D^\rho + e^{\delta_0}} s_A  \\
%= e^{\delta_j/\rho} \cdot D^{-1}  \frac{ e^{\delta_0}}{\alpha_A D^\rho + e^{\delta_0}} s_A  \\
%= s_{j\mid A}  \frac{ e^{\delta_0}}{\alpha_A D^\rho + e^{\delta_0}} s_A  \\
&= s_{j\mid A}  (1-s_A)s_A 
\end{align}


Assuming $\delta_j = u_j - \alpha p_j, $. We can use the chain rule and equations \ref{eq:a1} and \ref{eq:a2} to obtain the derivatives of market shares with respect to prices: 

\begin{align}\label{eq:a3}
     \frac{\partial s_{j|A}}{\partial p_j} = \frac{\partial s_{j|A}}{\partial \delta_j} \cdot \frac{\partial \delta_j}{\partial p_j} = \left(\frac{1}{\rho}s_{j|A}(1-s_{j|A})\right) \cdot (-\alpha) = -\frac{\alpha}{\rho}s_{j|A}(1-s_{j|A})
\end{align}
\begin{align}\label{eq:a4}
    \frac{\partial s_A}{\partial p_j} = \frac{\partial s_A}{\partial \delta_j} \cdot \frac{\partial \delta_j}{\partial p_j} = \frac{\partial s_A}{\partial p_j} = (s_A(1-s_A)s_{j|A}) \cdot (-\alpha) = -\alpha s_A(1-s_A)s_{j|A}    
\end{align}

Finally 
\begin{align}\label{eq:a5}
    \frac{\partial s_j}{\partial p_j} &= \frac{\partial s_{j|A}}{\partial p_j} \cdot s_A + s_{j|A} \cdot \frac{\partial s_A}{\partial p_j} \notag \\
    %&= \left(-\frac{\alpha}{\rho}s_{j|A}(1-s_{j|A})\right) \cdot s_A + s_{j|A} \cdot \left(-\alpha s_A(1-s_A)s_{j|A}\right) \notag \\
    &= -\alpha \cdot s_A \cdot s_{j|A}\left[ \left(\frac{1}{\rho}(1-s_{j|A})\right)  + s_{j|A}  (1-s_A) \right] \notag \\
    &= -\alpha \ s_j \left[ \left(\frac{1}{\rho}(1-s_{j|A})\right)  + s_{j|A}  (1-s_A) \right]  
\end{align}
with a marginal decrease in prices 
note that when decreasing prices the increase due consumers diverting from other insurers is: 

\begin{align}\label{eq:a6}
    \alpha \ s_j  \left(\frac{1}{\rho}(1-s_{j|A})\right)    
\end{align}
and the buyers diverted from the outside option are: 
\begin{align}\label{eq:a7}
    \alpha \ s_j  s_{j|A}  (1-s_A)  
\end{align}

\pagebreak


\subsection{Cross elasticity in nested logit}\label{sec:appendix1.1}


Differentiate  equation \ref{eq:choice_prob} with respect to $p_k$: 

\begin{align}\label{eq:app1}
    \frac{\partial s_j}{\partial p_k} = \frac{\partial s_{j|A}}{\partial p_k}s_A + \frac{\partial s_j}{\partial s_A}\frac{\partial s_A}{\partial p_k}, 
\end{align}



Write $D \equiv \sum_{j \in A} e^{\delta_j/\rho}$. Then  $\partial D/\partial\delta_k = (1/\rho)e^{\delta_k/\rho}$. Since $s_{j|A} = e^{\delta_j/\rho}/D$  we have: 

\begin{align}
    \frac{\partial s_{j|A}}{\partial \delta_k} %=  \frac{-(\partial D/\partial\delta_k) e^{\delta_j/\rho}}{D^2} 
    %= \frac{-( (1/\rho)e^{\delta_k/\rho}) e^{\delta_j/\rho}}{D^2} \\
    %=-\frac{1}{\rho} \frac{e^{\delta_k/\rho} e^{\delta_j/\rho}}{D^2} \\
    =-\frac{1}{\rho}s_{j|A}s_{k|A}
\end{align}


For the outer share, from equation \ref{eq:outer_nest} we have: 

\begin{align}
    \frac{\partial s_A}{\partial\delta_k} = s_A(1-s_A)s_{k|A}
\end{align}

Assuming $\delta_k = u_k - \alpha p_k$, we have $\frac{\partial\delta_k}{\partial p_k} = -\alpha$. Thus: 
\begin{align}\label{eq:app4}
    \frac{\partial s_{j|A}}{\partial p_k} = \frac{\partial s_{j|A}}{\partial \delta_k}\frac{\partial \delta_k}{\partial p_k} = \left(-\frac{1}{\rho}s_{j|A}s_{k|A}\right)(-\alpha) = \frac{\alpha}{\rho}s_{j|A}s_{k|A}.
\end{align}

\begin{align}\label{eq:app5}
    \frac{\partial s_A}{\partial p_k} = \frac{\partial s_A}{\partial \delta_k}\frac{\partial \delta_k}{\partial p_k} = (s_A(1-s_A)s_{k|A})(-\alpha) = -\alpha s_A(1-s_A)s_{k|A}    
\end{align}

Pluging equations \ref{eq:app4} and \ref{eq:app5} into \ref{eq:app1} we have: 


\begin{align}\label{eq:app6}
    \frac{\partial s_j}{\partial p_k} 
    %&=\left[\frac{\alpha}{\rho}s_{j|A}s_{k|A}.\right]s_A - s_{j\mid A} \alpha s_A(1-s_A)s_{k|A} \\
    &= \alpha s_A s_{j|A}s_{k|A}\left(\frac{1}{\rho}  - (1-s_A)\right)
\end{align}




\subsection{Cost difference}\label{sec:appendix2}

From equation \ref{eq:into_selection} 
$s_A(h)$ is strictly increasing in $h$.  By Bayes' rule the conditional densities are
    $$f_A(h) = \frac{s_A(h)f(h)}{s_A}, \quad f_0(h) = \frac{(1-s_A(h))f(h)}{s_0}.$$
Consider the likelihood-ratio
    $$R(h) = \frac{f_A(h)}{f_0(h)} = \frac{s_0}{s_A} \cdot \frac{s_A(h)}{1-s_A(h)}.$$

Since $s_A(h)$ is strictly increasing in $h$ then $R(h)$ also is. 
Given that $f_A/f_0$ exhibits the monotone likelihood ratio (MLR) property, then $f_A$ exhibits first order stochastic dominance (FOSD) with respect to $f_0$\footnote{See the \href{https://en.wikipedia.org/wiki/Monotone_likelihood_ratio?utm_source=chatgpt.com}{here}}. FOSD implies that\footnote{See \href{https://economics.stackexchange.com/questions/15134/implication-of-first-order-stochastic-dominance?utm_source=chatgpt.com}{here} for a proof that FOSD implies an inequality of expected values. }: 

$$E[c(H)|A] \geq E[c(H)|0],$$

\subsection{Cost difference(1)}\label{sec:appendix2.1}


Assume that $c_j(h) = c(h), \forall j$ then, for any firm $j$: 
\begin{align}
\bar{c}_j = \frac{\int c(h)s_j(h)f(h)\,dh}{\int s_j(h)f(h)\,dh} = \frac{\int c(h)s_A(h) s_{j\mid A}(h)f(h)\,dh}{\int s_A(h) s_{j\mid A}(h)f(h)\,dh} 
=  \frac{\int c(h)s_A(h) f(h)\,dh}{\int s_A(h) f(h)\,dh} = \bar{c}_A    
\end{align}
where the last equality uses equation \ref{eq:no_selection}. 
 

\subsection{Cost difference (2)}\label{sec:appendix3}

    

using equation \ref{eq:price_der} in equation \ref{eq:avg_cost} we have: 
\begin{align}
    c^m_j = \frac{\int c_j(h) \left[\alpha \ s_A(h) s_{j\mid A}(h) \left[ \left(\frac{1}{\rho}(1-s_{j|A}(h))\right)  + s_{j|A}(h)  (1-s_A(h)) \right]  \right]  f_h(h) dh }{\int  \left[\alpha \ s_A(h) s_{j\mid A}(h) \left[ \left(\frac{1}{\rho}(1-s_{j|A}(h))\right)  + s_{j|A}(h)  (1-s_A(h)) \right]  \right]  f_h(h) dh }
\end{align}


using equation \ref{eq:no_selection} we remove the argument of $s_{j\mid A}(h)$: 
\begin{align}
    c^m_j 
    &= \frac{\int c_j(h) \left[\alpha \ s_A(h) s_{j\mid A} \left[ \left(\frac{1}{\rho}(1-s_{j|A})\right)  + s_{j|A}  (1-s_A(h)) \right]  \right]  f_h(h) dh }{\int  \left[\alpha \ s_A(h) s_{j\mid A} \left[ \left(\frac{1}{\rho}(1-s_{j|A})\right)  + s_{j|A}  (1-s_A(h)) \right]  \right]  f_h(h) dh } 
\end{align}
where the numerator is:
\begin{align}\label{eq:numerator}
  \int c_j(h) \left[\alpha \ s_A(h) s_{j\mid A} \frac{1}{\rho}(1-s_{j|A}) \right]  f_h(h) dh +\int c_j(h)\alpha \ s_A(h) s_{j\mid A}  s_{j|A}  (1-s_A(h))  f_h(h) dh \notag \\  
   %= \alpha s_{j\mid A} \frac{1}{\rho}(1-s_{j|A})\int c_j(h)  \ s_A(h) f_h(h) dh +\alpha s_{j\mid A}^2 \int c_j(h) \ s_A(h)    (1-s_A(h))  f_h(h) dh \notag \\  
    = a \int c_j(h)  \ s_A(h) f_h(h) dh +b \int c_j(h) \ s_A(h)    (1-s_A(h))  f_h(h) dh  
\end{align}
where the last step used the definitions $a\equiv  \alpha s_{j\mid A} \frac{1}{\rho}(1-s_{j|A}), \quad b \equiv \alpha s_{j\mid A}^2 $
similarly the  denominator is: 

\begin{align}\label{eq:denominator}
    a \int s_A(h) f_h(h) dh +b \int s_A(h)    (1-s_A(h))  f_h(h) dh  \equiv W
\end{align}
replacing equations \ref{eq:numerator} and \ref{eq:denominator}, we have the marginal cost being: 

\begin{align}\label{eq:mg_cost2}
    c^m_j 
    &= \frac{a \int c_j(h)  \ s_A(h) f_h(h) dh +b \int c_j(h) \ s_A(h)    (1-s_A(h))  f_h(h) dh  }{a \int s_A(h) f_h(h) dh +b \int s_A(h)    (1-s_A(h))  f_h(h) dh }
\end{align}


Define

\begin{align}
X =   \int c_j(h)  \ s_A(h) f_h(h) dh \quad 
A =   \ s_A(h) f_h(h) dh \notag \\
Y =   \int c_j(h)  \ s_A(h) (1-s_A(h)) f_h(h) dh \quad 
B =   \int s_A(h) (1-s_A(h)) f_h(h) dh \notag 
\end{align}
 replacing in the marginal cost (equation \ref{eq:mg_cost2}) we have: 

 \begin{align}
    c^m_j 
    &= \frac{a X +b Y }{aA + bB } = \frac{aA\frac{X}{A}+bB\frac{Y}{B}}{aA+bB} = \frac{aA}{aA+bB}\cdot\frac{X}{A} + \frac{bB}{aA+bB}\cdot\frac{Y}{B}.
\end{align}


Define
$$ \lambda := \frac{aA}{aA+bB} \in [0,1], \quad 1-\lambda = \frac{bB}{aA+bB} \in [0,1].$$


 \begin{align}
    c^m_j = \lambda \frac{X}{A} + (1-\lambda)\frac{Y}{B}.
\end{align}

Now we need to prove that $\frac{Y}{B} \le \frac{X}{A}$. 


Introduce the probability measure $d\nu(h) := \frac{s_A(h)f(h)}{A}\,dh$.
Let $r(h) := 1-s_A(h)$. Then
$$E_{\nu}[c_j] = \frac{X}{A}, \quad E_{\nu}[r] = \frac{B}{A}, \quad E_{\nu}[c_j r] = \frac{Y}{A}.$$
Hence
 
$$ \frac{Y}{B} = \frac{E_{\nu}[c_j r]}{E_{\nu}[r]} = \frac{E_{\nu}[c_j]E_{\nu}[r] + Cov_{\nu}(c_j,r)}{E_{\nu}[r]} = E_{\nu}[c_j] + \frac{Cov_{\nu}(c_j,r)}{E_{\nu}[r]} \le E_{\nu}[c_j] = \frac{X}{A},$$
since $c_j$ is (weakly) increasing in $h$ while $r(h)=1-s_A(h)$ is (weakly) decreasing in $h$, implying $Cov_{\nu}(c_j,r) \le 0$. The inequality is strict if both are nonconstant on sets of positive measure.

then we have: 
 \begin{align}
    c^m_j = \lambda \frac{X}{A} + (1-\lambda)\frac{Y}{B} \leq  \lambda \frac{X}{A} + (1-\lambda)\frac{X}{A} = \frac{X}{A} =\bar{c}_j 
\end{align}

\subsection{Outside share}\label{sec:appendix4}

From equation (\ref{eq:outer_nest}) we have: 

\begin{equation}\label{eq:outside_share}
    s_0(h) = \frac{\exp(\delta_0)}{\alpha_A\left(\sum_{j=1}^J \exp(\delta_j/\rho)\right)^\rho +  \exp \left(\delta_0 \right)}
\end{equation}
Assuming  $\delta_j(h) = f(h) + \bar{\delta}_j$, and defining the within nest inclusive value: 

\[
I_A(h) \equiv \ln \sum_{j=1}^{J} e^{\delta_j(h)/\rho} = \frac{f(h)}{\rho} + \ln \sum_{j=1}^{J} e^{\bar{\delta}_j/\rho}.
\]
Then, replacing in equation \ref{eq:outside_share}, we can write: 


\begin{align}
    s_0(h) %&=\frac{\exp(\delta_0)}{\alpha_A\left(\sum_{j=1}^J \exp(\delta_j/\rho)\right)^\rho +  \exp \left(\delta_0 \right)} \notag \\
    %&= \frac{\exp(\delta_0)}{\alpha_A\left(\exp\{\log(\sum_{j=1}^J \exp(\delta_j/\rho))\}\right)^\rho +  \exp \left(\delta_0 \right)} \notag \\
    %&= \frac{\exp(\delta_0)}{\alpha_A\left(\exp\{I_A(h)\}\right)^\rho +  \exp \left(\delta_0 \right)} \notag \\
    &= \frac{1}{\alpha_A\exp\{I_A(h)\rho - \delta_0\} + 1 } 
\end{align}
replacing the functional form of the inclusive value and parametrizing  $\tilde{\delta}_j = \hat{\delta}_j - \alpha p_j$: 

\begin{align}
    s_0(h) %&= \frac{1}{\alpha_A\exp\{\left[ \frac{f(h)}{\rho} + \ln \sum_{j=1}^{J} e^{\bar{\delta}_j/\rho}\right]\rho - \delta_0\} + 1 } \notag \\
    %&= \frac{1}{\alpha_A\exp\{\left[ f(h) + \rho\ln \sum_{j=1}^{J} e^{\bar{\delta}_j/\rho}\right] - \delta_0\} + 1 } \notag \\
    %&= \frac{1}{\alpha_A e^{f(h) -\delta_0}\exp\{\rho\ln \sum_{j=1}^{J} e^{\bar{\delta}_j/\rho}\} + 1 } \notag \\
    %&= \frac{1}{\alpha_A e^{f(h) -\delta_0}\left[\sum_{j=1}^{J}e^{\bar{\delta}_j/\rho}\right]^\rho + 1 } \notag \\
    &= \frac{1}{\alpha_A e^{f(h) -\delta_0}\left[\sum_{j=1}^{J}e^{(\hat{\delta}_j  - \alpha p_j)/\rho}\right]^\rho + 1 } 
\end{align}


Given the outside share for a given health status we can obtain the outside share in the general popoulation using

\begin{align}
    s_0 = \int s_0(h) f_h(h) dh
\end{align}

 
\end{document}