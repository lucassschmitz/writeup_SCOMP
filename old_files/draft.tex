\documentclass[12pt]{article}
%%%%%%%%%%%%%%%%%%%%%%%%%%%%%%%%%%%%%%%%%%%%%%%%%%%%%%%%%%%%%%%%%%%%%%%%%%%%%%%%%%%%%%%%%%%%%%%%%%%%%%%%%%%%%%%%%%%%%%%%%%%%%%%%%%%%%%%%%%%%%%%%%%%%%%%%%%%%%%%%%%%%%%%%%%%%%%%%%%%%%%%%%%%%%%%%%%%%%%%%%%%%%%%%%%%%%%%%%%%%%%%%%%%%%%%%%%%%%%%%%%%%%%%%%%%%
\usepackage{amsfonts}
\usepackage{eurosym}
\usepackage{geometry}
\usepackage{amsmath,amsthm,amssymb}
\usepackage{graphicx}
\usepackage{comment}
%\usepackage[sort,comma]{natbib}
\usepackage[backend=biber, style = apa]{biblatex}
\usepackage{placeins} % to separate sections

\usepackage{adjustbox}
\usepackage{array}
\usepackage{multirow}
\usepackage{graphicx}
\usepackage{subcaption}
\usepackage{pifont}
\usepackage{amssymb}
\usepackage{comment}
\usepackage[utf8]{inputenc}
\usepackage{setspace}
\usepackage[hang, flushmargin, bottom]{footmisc}
\usepackage{footnotebackref}
\usepackage{xcolor}
\usepackage{hyperref}
\usepackage{booktabs}
\usepackage{pifont}
\usepackage{caption}
\usepackage{float}
\usepackage{todonotes}
\setcounter{MaxMatrixCols}{10}
%TCIDATA{OutputFilter=LATEX.DLL}
%TCIDATA{Version=5.50.0.2960}
%TCIDATA{<META NAME="SaveForMode" CONTENT="1">}
%TCIDATA{BibliographyScheme=BibTeX}
%TCIDATA{LastRevised=Sunday, April 28, 2024 18:12:38}
%TCIDATA{<META NAME="GraphicsSave" CONTENT="32">}
%TCIDATA{Language=American English}

%\setlength{\bibsep}{0.3pt}
\setlength{\textfloatsep}{5pt}
\hypersetup{breaklinks=true,hypertexnames=false,colorlinks=true,citecolor = teal}
\captionsetup{font=normalsize}
\newcommand{\cmark}{\ding{51}}
\def\sym#1{\ifmmode^{#1}\else\(^{#1}\)\fi}
\renewcommand{\thetable}{\Roman{table}}
\geometry{verbose,tmargin=1.252in,bmargin=1.252in,lmargin=1.2in,rmargin=1.2in,nomarginpar}
\makeatletter
\DeclareTextSymbolDefault{\textquotedbl}{T1}
\theoremstyle{plain}
\newtheorem{thm}{\protect\theoremname}
\theoremstyle{plain}
\newtheorem{prop}[thm]{\protect\propositionname}
\providecommand{\propositionname}{Proposition}
\providecommand{\theoremname}{Theorem}
\makeatother
\providecommand{\propositionname}{Proposition}
\providecommand{\theoremname}{Theorem}
\newtheorem{ass}[thm]{Assumption}
% \input{tcilatex}
\usepackage{tikz}
\usetikzlibrary{shapes.geometric, arrows, positioning}

\tikzstyle{startstop} = [rectangle, rounded corners, minimum width=3cm, minimum height=1cm, text centered, draw=black, fill=blue!30]
\tikzstyle{process} = [rectangle, minimum width=3cm, minimum height=1cm, text centered, draw=black, fill=orange!30]
\tikzstyle{decision} = [rectangle, minimum width=3.5cm, minimum height=1cm, text centered, draw=black, fill=red!30]
\tikzstyle{mechanism} = [rectangle, minimum width=3cm, minimum height=1cm, text centered, draw=black, fill=green!30]
\tikzstyle{arrow} = [thick,->,>=stealth]



\addbibresource{references.bib}
\begin{document}
 

\newpage
 \title{{\Large Centralized annuities marketplace}}
\author{Lucas Schmitz\thanks{Yale University \texttt{lucas.schmitz@yale.edu}}} 
\date{}
\maketitle


%

%\begin{abstract}
%\baselineskip0.5cm Here goes the abstract.
%\end{abstract}

%\thispagestyle{empty}

\vspace{3cm}

%\newpage \onehalfspacing
%\setcounter{page}{1}







%%%%%%%%%%%%%%%%%%%%%%%%%%%%%%%%%
\section{Data}
There are already a couple of papers that have studied this market (\cite{boehm_intermediation_2024,illanes_retirement_2019,alcalde_intermediary_nodate} the data sources we could use are: 

\begin{itemize}
    \item The first one is the centralized offer exchange database SCOMP, available from the regulating agencies, which contains all retirees from August 2004 until July 2020. This database includes basic demographic information about retirees – age, gender, and legal dependents – total savings, geographic location at the city/precinct level. The data also record every offer received by each potential retiree, information about intermediation, pension product accepted, and commission paid. Finally, I also observe the date of death if it occurs before July 2021. I complement this information with publicly available reports on insurance companies’ risk ratings, information about the number of intermediaries, and their registered locations. For a subset of the independent advisors, I also link scores obtained in the knowledge tests required for their certification after 2017. For most of the analysis, I restrict the sample to individuals retiring between 2010 and 2018, at or after legal age, and without legal dependents other than a spouse. This sample selection yields approx. 150,000 observations.

    \item Administrative database of Pension Histories (HPA) from the Superintendence of Pensions.(\cite{halcartegaray_efectos_2011}) 
    
    \item Aggregate time series data on annuities and programmed withdrawals for normal old age and early retirement, 1983-2002, were obtained from the insurance regulator and the AFP regulator (\cite{james_pensiones_2005}.
    
    \item Individual-level data on all annuitants giving gender, size of pensions and dates of birth, retirement and death. (Unfortunately, comparable data on programmed withdrawal pensioners were not available) (\cite{james_pensiones_2005}). 
    
    \item Annuity quotes from several companies for 2003 and 1999, from which we calculated money’s worth ratios (\cite{james_pensiones_2005}). 


\item \textbf{Surveys: } The firs one is the Social Protection Survey, a representative panel survey conducted by the Department of Social Protection in Chile. Individuals are periodically interviewed on work history, education, health, income, wealth, and information regarding social security, pensions, and their knowledge of the system. The second survey conducted as part of the choice-architecture experiment of Duch et al. (2021), who elicit information about soon-to-be retirees’ income, education level, financial literacy, risk preferences, and their plans for retirement. The survey also asks about preferences for different pension products after conducting an information intervention about their characteristics

\item \textcolor{red}{What is the data that we have from before the SCOMP? }

\end{itemize}

The great advantage of the setting and associated data is that 1. we observe the non-selected offers and 2. the consumer observes all the offers. If the market is not centralized the consumer might not request a quote from every single insurer, whereas here they request one quote and all the insurers bid, and also even when the consumer requests multiple quotes, if the data only includes realized transactions, the quotes will not be observed by the econometrician. Whereas most models of insurance markets (Akerloff 70 \& Roth. and Stiglitz 76) model a perfectly competitive insurance market we allow for differentiation. 


\section{Motivation}

\begin{itemize}
    \item To increase the annuitization rate \textcite{benartzi_annuitization_2011} propose: 
    \begin{enumerate}
        \item  An alternative strategy would be to make the annuity products safer via a government guarantee along the lines of the Federal Deposit Insurance Corporation  for banks and the Pension Benefit Guaranty Corporation for pension plans.   
        \item " increasing the supply of easy-to-find annuity options for those of retirement age with 401(k) and other defined contribution plans.  The goal should be to emulate the progress that has taken place during the last two decades in the design of plans for the accumulation taken place during the last two decades in the design of plans for the accumulation phase: specifically, the widespread adoption of “automatic” features, including automatic enrollment, automatic escalation, and default investment strategies such as “target date funds” that rebalance a portfolio for a decreasing level riskiness as the participant ages."
        
    \end{enumerate}
    with respect to the first proposal, our setting allows us to directly estimate the increase in consumer welfare that a government guarantee would generate and the impact on the annuitization rate. NOT CLEAR WHAT DO WE HAVE TO SAY WRT THE SECOND PROPOSAL, MAYBE WE COULD SAY SOMETHING ELSE IF WE GET DATA FROM THE SYSTEM PRIOR TO SCOMP. 
    
\end{itemize}

Other aspects that are interesting of our context is that in standard models of insurance markets firms compet either in price (Akerlof 70) or price and coverage (Rothschild and Stiglitz 76). In our case consumers by choosing the number of guaranteed periods can themselves determine something akin to the coverage, which changes the way firms compete.  


There are a couple of other markets where sellers post a price and then the buyer can bargain with the seller, e.g. car dealerships and the housing markets where if you go to Zillow you have a price but you can always contact the seller to get a better price. The difference is that in those markets the buyer has less information than the seller whereas in our case is the opposite.  

\section{Research questions}

\begin{itemize}
    \item One of the big questions is why is it that people do not buy more annuities. But one has to consider they are not backed, there is a risk the insurance company goes bankrupt\footnote{The government reinsures a minimum pension plus 75\% of the difference between the annuity payment and the minimum, up to a cap of 45 UFs. There has been only one bankruptcy since the system’s introduction in the 1980s, and that company’s annuitants received their full payments for 124 months after bankruptcy.(\cite{illanes_retirement_2019})}. This setting is particularly well suited to see how much people value the credit risk of the companies since one observes all the offers \footnote{In normal contexts might be difficult to observe the choice set of the individuals}. Here one could 1. see how much people value decreases in the insurers risk and simulate take up in a case where there is no insurer risk at all. 

    \item A classical problem in insurance markets is that adverse selection might generate unraveling (\cite{rothschild_equilibrium_1976})\footnote{Contracts in the \textcite{rothschild_equilibrium_1976} model have two dimensions: premiums and coverage, in annuities contracts are uni-dimensional. We are not sure about the implications of this distinction.}. Moreover a long-standing puzzle in finance is the low share of the population who buys annuities (\cite{modigliani_life_1986}), since under a broad range of assumptions models predict annuitization. This issue is particularly salient in the Chilean insurance market where the money worth's ratio \footnote{The Money Worth's ratio denotes the ratio of expected present value of insurance payments divided by the price of the insurance contract.} is greater than 1. A possible policy intervention is to make it mandatory to buy the annuities. There are two research questions related to this policy 1. is there adverse selection? (\cite{chiappori_testing_2000, einav_estimating_2010}) and 2. what are the welfare gains from a mandatory annuities purchase? 

    \begin{itemize}
        \item \textcite{einav_selection_2011} say "The canonical solution to the inefficiency created by adverse selection is to mandate that everyone purchase insurance."

        \item \textcite{illanes_retirement_2019} already study this issue. 
    \end{itemize}

    \item Before the 2004 reform that created SCOMP comissions reached an average of 6\% with fees for particular transactions as high as 11\% (\cite[p. 390]{morales_chilean_2017}. The 2004 reform increased competition by making the system more transparent, and at the same time it capped comissions to 2.5\% (\cite[p. 391]{morales_chilean_2017}), what were the impacts of this? Normally if you cap comissions in a market you decrease access, but probably is not the case if there is some behavioral bias. The answer could provide information of what happens when you decrease search costs\footnote{For example \textcite{cannon_price_2015} finds huge returns to searching. But is not very well published, although I would expect to see more papers showing that the estimated search costs (equivalently the returns of searching) are huge }.  Maybe SCOMP coul also be linked to the idea of bargaining vs auctions, is it better to have SCOMP (auction system) or allow the beneficiaries to bargain? 

    \begin{itemize}
        \item A solution to the annuitization puzzle proposed by \textcite{benartzi_annuitization_2011} is that people do not buy annuities because is not the default. They mention some difficulties (making a call) as the explanation of low 401k enrollment and this could also be the case with annuities\footnote{They say "\textit{Much research shows that even tiny obstacles such as the need to make a phone call or fill in a form can result in  procrastination and lack of action in a retirement savings plan. .... 
        The same issues apply to the choice of whether to annuitize. Is the low annuitization rate a reflection of underlying preferences or of features of the choice environment? This question has important practical implications, because in most retirement plans when an employee stops working, that retiree would have to shop  around actively if interested in investing some or all of the retirement plan balance in an annuity. Remember, few defined contribution plans offer annuities.  Owners of Individual Retirement Accounts (IRAs) are in the same boat; they have to seek out  annuity products as they reach retirement if they want to ensure lifetime income.}"}. By studying the impact of centrlaizing SCOMP we would be able to provide evidence to back or reject this argument. They even speak about cases where people actively choose between PW and annuity as a context where to study their claims (p. 150-152). 
    \end{itemize}

    \item A question is what are the effects of having an aftermarket where you can bargain with sellers, when you already have posted prices. There are a couple of other situations where this happen 
    \begin{enumerate}
        \item Charles mentioned in our meeting (18june25) that since centralized insurance markets are used in many places understanding whether the aftermarket is good or bad seems to be super important. Apparently in the UK you can get some particular health insurance conditional on getting an exam.
    \end{enumerate}
    
\end{itemize}

\section{Estimation}
Estimation of the demand model can be done via ML: 
\begin{enumerate}
    \item Estimate  ($\delta_{jt}, \alpha_k$) to maximize the likelihood 
    \item Use $\mathbb{E}[\delta_{jt}|x_{jt},x_{-jt}]= 0$ to estimate ($\beta, \xi_j, \xi_{jt}$)
\end{enumerate}

Moreover, if we assume that firms use the mortality tables published by the regulator to create beliefs over the remaining time, then from equation \ref{eq:FOC_simplified} we can recover the financing costs of the firms. Note that in this case identifying firm-specific financing costs -firm specific $\bar{r}_{jt}$ would be possible. 

The fixed cost is identified from the variation in choosing to apply for external offers. 






\section{Counterfactuals.}

Once we estimated the parameters, we can answer the questions that we are interested in. 

i\textbf{Government guarantee}: we assume that implementing a government guarantee is equivalent to setting the credit rating of all the firms to the maximum possible level. We can use the model to change the product characteristics of the firms and use the model to estimate the changes in consumer welfare and to simulate what would be the impact of this policy on the interest rates set by the firms. 

\textcolor{red}{SOMETHING TO THINK ABOUT: IF I WANT TO ESTIMATE THE IMPACT OF THE GOV. GUARANTEE ON THE ANNUITIZATION RATE I NEED TO PREDICT THE SHARE OF PEOPLE WHO CHOOSE A PHASED WITHDRAWAL AND HOW THIS CHANGES WITH THE POLICY, BUT THIS PEOPLE DO NOT EVEN GET OFFERS, CURRENTLY THE MODEL PREDICTS CHOICES GIVEN OFFERS, BUT THERE IS NO MODEL FOR THE DECISION OF PHASED WITHDRAWAL VS. ANNUITY.  }


\section{Random thought}
\begin{itemize}
    \item Figures \ref{fig:offer_ann1} and \ref{fig:offer_ann2} appear to be the offer for different years. In the second case the offer does not include information on the financial ratings. If there is variation in the way the offers are presented one could use the variation to estimate the importance of the financial ratings, because in the second case they are not communicated in the offers. 

    \item \textcite[p.16]{james_pensiones_2005} mentions the issue that annuitizations are exposed to interest rate risk. 
\end{itemize}


 


\newpage
\printbibliography

 
\end{document}
